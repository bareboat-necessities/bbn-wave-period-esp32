\documentclass[12pt]{article}
\usepackage{amsmath,amssymb}
\usepackage{graphicx}
\usepackage{authblk}
\usepackage{siunitx}
\usepackage{physics}
\usepackage{fullpage}

\title{Wave Direction Estimation from IMU Accelerations Using a Kalman Filter}
\author{Mikhail Grushinskiy}
\affil{Independent Researcher, 2025}

\begin{document}
\maketitle

\begin{abstract}
This paper presents a method for estimating the direction of ocean wave propagation from horizontal acceleration measurements obtained by an Inertial Measurement Unit (IMU). The method employs a Kalman filter that estimates the amplitude vector of an oscillatory signal with known frequency and derives the direction from the estimated amplitude. The algorithm operates in real time, is robust to noise, and tracks confidence based on Kalman uncertainty. This technique is particularly suited for embedded oceanographic platforms such as wave buoys or autonomous surface vehicles.
\end{abstract}

\section{Introduction}
Estimating wave direction using inertial sensors is challenging due to the noise and ambiguity in horizontal accelerations. Unlike traditional spectral methods requiring long data windows, this method operates recursively and assumes a known or slowly varying wave frequency. The core idea is to model the horizontal acceleration as a sinusoid in an unknown but constant direction and track the amplitude vector using a linear Kalman filter.

\section{Mathematical Model}

Let \( \vec{a}(t) \in \mathbb{R}^2 \) be the horizontal acceleration vector measured by the IMU at time \( t \). We assume it can be modeled as:
\[
\vec{a}(t) = \vec{A} \cos(\omega t + \phi)
\]
where:
\begin{itemize}
  \item \( \vec{A} \in \mathbb{R}^2 \) is the amplitude vector (encodes direction and magnitude)
  \item \( \omega \) is the angular frequency of the wave (assumed known or externally estimated)
  \item \( \phi \) is the global phase offset
\end{itemize}

\subsection{State Space Model}
We define the state as the amplitude vector \( \vec{A}_k \) at time step \( k \). The state evolves as:
\[
\vec{A}_k = \vec{A}_{k-1} + \vec{w}_k, \quad \vec{w}_k \sim \mathcal{N}(0, \mathbf{Q})
\]
The observation at time step \( k \) is:
\[
\vec{z}_k = \vec{a}_k = \vec{A}_k \cos(\theta_k) + \vec{v}_k, \quad \vec{v}_k \sim \mathcal{N}(0, \mathbf{R})
\]
where \( \theta_k = \omega t_k + \phi \) is the known phase, updated incrementally each timestep.

The observation matrix is:
\[
\mathbf{H}_k = \cos(\theta_k) \cdot \mathbf{I}_2
\]

\section{Kalman Filter Equations}

At each timestep, the filter performs the standard Kalman update:
\begin{align*}
\text{Predict:} \quad & \vec{A}_k^- = \vec{A}_{k-1} \\
& \mathbf{P}_k^- = \mathbf{P}_{k-1} + \mathbf{Q} \\
\text{Update:} \quad & \mathbf{K}_k = \mathbf{P}_k^- \mathbf{H}_k^\top \left( \mathbf{H}_k \mathbf{P}_k^- \mathbf{H}_k^\top + \mathbf{R} \right)^{-1} \\
& \vec{A}_k = \vec{A}_k^- + \mathbf{K}_k \left( \vec{z}_k - \mathbf{H}_k \vec{A}_k^- \right) \\
& \mathbf{P}_k = (\mathbf{I} - \mathbf{K}_k \mathbf{H}_k) \mathbf{P}_k^-
\end{align*}

The instantaneous direction estimate is:
\[
\hat{\vec{d}}_k = \frac{\vec{A}_k}{\norm{\vec{A}_k}}
\]

To prevent direction flipping, the estimate is smoothed and stabilized over time using a weighted average with the previous stable direction.

\section{Implementation Details}

\subsection{Phase Tracking}
The phase \( \theta_k \) is updated as:
\[
\theta_k = \text{remainder}(\theta_{k-1} + \omega \Delta t, 2\pi)
\]
where \( \omega \) is the current angular frequency estimate, and \( \Delta t \) is the sampling period.

\subsection{Directional Stability}
To prevent instability near low amplitudes or during noise bursts, a confidence metric is introduced:
\[
\text{confidence}_k = \frac{1}{\operatorname{trace}(\mathbf{P}_k) + \varepsilon}
\]
If the amplitude \( \norm{\vec{A}_k} \) is below a threshold or the confidence is low, the direction estimate is frozen.

\section{Outputs}

The filter exposes the following quantities:
\begin{itemize}
  \item \textbf{Direction (unit vector)}: \( \hat{\vec{d}}_k \)
  \item \textbf{Direction in degrees}: \( \theta_k = \text{atan2}(d_y, d_x) \in [0^\circ, 180^\circ) \)
  \item \textbf{Amplitude vector}: \( \vec{A}_k \)
  \item \textbf{Filtered signal}: \( \vec{a}_k = \vec{A}_k \cos(\theta_k) + \vec{A}_k^\perp \sin(\theta_k) \)
  \item \textbf{Oscillation component}: \( \vec{A}_k \cos(\theta_k) \)
\end{itemize}

\section{Simulation Results}

In tests using simulated wave signals with Gaussian noise, the method was able to accurately track wave direction with sub-degree precision once the confidence stabilized. The algorithm is robust to amplitude modulation, direction flipping, and moderate noise.

\section{Conclusion}

We presented a computationally efficient, real-time method for estimating ocean wave direction from horizontal accelerometer measurements. It relies on a Kalman filter with known-frequency sinusoidal excitation and robust confidence-based direction stabilization. The approach is directly suitable for embedded applications and can be combined with frequency or heave estimation modules for full wave characterization.

\section*{Acknowledgements}
Implementation and design by Mikhail Grushinskiy, 2025.

\end{document}
