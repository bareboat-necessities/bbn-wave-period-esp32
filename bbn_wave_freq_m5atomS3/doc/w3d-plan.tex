\documentclass[10pt]{extarticle}

% === Typography and structure ===
\usepackage{amsmath, amssymb, amsfonts, bm}
\usepackage{geometry, setspace}
\usepackage{hyperref}
\usepackage[utf8]{inputenc}
\usepackage[T1]{fontenc}
\usepackage{titlesec}
\usepackage{nccmath, relsize}
\usepackage{array, booktabs, makecell, adjustbox}
\usepackage{mathtools}
\usepackage{amsthm}

% Theorem environments
\newtheorem{lemma}{Lemma}
\newtheorem*{theorem*}{Theorem}

% Make inline math slightly smaller
\everymath{\smaller}

% Make display equations typeset in \small
\everydisplay\expandafter{\the\everydisplay\small}

% Reduce spacing before/after headings
\titlespacing*{\section}{0pt}{1.0ex plus .2ex}{0.6ex}
\titlespacing*{\subsection}{0pt}{0.8ex plus .2ex}{0.4ex}
\titlespacing*{\subsubsection}{0pt}{0.6ex plus .2ex}{0.3ex}

% Make section fonts smaller (like textbooks)
\titleformat{\section}{\normalfont\small\bfseries}{\thesection}{0.8em}{}
\titleformat{\subsection}{\normalfont\small\itshape}{\thesubsection}{0.8em}{}
\titleformat{\subsubsection}{\normalfont\footnotesize\itshape}{\thesubsubsection}{0.8em}{}

\geometry{letterpaper, margin=0.65in}
\setstretch{0.95}

\title{A Multiplicative EKF with Latent OU Acceleration for Drift-Robust Wave Kinematics:\\
Theory, Discretization, and Bias Compensation}
\author{Mikhail Grushinskiy}
\date{2025}

\begin{document}
\maketitle

\input{w-abstract.tex-part}

\tableofcontents

% =====================================================
\section{Introduction}
\label{sec:intro}

Accurate estimation of sea-surface motion from inertial sensors requires filtering algorithms
that can accommodate colored, correlated acceleration disturbances arising from ocean waves.
Classical Kalman filters assume white process noise, which causes bias and drift in such conditions.

\noindent \textbf{Contributions:}
\begin{enumerate}
  \item Continuous-time \emph{OU-driven kinematic model} for world acceleration with analytical discretization and stability-preserving small-step series expansion.
  \item Quaternion-Multiplicative EKF (QMEKF) unifying rotational and translational motion under a single consistent covariance framework.
  \item Block-structured decomposition enabling efficient implementation and covariance updates in $O(n_\text{att}^3+n_\text{lin}^3)$ time.
  \item Adaptive pseudo-measurement regularization $R_S \propto \sigma_a\tau^3$ with formal input-to-state stability (ISS) proof.
\end{enumerate}

\noindent \textbf{Key references:}
\cite{jazwinski1970,maybeck1979,crassidis2012,brown2012,vanloan1978integrating,sontag1996,jiang1997}


% =====================================================
\section{Coordinate Frames and Notation}
\label{sec:frames}

Define world (NED or Z-up) and body frames.
Quaternion convention: right-multiplicative error, rotation $R_{wb}(q)$.
State vector:
\[
x = [\delta\theta,\, b_g,\, v,\, p,\, S,\, a_w,\, b_a]^\top.
\]
Notation for units, dimensions, and covariance matrices.

% =====================================================
\section{Continuous-Time Process Model}
\label{sec:process}

\subsection{Attitude Kinematics and Gyroscope Model}
\[
\dot q = \tfrac12\,q\otimes[0,\,\omega_m - b_g - n_g], \qquad
\dot b_g = n_{b_g},
\]
where $n_g,n_{b_g}$ are zero-mean white noises
\cite{crassidis2012,maybeck1979}.

\subsection{OU-driven Kinematic Chain}
We model translational motion as an \emph{OU-driven kinematic chain}:
\[
\dot v = a_w,\quad
\dot p = v,\quad
\dot S = p,\quad
\dot a_w = -\tfrac{1}{\tau}a_w + \eta_a,
\]
where $a_w$ follows an Ornstein–Uhlenbeck process with stationary variance $\sigma_a^2$
and correlation time $\tau$.
This represents a hierarchy of kinematic derivatives driven by a colored acceleration process,
linking acceleration, velocity, position, and the integral of position in a continuous-time linear SDE.

The covariance density of the driving white noise $\eta_a$ is
\[
Q_c = \frac{2\sigma_a^2}{\tau}.
\]
\cite{uhlenbeck1930,brown2012}

\subsection{Accelerometer Bias and Thermal Drift}
\[
\dot b_a = n_{b_a}, \qquad b_a(T)=b_{a0}+k_a(T-T_0).
\]

\subsection{Combined Continuous SDE}
\[
\dot x = F x + G w, \quad w=[n_g,n_{b_g},\eta_a,n_{b_a}]^\top.
\]

% =====================================================
\section{Measurement Models}
\label{sec:measurements}

\subsection{Gyroscope}
\[
\omega_m = \omega_\text{true} + b_g + n_g,\quad R_g=\mathrm{diag}(\sigma_g^2).
\]

\subsection{Accelerometer}
\[
f_b = R_{wb}(a_w - g_w) + b_a + n_a,\quad R_a=\mathrm{diag}(\sigma_a^2).
\]
Jacobian derivations $J_\theta, J_{a_w}, J_{b_a}$.

\subsection{Magnetometer}
\[
m_b = R_{wb} B_w + n_m, \quad R_m=\mathrm{diag}(\sigma_m^2).
\]

\subsection{Integral Pseudo-Measurement}
\[
z_S = 0 = S + n_S, \quad R_S=\mathrm{diag}(\sigma_S^2).
\]

% =====================================================
\section{Linearity and the Matrix Exponential}
\label{sec:linearity}

For $\dot x = F x + G w$:
\[
x_{k+1} = e^{Fh}x_k + w_d,\quad
Q_d = \int_0^h e^{F(h-s)}GQ_cG^\top e^{F^\top(h-s)}ds.
\]
\cite{jazwinski1970,vanloan1978integrating}

Show OU-driven kinematic and quaternion error systems are locally LTI.
Discuss time-varying $F$ and stability of $e^{Fh}$.

% =====================================================
\section{Discretization of the OU-driven Kinematic Chain}
\label{sec:ou_discretization}

\subsection{Derivation of $\Phi_\text{axis}$}
From $e^{Fh}$ for 4-state [v,p,S,a] subsystem; derive $\phi_{va},\phi_{pa},\phi_{Sa}$.

\subsection{Derivation of $Q_{d,\text{axis}}$}
Compute covariance integral analytically; introduce $A_1,A_2$.

\subsection{Small-$h/\tau$ Maclaurin Expansions}
Series expansion up to $O(x^5)$ for numerical stability.

\subsection{3D and Correlated OU Extension}
Use Kronecker product covariance; allow cross-axis correlation $\rho_{ij}$
\cite{brown2012}.

\subsection{Summary Table}
Closed-form and series results.

% =====================================================
\section{Derivation of the Quaternion-Multiplicative EKF (QMEKF)}
\label{sec:qmekf}

\subsection{Quaternion Error Parameterization}
True vs. estimated attitude:
\[
q_t = q\otimes\delta q,\qquad
\delta q \approx [1,\,\tfrac12\,\delta\theta^\top]^\top.
\]
Linearization:
\[
\dot{\delta\theta} = -[\omega_m - b_g]_\times\delta\theta - \delta b_g - n_g.
\]
\cite{shuster1993,crassidis2012}

\subsection{Augmented State Dynamics}
\[
\dot x = F x + G w,
\]
with $F_{\theta\theta}=-[\omega_m-b_g]_\times,\;F_{\theta b_g}=-I_3$, and OU-driven kinematic chain blocks.

\subsection{Continuous Covariance Propagation}
\[
\dot P = F P + P F^\top + G Q_c G^\top.
\]

\subsection{Linearized Measurement Models}
\[
r = z - h(\hat x) \approx H x + n.
\]
Derive $H_\text{acc},H_\text{mag},H_S$.

\subsection{Discrete Prediction}
\[
x_{k|k-1} = \Phi x_{k-1|k-1},\quad
P_{k|k-1} = \Phi P_{k-1|k-1}\Phi^\top + Q_d.
\]

\subsection{Update and Quaternion Correction}
\[
K = P H^\top(HPH^\top+R)^{-1},\quad
x_{k|k}=x_{k|k-1}+K r.
\]
Quaternion update:
\[
q_{k|k}=q_{k|k-1}\otimes
\mathrm{exp}_q\!\left(\tfrac12\,\delta\theta_{k|k}\right),\quad
\delta\theta_{k|k}\!\leftarrow0.
\]
\cite{markley2003,crassidis2012}

\subsection{Connection to Implementation}
Corresponds to the core update structure in the \texttt{Kalman3D\_Wave} class.

% =====================================================
\section{Discrete Kalman Filter in Canonical Form}
\label{sec:canonical}
\smallskip
\[
\begin{aligned}
\hat x_{k|k-1}&=\Phi\hat x_{k-1|k-1},\\
P_{k|k-1}&=\Phi P_{k-1|k-1}\Phi^\top+Q_d,\\
K_k&=P_{k|k-1}H^\top(HP_{k|k-1}H^\top+R)^{-1},\\
\hat x_{k|k}&=\hat x_{k|k-1}+K_k(z_k-h(\hat x_{k|k-1})),\\
P_{k|k}&=(I-K_kH)P_{k|k-1}(I-K_kH)^\top+K_kRK_k^\top.
\end{aligned}
\]
Use Joseph form for PSD guarantee
\cite{maybeck1979,crassidis2012}.

% =====================================================
\section{From Full Matrices to Block Structure}
\label{sec:block}

\subsection{Block Partitioning}
\[
x=[x_\text{att},x_\text{lin},x_\text{bias}]^\top,\quad
P=\begin{bmatrix}
P_{aa}&P_{al}&P_{ab}\\
P_{al}^\top&P_{ll}&P_{lb}\\
P_{ab}^\top&P_{lb}^\top&P_{bb}
\end{bmatrix}.
\]

\subsection{Near-Block-Diagonal Structure}
\[
\Phi \approx \mathrm{diag}(\Phi_a,\Phi_l,I),\quad
Q_d \approx \mathrm{diag}(Q_a,Q_l,Q_b).
\]

\subsection{Block Covariance Propagation}
\[
\begin{aligned}
P_{aa}' &= \Phi_a P_{aa}\Phi_a^\top + Q_a,\\
P_{ll}' &= \Phi_l P_{ll}\Phi_l^\top + Q_l,\\
P_{al}' &= \Phi_a P_{al}\Phi_l^\top,\\
P_{bb}' &= P_{bb}+Q_b.
\end{aligned}
\]

\subsection{Joseph Update by Blocks}
Show block-wise covariance updates preserve symmetry and PSD.

\subsection{Cross-Covariance Logic}
Include correlated OU axes using Kronecker construction.

\subsection{PSD Projection}
\[
P \leftarrow \tfrac12(P+P^\top),\quad
P \leftarrow V\max(\Lambda,\epsilon I)V^\top.
\]

\subsection{Complexity Analysis}
Show cost reduction $O(N^3)\to O(n_\text{att}^3+n_\text{lin}^3)$.

% =====================================================
\section{Initialization and Alignment}
\label{sec:init}

Compute quaternion from accelerometer and magnetometer.
Yaw alignment from magnetic declination.
Initialize $P_0$ and bias covariances
\cite{crassidis2012}.

% =====================================================
\section{Adaptive Tuning Law}
\label{sec:adaptive}

Estimate $\sigma_a$ and dominant frequency $f$.
Adaptive rule:
\[
R_S = \mathrm{diag}(k_i\,\sigma_a\tau^3),\quad k_i>0.
\]
Anisotropic scaling $k_x,k_y\ll k_z$.
Adaptation schedule, warm-up, and physical motivation.

% =====================================================
\section{Stability Analysis of the Adaptive Law}
\label{sec:stability}

Define frozen-parameter linearized system.
Lyapunov function $V(e,\theta)$.
Show discrete inequality $\Delta V\le-α‖e‖^2+β‖\Delta\theta‖^2$.
Apply ISS theorem
\cite{sontag1996,jiang1997}.
Conclude boundedness and ISS of adaptive loop.

% =====================================================
\section{Implementation and Numerical Stability}
\label{sec:impl}

Discuss Maclaurin fallback for small $h/τ$,
Joseph form enforcement,
PSD projection,
LDLT fallback,
warm-up, reinitialization,
and deterministic seeds for reproducibility.

% =====================================================
\section{Results and Discussion (optional)}
\label{sec:results}

Include RMS errors, $\%H_s$ metrics, adaptation trajectories,
and analysis of drift suppression and convergence.

% =====================================================
\appendix
\section{Appendix A: Derivation of $Q_d$ Integral}
\label{app:qd_integral}

\section{Appendix B: Laplace-Domain Equivalence}
\label{app:laplace}

\section{Appendix C: Symbolic Jacobians $F,H$}
\label{app:jacobians}

\section{Appendix D: Block Propagation Algebra}
\label{app:block_algebra}

\section{Appendix E: Constants and Simulation Parameters}
\label{app:constants}


\input{w-frames.tex-part}
\input{w-model.tex-part}
\input{w-measurements.tex-part}
\input{w-linearity.tex-part}
\input{w-ou_discretization.tex-part}


\bibliographystyle{ieeetr}
\bibliography{kalman3d_refs}

\end{document}

