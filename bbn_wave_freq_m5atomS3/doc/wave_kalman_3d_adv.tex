\documentclass[11pt]{article}
\usepackage{amsmath, amssymb, amsfonts}
\usepackage{bm}
\usepackage{geometry}
\geometry{margin=1in}

\title{Mathematical Foundation of the Extended Kalman3D\_Wave Filter}
\author{Mikhail Grushinskiy \\ 2025}
\date{}

\begin{document}
\maketitle

\begin{abstract}
This document presents the mathematical formulation underlying the 
\texttt{Kalman3D\_Wave} filter, an extended multiplicative 
extended Kalman filter (MEKF) designed for ocean-wave sensing. 
The filter estimates attitude, gyroscope and accelerometer biases, 
and linear motion states by embedding a latent Ornstein--Uhlenbeck (OU) 
world-frame acceleration process, and stabilizes long-chain integrations 
using a pseudo-measurement on the third integral of displacement. 
All variables are explicitly defined for clarity.
\end{abstract}

\section{State Definition}
World frame: NED (North, East, Down).  
Quaternion $q$ represents the rotation from world to body.  

\begin{itemize}
\item $\delta\bm\theta \in \mathbb{R}^3$ — small attitude error (rad).
\item $\bm b_g \in \mathbb{R}^3$ — gyroscope bias (rad/s).
\item $\bm v \in \mathbb{R}^3$ — velocity in world frame (m/s).
\item $\bm p \in \mathbb{R}^3$ — displacement/position in world frame (m).
\item $\bm S \in \mathbb{R}^3$ — integral of displacement (m·s).
\item $\bm a_w \in \mathbb{R}^3$ — latent world acceleration (m/s$^2$).
\item $\bm b_a \in \mathbb{R}^3$ — accelerometer bias (m/s$^2$).
\end{itemize}

State vector:
\[
\bm{x} = \begin{bmatrix}
\delta\bm\theta & \bm{b}_g & \bm{v} & \bm{p} & \bm{S} & \bm{a}_w & \bm{b}_a
\end{bmatrix}^\top.
\]

Small attitude error is right-multiplicative:
\[
q^+ = q \otimes \delta q(\delta\bm\theta), \qquad 
\delta q(\delta\bm\theta) \approx [\,1, \tfrac{1}{2}\delta\bm\theta\,].
\]

\section{Process Models}
\subsection{Attitude}
Attitude kinematics:
\[
\dot q = \tfrac{1}{2}\, q \otimes \bm\omega_b, \qquad 
\bm\omega_b = \bm\omega_m - \bm{b}_g,
\]
where $\bm\omega_m$ is the measured angular velocity (rad/s).

Linearized error dynamics:
\[
\dot{\delta\bm\theta} = -[\bm\omega_b]_\times \delta\bm\theta - \delta\bm b_g + \bm{n}_g,
\]
with $[\cdot]_\times$ the skew-symmetric operator and $\bm n_g$ gyro noise.

\subsection{Linear Kinematics with OU Acceleration}
\[
\dot{\bm v} = \bm a_w, \qquad
\dot{\bm p} = \bm v, \qquad
\dot{\bm S} = \bm p.
\]

Latent acceleration $\bm a_w$ follows an Ornstein--Uhlenbeck (OU) process:
\[
\dot{\bm a}_w = -\tfrac{1}{\tau_{aw}} \bm a_w + \bm w_{aw}(t),
\]
with correlation time $\tau_{aw}$ (s), excitation noise $\bm w_{aw}(t)$, and
\[
\Sigma_c = \tfrac{2}{\tau_{aw}} \Sigma_{aw}^{\mathrm{stat}},
\]
so that $\Sigma_{aw}^{\mathrm{stat}}$ is the stationary variance of $\bm a_w$.

Stacked linear subsystem:
\[
\dot{\bm x}_{lin} = A \bm x_{lin} + G \bm w_{aw},
\]
with
\[
A =
\begin{bmatrix}
0 & 0 & 0 & I\\
I & 0 & 0 & 0\\
0 & I & 0 & 0\\
0 & 0 & 0 & -\tfrac{1}{\tau_{aw}}I
\end{bmatrix}, \quad
G = \begin{bmatrix}0\\0\\0\\I\end{bmatrix}.
\]

\section{Measurement Models}
\subsection{Accelerometer}
Predicted specific force in body frame:
\[
\bm f_b = R_{wb}(q)\,(\bm a_w - \bm g) + \bm b_a + \bm n_a,
\]
with $R_{wb}$ the world-to-body rotation, $\bm g=(0,0,g_0)^\top$, 
$g_0=9.80665$ m/s$^2$, and $\bm n_a$ accelerometer noise.

Jacobians:
\[
\frac{\partial \bm f_b}{\partial \delta\bm\theta} = -[\bm f_b]_\times, \quad
\frac{\partial \bm f_b}{\partial \bm a_w} = R_{wb}, \quad
\frac{\partial \bm f_b}{\partial \bm b_a} = I.
\]

\subsection{Magnetometer}
Predicted magnetometer measurement:
\[
\bm m_b = R_{wb}(q)\,\bm m_w + \bm n_m,
\]
with $\bm m_w$ the world-frame magnetic reference vector.

Jacobian:
\[
\frac{\partial \bm m_b}{\partial \delta\bm\theta} = -[\bm m_b]_\times.
\]

\subsection{Pseudo-measurement on $\bm S$}
\[
\bm z_S = 0 \approx H_S \bm x + \bm n_S,
\]
with $H_S$ extracting the $\bm S$ components.  
This constrains the triple integral to suppress drift.

\section{Discretization}
\subsection{Attitude Error Block via Rodrigues Formula}
Discrete transition:
\[
\delta\bm\theta_{k+1} = F_{\theta\theta}\,\delta\bm\theta_k,
\qquad
F_{\theta\theta} = \exp(-[\bm\Omega]_\times \Delta t).
\]

Let $\omega=\|\bm\Omega\|$, $\hat\omega=\bm\Omega/\omega$. Then:
\[
F_{\theta\theta} =
I - \sin(\omega\Delta t)[\hat\omega]_\times 
+ (1-\cos(\omega\Delta t))[\hat\omega]_\times^2.
\]

Small-angle expansion:
\[
F_{\theta\theta} \approx I - [\bm\Omega]_\times \Delta t + \tfrac{1}{2}[\bm\Omega]_\times^2 \Delta t^2.
\]

\subsection{Linear OU + Integrator Chain via Van Loan}
For $\dot{\bm x}=A\bm x + G w$ with covariance $\Sigma_c$:
\[
\bm x_{k+1} = \Phi \bm x_k + \eta_k, \quad 
\eta_k\sim \mathcal{N}(0,Q_d),
\]
\[
\Phi = e^{A\Delta t}, \qquad
Q_d = \int_0^{\Delta t} e^{A\tau} G \Sigma_c G^\top e^{A^\top\tau}\,d\tau.
\]

Van Loan method: form
\[
M = \begin{bmatrix}
- A\Delta t & G \Sigma_c G^\top \Delta t \\
0 & A^\top \Delta t
\end{bmatrix}, 
\]
compute $\exp(M)$, then extract
\[
\Phi = M_{22}^\top, \qquad Q_d = \Phi M_{12}.
\]

\section{Padé(6) Approximation for Matrix Exponentials}

\subsection{What is Padé(6)?}
Padé(6) is a rational approximation to the exponential:
\[
e^A \approx R_{6}(A) = \left(I - \tfrac{1}{2}A + \tfrac{1}{24}A^2 - \tfrac{1}{720}A^3\right)^{-1}
\left(I + \tfrac{1}{2}A + \tfrac{1}{24}A^2 + \tfrac{1}{720}A^3\right).
\]

\subsection{Scaling and Squaring}
To extend accuracy:
\begin{enumerate}
\item Scale: $A_s = A/2^s$ with $s$ chosen so $\|A_s\|$ small.
\item Approximate: $R_6(A_s)$.
\item Square: $(R_6(A_s))^{2^s}$.
\end{enumerate}

\subsection{Why It Is Needed}
\begin{itemize}
\item On desktop: full matrix exponential available.
\item On embedded: Padé(6) is lightweight and stable.
\item Ensures $\Phi$ is stable and $Q_d$ positive semidefinite.
\end{itemize}

\subsection{Role in the Filter}
Padé(6) is used to discretize small fixed-size blocks ($4\times 4$, $8\times 8$)
in the OU + integrator chain. This preserves stability and accuracy 
at high sampling rates.

\section{The Ornstein--Uhlenbeck (OU) Process for Wave Acceleration}

\subsection{Definition}
OU process:
\[
\dot{\bm a}_w(t) = -\tfrac{1}{\tau}\bm a_w(t) + \bm w(t).
\]
$\tau$ = correlation time (s).  
$\bm w(t)$ = white noise.  
$\Sigma_c$ = noise intensity.

\subsection{Stationary Distribution}
\[
\mathrm{Var}[\bm a_w] = \tfrac{\tau}{2}\,\Sigma_c.
\]

\subsection{Autocorrelation and Spectrum}
\[
R_{a}(\Delta) = \sigma_{aw}^2 \exp(-|\Delta|/\tau), \qquad
S_a(\omega) = \frac{2\sigma_{aw}^2 \tau}{1+(\omega\tau)^2}.
\]

\subsection{Roles of $\tau$ and $\sigma$}
\begin{itemize}
\item $\tau$: controls bandwidth ($f_c=1/(2\pi\tau)$).
\item $\sigma_{aw}$: sets stationary RMS magnitude.
\end{itemize}

\subsection{Why OU is a Solid Choice}
\begin{enumerate}
\item Stationary and bounded variance.
\item Mean-reverting, physically plausible.
\item Markov, compatible with Kalman filter.
\item Tunable by $(\tau,\sigma)$.
\end{enumerate}

\section{Pseudo-Measurement of the Third Integral $\bm S$}

\subsection{Definition}
\[
\bm z_S = 0 \approx H_S \bm x + \bm n_S,
\]
with $H_S$ extracting $\bm S$.

\subsection{Why It Is Needed}
\begin{itemize}
\item $(v,p,S)$ chain has three integrators.
\item $\bm S$ drifts unboundedly and is unobservable.
\item Anchoring $\bm S\approx 0$ prevents divergence.
\end{itemize}

\subsection{Role of $R_S$}
\begin{itemize}
\item Small $R_S$: strong suppression of drift.
\item Large $R_S$: loose constraint, more freedom.
\end{itemize}

\subsection{Kalman Interpretation}
\[
K_S = P H_S^\top (H_S P H_S^\top + R_S)^{-1}, \quad
\bm x^+ = \bm x + K_S (-\bm S).
\]

\section{Bias States: Models and Roles}

\subsection{Gyroscope Bias $\bm b_g$}
\[
\dot{\bm b}_g = \bm w_{bg}, \qquad 
\bm w_{bg} \sim \mathcal{N}(0, Q_{bg}).
\]
Needed to prevent orientation drift from constant gyro offsets.

\subsection{Accelerometer Bias $\bm b_a$}
\[
\dot{\bm b}_a = \bm w_{ba}, \qquad 
\bm w_{ba} \sim \mathcal{N}(0, Q_{ba}).
\]
Prevents misattribution of accelerometer offset to $\bm a_w$ or attitude.

\subsection{Why Biases Are Essential}
\begin{enumerate}
\item Long-term stability of attitude and displacement.
\item Clean separation of sensor error from physical motion.
\item Degrees of freedom for measurement consistency.
\end{enumerate}

\section{Kalman Filter Equations}
\subsection{Time Update}
\[
\begin{aligned}
q_{k|k-1} &= q_{k-1}\otimes\exp\!\left(\tfrac{1}{2}(\bm\omega_m - \bm b_g)\Delta t\right), \\
\bm x_{k|k-1} &= F \bm x_{k-1|k-1}, \\
P_{k|k-1} &= F P_{k-1|k-1} F^\top + Q.
\end{aligned}
\]

\subsection{Measurement Update}
\[
\begin{aligned}
\nu &= \bm y_k - h(q,\bm a_w,\bm b_a),\\
S &= HPH^\top + R,\\
K &= PH^\top S^{-1},\\
\bm x^+ &= \bm x + K\nu, \\
P^+ &= (I-KH)P(I-KH)^\top + K R K^\top.
\end{aligned}
\]

Quaternion correction:
\[
q^+ \leftarrow q \otimes \delta q(\delta\bm\theta),\qquad
\delta\bm\theta\leftarrow 0.
\]

\section{Observability and Roles}
\begin{itemize}
\item Roll/pitch: observable from accelerometer.
\item Yaw: observable from magnetometer.
\item Gyro/accel biases: observable via long-term innovations.
\item Latent $a_w$: observable through accelerometer + OU prior.
\item $v,p,S$: indirectly constrained via $a_w$ and pseudo-measure on $S$.
\end{itemize}

\section{Tuning Parameters}
\begin{itemize}
\item $Q_{bg}$: gyro bias random walk variance.
\item $Q_{ba}$: accelerometer bias random walk variance.
\item $\tau_{aw}$: OU correlation time.
\item $\Sigma_{aw}^{stat}$: OU stationary variance.
\item $R_{acc},R_{mag}$: sensor noise covariance.
\item $R_S$: pseudo-measurement covariance for $\bm S$.
\end{itemize}

\end{document}
