\documentclass[11pt]{article}
\usepackage{amsmath, amssymb, amsfonts}
\usepackage{bm}
\usepackage{geometry}
\geometry{margin=1in}
\usepackage{hyperref}

\title{Mathematical Foundations and Modeling Rationale of the Extended \texttt{Kalman3D\_Wave} Filter}
\author{Mikhail Grushinskiy}
\date{2025}

\begin{document}
\maketitle

\begin{abstract}
This paper presents the mathematical foundation, modeling rationale, and 
implementation details of the \texttt{Kalman3D\_Wave} filter, an extended 
multiplicative extended Kalman filter (MEKF) tailored to ocean-wave sensing 
from inertial measurements. The method augments a quaternion-based MEKF with 
linear kinematic states (velocity, position, triple-integrated displacement) 
driven by a latent Ornstein--Uhlenbeck acceleration process. Sensor biases are 
explicitly modeled, including stochastic random-walk gyroscope and accelerometer 
biases, and systematic temperature-dependent accelerometer drift. 
Pseudo-measurements are introduced to control unobservable triple integrals. 
We derive continuous- and discrete-time models, present exact and approximate 
discretization methods (Rodrigues formula, Van Loan method, Padé(6) approximation), 
and provide full Jacobians and transition matrices. The paper concludes with 
discussions of observability and tuning.
\end{abstract}

\section{Introduction}
Wave-induced motion estimation from IMU data presents several challenges:
\begin{enumerate}
\item Raw accelerometer signals contain both gravity and wave-induced accelerations.
\item Velocity and displacement obtained by integration of acceleration drift unboundedly.
\item Gyroscope and accelerometer biases, as well as thermal drifts, corrupt orientation and motion estimates.
\item Wave accelerations are temporally correlated, not white, requiring a realistic dynamic model.
\end{enumerate}

The \texttt{Kalman3D\_Wave} filter addresses these issues by:
\begin{itemize}
\item Using a quaternion-based MEKF for attitude.
\item Introducing a latent world-frame acceleration modeled as Ornstein--Uhlenbeck (OU).
\item Estimating biases explicitly.
\item Applying a pseudo-measurement on the triple integral $\bm S$.
\item Employing exact or approximate discretization methods for numerical stability.
\end{itemize}

\section{State Definition}
The full state vector is
\begin{equation}
\bm{x} = \begin{bmatrix}
\delta\bm\theta & \bm b_g & \bm v & \bm p & \bm S & \bm a_w & \bm b_{a0}
\end{bmatrix}^\top \in \mathbb{R}^n,
\end{equation}
with components:
\begin{itemize}
\item $\delta\bm\theta \in \mathbb{R}^3$: small attitude error.
\item $\bm b_g \in \mathbb{R}^3$: gyroscope bias.
\item $\bm v \in \mathbb{R}^3$: velocity (m/s).
\item $\bm p \in \mathbb{R}^3$: displacement (m).
\item $\bm S \in \mathbb{R}^3$: triple-integrated displacement (m·s).
\item $\bm a_w \in \mathbb{R}^3$: latent wave acceleration (m/s$^2$).
\item $\bm b_{a0} \in \mathbb{R}^3$: baseline accelerometer bias at reference temperature.
\end{itemize}

Temperature-dependent accelerometer bias is modeled separately (see Section~\ref{sec:tempbias}).

Quaternion $q$ represents the rotation from world (NED) to body frame.  
Attitude error is right-multiplicative:
\begin{equation}
q^+ = q \otimes \delta q(\delta\bm\theta), 
\quad 
\delta q(\delta\bm\theta) \approx \left[1,\; \tfrac{1}{2}\delta\bm\theta \right].
\end{equation}

\section{Attitude Dynamics}
Quaternion kinematics are
\begin{equation}
\dot q = \tfrac{1}{2} q \otimes \bm\omega_b, 
\quad
\bm\omega_b = \bm\omega_m - \bm b_g,
\end{equation}
where $\bm\omega_m$ is measured angular velocity.  

Linearized small-angle dynamics:
\begin{equation}
\dot{\delta\bm\theta} = -[\bm\omega_b]_\times \delta\bm\theta - \delta\bm b_g + \bm n_g.
\end{equation}

\subsection{Rodrigues Formula}
The discrete propagation of small-angle errors is obtained using Rodrigues’ rotation formula.  
Let $\bm w$ be the bias-corrected angular velocity, $\theta=\|\bm w\|\Delta t$, and $\hat{\bm w}=\bm w/\|\bm w\|$. Then
\begin{equation}
F_{\theta\theta} = 
I - \sin(\theta)[\hat{\bm w}]_\times + (1-\cos(\theta))[\hat{\bm w}]_\times^2.
\label{eq:rodrigues}
\end{equation}
For small $\theta$,
\begin{equation}
F_{\theta\theta} \approx I - [\bm w]_\times \Delta t + \tfrac{1}{2}[\bm w]_\times^2 \Delta t^2.
\end{equation}

\section{Linear Kinematics and OU Acceleration}
The linear subsystem is
\begin{align}
\dot{\bm v} &= \bm a_w, \\
\dot{\bm p} &= \bm v, \\
\dot{\bm S} &= \bm p.
\end{align}

Latent acceleration:
\begin{equation}
\dot{\bm a}_w = -\tfrac{1}{\tau_{aw}}\bm a_w + \bm w_{aw}(t),
\end{equation}
with $\tau_{aw}$ correlation time.  
Noise intensity is scaled so that stationary variance is $\Sigma_{aw}^{stat}$:
\begin{equation}
\Sigma_c = \tfrac{2}{\tau_{aw}} \Sigma_{aw}^{stat}.
\end{equation}

\subsection{OU Autocorrelation and Spectrum}
The OU process has autocorrelation
\begin{equation}
R(\Delta) = \sigma^2 \exp(-|\Delta|/\tau_{aw}),
\end{equation}
and spectrum
\begin{equation}
S(\omega) = \frac{2\sigma^2\tau_{aw}}{1+(\omega\tau_{aw})^2}.
\end{equation}
Thus $\tau_{aw}$ sets correlation time / cutoff frequency, while $\sigma$ sets RMS magnitude.

\section{Process Matrix}
The transition matrix has block form
\begin{equation}
F = \begin{bmatrix}
F_{\theta\theta} & F_{\theta bg} & 0 & 0 & 0 & 0 & 0 \\
0 & I & 0 & 0 & 0 & 0 & 0 \\
0 & 0 & 0 & 0 & 0 & I & 0 \\
0 & 0 & I & 0 & 0 & 0 & 0 \\
0 & 0 & 0 & I & 0 & 0 & 0 \\
0 & 0 & 0 & 0 & 0 & -\tfrac{1}{\tau_{aw}}I & 0 \\
0 & 0 & 0 & 0 & 0 & 0 & I
\end{bmatrix},
\end{equation}
where $F_{\theta\theta}$ is from Eq.~\eqref{eq:rodrigues} and $F_{\theta bg}=-I\Delta t$.

\section{Discretization}
\subsection{Van Loan Method}
For $\dot x = A x + G w$, $Q_d$ is
\begin{equation}
Q_d = \int_0^{\Delta t} e^{A\tau} G \Sigma_c G^\top e^{A^\top\tau} d\tau.
\end{equation}
Van Loan’s method constructs
\begin{equation}
M = \begin{bmatrix}
- A \Delta t & G \Sigma_c G^\top \Delta t \\
0 & A^\top \Delta t
\end{bmatrix},
\end{equation}
then computes $\exp(M)$, extracts $\Phi$ and $Q_d$ via
\begin{equation}
\Phi = (M_{22})^\top, \qquad Q_d = \Phi M_{12}.
\end{equation}

\subsection{Padé(6) Approximation}
The exponential is approximated as
\begin{equation}
e^A \approx R_6(A) = \left(I - \tfrac{1}{2}A + \tfrac{1}{24}A^2 - \tfrac{1}{720}A^3\right)^{-1}
\left(I + \tfrac{1}{2}A + \tfrac{1}{24}A^2 + \tfrac{1}{720}A^3\right).
\end{equation}
Scaling and squaring is applied:
\begin{equation}
e^A \approx \left( R_6\!\left(\tfrac{A}{2^s}\right) \right)^{2^s}.
\end{equation}
This balances efficiency and stability.

\section{Measurement Models}
\subsection{Accelerometer}
\begin{equation}
\bm f_b = R_{wb}(q)(\bm a_w - \bm g) + \bm b_a(T) + \bm n_a.
\end{equation}
Jacobians:
\begin{equation}
\frac{\partial \bm f_b}{\partial \delta\bm\theta} = -[\bm f_b]_\times, \quad
\frac{\partial \bm f_b}{\partial \bm a_w} = R_{wb}, \quad
\frac{\partial \bm f_b}{\partial \bm b_a} = I.
\end{equation}

\subsection{Magnetometer}
\begin{equation}
\bm m_b = R_{wb}(q)\bm m_w + \bm n_m,
\end{equation}
Jacobian:
\begin{equation}
\frac{\partial \bm m_b}{\partial \delta\bm\theta} = -[\bm m_b]_\times.
\end{equation}

\section{Pseudo-Measurement of $\bm S$}
We constrain
\begin{equation}
z_S = 0 = H_S \bm x + n_S,
\end{equation}
with $H_S$ selecting $\bm S$.  
This suppresses drift of the triple integral.

\section{Bias States}
\subsection{Gyroscope Bias}
Modeled as random walk:
\begin{equation}
\dot{\bm b}_g = w_{bg}, \quad w_{bg}\sim\mathcal{N}(0,Q_{bg}).
\end{equation}

\subsection{Accelerometer Bias}
\begin{equation}
\dot{\bm b}_{a0} = w_{ba}, \quad w_{ba}\sim\mathcal{N}(0,Q_{ba}).
\end{equation}

\subsection{Temperature-Dependent Bias Drift}
\label{sec:tempbias}
Accelerometer bias depends on temperature:
\begin{equation}
\bm b_a(T) = \bm b_{a0} + \bm k_a (T - T_{ref}),
\end{equation}
where $\bm k_a$ is a calibrated per-axis coefficient.

\section{Kalman Filter Equations}
\subsection{Time Update}
\begin{align}
q_{k|k-1} &= q_{k-1}\otimes\exp\!\left(\tfrac{1}{2}(\bm\omega_m - \bm b_g)\Delta t\right), \\
\bm x_{k|k-1} &= F \bm x_{k-1|k-1}, \\
P_{k|k-1} &= F P_{k-1|k-1} F^\top + Q.
\end{align}

\subsection{Measurement Update}
\begin{align}
\nu &= y - h(x), \\
S &= HPH^\top + R, \\
K &= PH^\top S^{-1}, \\
\bm x^+ &= \bm x + K\nu, \\
P^+ &= (I-KH)P(I-KH)^\top + KRK^\top.
\end{align}
Quaternion correction:
\begin{equation}
q^+ \leftarrow q \otimes \delta q(\delta\bm\theta), \quad \delta\bm\theta\leftarrow 0.
\end{equation}

\section{Observability}
\begin{itemize}
\item Roll/pitch: observable from accelerometer.
\item Yaw: observable from magnetometer.
\item Biases: observable from long-term consistency.
\item Latent $\bm a_w$: observable via accelerometer with OU prior.
\item $\bm S$: constrained via pseudo-measurement.
\end{itemize}

\section{Tuning Guidelines}
\begin{itemize}
\item $Q_{bg},Q_{ba}$ from sensor specs.
\item $\tau_{aw}$: 0.5–2 s for typical sea states.
\item $\Sigma_{aw}^{stat}$: sets expected acceleration variance.
\item $R_{acc},R_{mag}$ from noise densities.
\item $R_S$: controls strength of triple-integral suppression.
\item $\bm k_a$: from thermal calibration of accelerometer.
\end{itemize}

\end{document}

