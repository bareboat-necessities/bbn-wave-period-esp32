\documentclass[11pt]{article}
\usepackage{amsmath, amssymb, amsfonts, bm}
\usepackage{geometry}
\usepackage{hyperref}
\geometry{margin=1in}

\title{A Multiplicative EKF with Latent OU Acceleration for Drift-Robust Wave Kinematics: \\
Theory, Discretization (Rodrigues, Van Loan, Pad\'e(6)), and Temperature-Dependent Bias Compensation}
\author{Mikhail Grushinskiy}
\date{2025}

\begin{document}
\maketitle

\begin{abstract}
We present a detailed mathematical development of a quaternion multiplicative EKF (MEKF) fused with an extended 
linear kinematic chain for ocean-wave motion estimation. 
The method augments attitude (with optional gyroscope bias) by the linear states velocity $\bm v$, displacement $\bm p$, 
and the integral of displacement $\bm S$, driven by a \emph{latent} world-frame acceleration $\bm a_w$ modeled as an Ornstein--Uhlenbeck (OU) process. 
The OU prior confers bounded variance and realistic temporal correlation, preventing the drift explosion typical of double integration of noisy accelerometer data. 
Biases are explicitly modeled: gyroscope bias as a random walk, accelerometer bias both as random walk and as systematic temperature-dependent drift. 
A pseudo-measurement is introduced on $\bm S$ to control triple integral divergence. 
We derive continuous- and discrete-time process models, show Rodrigues and Van Loan discretizations, 
explain Pad\'e(6) approximation with scaling and squaring, and derive explicit Jacobians for accelerometer and magnetometer updates. 
Finally, we discuss tuning strategies and the prospect of adaptive parameter estimation.
\end{abstract}

\section{Introduction}
Estimating wave-induced motion from an IMU is fundamentally challenging. The IMU provides angular velocity and specific force, 
but extracting displacement requires triple integration if treated naively. Biases and noise cause unbounded drift. 
Moreover, wave accelerations are not white but correlated in time, reflecting the physics of sea surface gravity waves. 
Thus, a specialized state-space formulation is required. 

The \texttt{Kalman3D\_Wave} filter is designed with these challenges in mind. 
It is a multiplicative EKF in which the quaternion is the central orientation representation, 
augmented by translational kinematics driven by a latent Ornstein--Uhlenbeck acceleration process. 
This design provides bounded variance, stability, and physical realism.

The contributions of this paper are:
\begin{itemize}
\item A rigorous derivation of the state-space model with biases and OU latent acceleration.
\item Exact and approximate discretization methods: Rodrigues, Van Loan, Pad\'e(6).
\item Explicit Jacobians for accelerometer and magnetometer updates.
\item Introduction of a pseudo-measurement on the triple integral to suppress drift.
\item Discussion of tuning and future adaptive strategies.
\end{itemize}

\section{State Definition}

We define the full augmented error-state vector as
\begin{equation}
\bm{x} =
\begin{bmatrix}
\delta\bm\theta & \bm b_g & \bm v & \bm p & \bm S & \bm a_w & \bm b_{a0}
\end{bmatrix}^\top \in \mathbb{R}^n,
\label{eq:state_vector}
\end{equation}
with the following components:
\begin{itemize}
\item $\delta\bm\theta \in \mathbb{R}^3$: the \emph{attitude error}, parameterized as a small rotation vector in the multiplicative EKF framework.
\item $\bm b_g \in \mathbb{R}^3$: the gyroscope bias, modeled as a random walk driven by white noise.
\item $\bm v \in \mathbb{R}^3$: the velocity of the sensor in the world (NED) frame.
\item $\bm p \in \mathbb{R}^3$: the displacement (position) in the world frame.
\item $\bm S \in \mathbb{R}^3$: the \emph{integral of displacement}, i.e.
  \[
  \bm S(t) = \int_0^t \bm p(\tau)\, d\tau,
  \]
  which, although not physically measured, serves as a control variable for drift suppression.
\item $\bm a_w \in \mathbb{R}^3$: the latent world-frame acceleration, modeled as an Ornstein--Uhlenbeck (OU) process to enforce bounded variance and temporal correlation.
\item $\bm b_{a0} \in \mathbb{R}^3$: the baseline accelerometer bias at a fixed reference temperature $T_\text{ref}$.
\end{itemize}

The \emph{temperature-dependent accelerometer bias model} is given by
\begin{equation}
\bm b_a(T) = \bm b_{a0} + \bm k_a \,\big(T - T_\text{ref}\big),
\label{eq:accel_bias_temp}
\end{equation}
where $\bm k_a \in \mathbb{R}^3$ is a vector of per-axis temperature coefficients. 
This accounts for the systematic drift of MEMS accelerometers with changing temperature, 
typically on the order of $0.002$--$0.005 \,\text{m/s}^2$ per ${}^\circ$C in modern sensors.

\subsection{Quaternion Representation}
The orientation of the body frame with respect to the world (NED) frame is represented by a quaternion $q \in \mathbb{H}$. 
We adopt a right-multiplicative error convention:
\begin{equation}
q^+ = q \otimes \delta q(\delta\bm\theta),
\end{equation}
where $\otimes$ denotes quaternion multiplication and
\begin{equation}
\delta q(\delta\bm\theta) \approx
\begin{bmatrix}
1 \\ \tfrac{1}{2}\delta\bm\theta
\end{bmatrix}.
\label{eq:small_angle_quaternion}
\end{equation}

\subsection{State Dimension}
The total state dimension is
\[
n =
\begin{cases}
18 & \text{if gyro and accel biases are included}, \\
15 & \text{if only gyro bias is included}, \\
12 & \text{if no biases are included}.
\end{cases}
\]
This flexible design allows the filter to adapt to the sensor suite available and to application requirements.

\section{Attitude Dynamics}

The orientation of the sensor platform is represented by a quaternion $q(t)$ mapping from the world 
(North--East--Down, NED) frame to the body frame. 
We employ the \emph{multiplicative extended Kalman filter} (MEKF) convention: 
the mean orientation is stored as a quaternion $q$, while the small attitude error 
is represented as a 3-vector $\delta\bm\theta$ living in the tangent space $\mathfrak{so}(3)$.

\subsection{Quaternion Kinematics}
The quaternion time evolution is governed by the angular velocity measured by the gyroscope:
\begin{equation}
\dot q(t) = \tfrac{1}{2} \, \Omega(\bm\omega_b(t)) \, q(t),
\label{eq:quat_kinematics}
\end{equation}
where $\bm\omega_b$ is the angular velocity in the body frame, 
and $\Omega(\bm\omega)$ is the quaternion multiplication matrix:
\begin{equation}
\Omega(\bm\omega) =
\begin{bmatrix}
0 & -\bm\omega^\top \\
\bm\omega & -[\bm\omega]_\times
\end{bmatrix}.
\end{equation}

Here $[\bm\omega]_\times$ denotes the skew-symmetric matrix such that $[\bm\omega]_\times \bm v = \bm\omega \times \bm v$.

\subsection{Error Representation}
Instead of estimating $q$ directly, the MEKF keeps track of the \emph{error quaternion}:
\begin{equation}
q = \hat q \otimes \delta q(\delta\bm\theta),
\end{equation}
where $\hat q$ is the nominal quaternion and $\delta q$ is a small correction. 
This choice ensures that the state covariance remains minimal in dimension (3 instead of 4), 
while preserving the unit norm of $q$.

\subsection{Rodrigues' Formula for Discrete Propagation}
To propagate the quaternion over a sampling interval $\Delta t$, 
we use the matrix exponential of the skew operator:
\begin{equation}
R(t+\Delta t) = R(t) \exp\!\left([\bm\omega]_\times \Delta t\right).
\label{eq:rot_exp}
\end{equation}

Rodrigues' rotation formula provides a closed form for this exponential:
\begin{equation}
\exp([\bm\omega]_\times \Delta t) = I 
+ \frac{\sin \theta}{\theta} [\bm u]_\times
+ \frac{1 - \cos \theta}{\theta^2} [\bm u]_\times^2,
\label{eq:rodrigues}
\end{equation}
where $\theta = \|\bm\omega\| \Delta t$ is the rotation angle, 
and $\bm u = \bm\omega / \|\bm\omega\|$ is the unit rotation axis.

\paragraph{Interpretation.}
Equation \eqref{eq:rodrigues} shows how the rotation matrix can be expressed 
directly in terms of the angular increment. 
For small angles, a Taylor expansion recovers the familiar linearized form:
\[
\exp([\bm\omega]_\times \Delta t) \approx I + [\bm\omega]_\times \Delta t + \tfrac{1}{2} [\bm\omega]_\times^2 \Delta t^2.
\]
This makes Rodrigues' formula both numerically stable and exact for finite rotations.

\subsection{Error-State Dynamics}
The small attitude error $\delta\bm\theta$ evolves according to
\begin{equation}
\dot{\delta\bm\theta} = -[\bm\omega_b - \bm b_g]_\times \, \delta\bm\theta - \delta\bm b_g + \bm n_\theta,
\label{eq:att_err_dyn}
\end{equation}
where $\bm b_g$ is the gyro bias and $\bm n_\theta$ is gyro measurement noise mapped into the attitude error dynamics.

Equation \eqref{eq:att_err_dyn} highlights two important points:
\begin{enumerate}
\item The gyro bias directly drives the attitude error, which motivates its inclusion in the state vector.
\item The dynamics are linear in the small error, allowing standard EKF propagation.
\end{enumerate}



