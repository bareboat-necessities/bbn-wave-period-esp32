\documentclass[10pt,twocolumn]{extarticle}

\usepackage{pgf}
\usepackage{float}
\usepackage{mathrsfs}
\usepackage{graphicx}

% --- Packages ---
\usepackage{amsmath, amssymb, amsfonts, bm}
\usepackage{geometry, setspace}
\usepackage{hyperref}
\usepackage{bookmark}
\usepackage{titlesec}
\usepackage{nccmath, relsize}
\usepackage{array, booktabs, makecell, adjustbox}
\usepackage{mathtools}
\usepackage{enumitem}
\setlist[itemize]{before=\normalfont\normalsize, itemsep=2pt, topsep=4pt, leftmargin=*}
\setlist[enumerate]{before=\normalfont\normalsize, itemsep=2pt, topsep=4pt, leftmargin=*}
\setlist[description]{before=\normalfont\normalsize, font=\normalfont\bfseries,
  labelsep=0.6em, leftmargin=1.8em, itemsep=2pt, topsep=4pt}
\usepackage{amsthm}
\usepackage[nameinlink, capitalise, noabbrev]{cleveref}
\usepackage{balance} % optional: balance final page columns
\usepackage[breaklinks=true]{hyperref}

% Two-column tuning
\setlength{\columnsep}{18pt} % space between columns
\newcommand{\fitcol}[1]{\adjustbox{max width=\columnwidth}{$\displaystyle #1$}}

\usepackage{siunitx}
\sisetup{
  detect-weight = true,
  detect-family = true,
  per-mode = symbol,
}

% Make math/siunitx safe in PDF bookmarks
\pdfstringdefDisableCommands{%
  \def\SI#1#2{#1 #2}%
  \def\bm#1{#1}%
  \def\Skew#1{[#1]_\text{x}}%
  \def\Id{I}%
  \def\pct#1{#1\%}%
  \def\circ{deg}%
  \def\degree{deg}% siunitx \degree if it leaks into bookmarks
  \def\propto{∝}%
  \def\sigma{sigma}%
  \def\tau{tau}%
  \def\Phi{Phi}%
  \def\omega{omega}%
  \def\_{}%
  \def\cdot{}%
  \def\;{ }%
}

% Keep this — needed by PGF outputs that reference \mathdefault
\providecommand{\mathdefault}[1]{#1}

% --- Engine-safe encoding setup + Unicode punctuation handling ---
\usepackage{iftex}

\ifPDFTeX
  % pdfLaTeX path: need inputenc/fontenc and map Unicode punctuation
  \usepackage[utf8]{inputenc}
  \usepackage[T1]{fontenc}
  \usepackage{lmodern}
  \usepackage{textcomp} % provides \textendash, \textemdash, quotes, \textminus

  % Map common Unicode punctuation to TeX macros (only in pdfLaTeX)
  \DeclareUnicodeCharacter{2013}{\textendash}        % –
  \DeclareUnicodeCharacter{2014}{\textemdash}        % —
  \DeclareUnicodeCharacter{2018}{\textquoteleft}     % ‘
  \DeclareUnicodeCharacter{2019}{\textquoteright}    % ’
  \DeclareUnicodeCharacter{201C}{\textquotedblleft}  % “
  \DeclareUnicodeCharacter{201D}{\textquotedblright} % ”
  \DeclareUnicodeCharacter{2212}{\textminus}         % − (math minus)
\else
  % LuaLaTeX/XeLaTeX path: native Unicode, no inputenc/fontenc
  \usepackage{fontspec}
  \defaultfontfeatures{Ligatures=TeX}
  \setmainfont{Latin Modern Roman}
  \setsansfont{Latin Modern Sans}
  \setmonofont{Latin Modern Mono}
\fi

% Theorems: small header/body
\newtheoremstyle{bodysmall}
  {3pt}{3pt}{\normalfont\small}{0pt}{\bfseries\small}{.}{0.5em}{}
\theoremstyle{bodysmall}
\newtheorem{lemma}{Lemma}
\newtheorem{theorem}{Theorem}
\newtheorem*{theorem*}{Theorem}

% Shortcuts
\newcommand{\Skew}[1]{\left[#1\right]_\times}
\newcommand{\Id}{\mathbf{I}}
\newcommand{\meanstd}[2]{#1\,{\tiny$\pm$}\,#2}
\newcommand{\pct}[1]{#1\%}
\newcommand{\ci}[2]{#1\,{\tiny(\,#2\,CI\,)}}

\DeclareMathOperator{\proj}{\mathcal{P}}
\DeclareMathOperator{\clip}{clip}
\DeclareMathOperator{\wrap}{wrap}
\DeclareMathOperator{\sgn}{sgn}

% Slightly smaller display math
\usepackage{etoolbox}
\AtBeginEnvironment{equation}{\small}
\AtBeginEnvironment{align}{\small}
\AtBeginEnvironment{gather}{\small}
\AtBeginEnvironment{multline}{\small}
\AtBeginEnvironment{flalign}{\small}
\AtBeginEnvironment{alignat}{\small}

% Heading spacing + sizes
\titlespacing*{\section}{0pt}{1.0ex plus .2ex}{0.6ex}
\titlespacing*{\subsection}{0pt}{0.8ex plus .2ex}{0.4ex}
\titlespacing*{\subsubsection}{0pt}{0.6ex plus .2ex}{0.3ex}

\titleformat{\section}{\normalfont\small\bfseries}{\thesection}{0.8em}{}
\titleformat{\subsection}{\normalfont\small\itshape}{\thesubsection}{0.8em}{}
\titleformat{\subsubsection}{\normalfont\footnotesize\itshape}{\thesubsubsection}{0.8em}{}
\titleformat{\paragraph}{\normalfont\small\bfseries}{\theparagraph}{0.6em}{}[]
\titlespacing*{\paragraph}{0pt}{0.4ex plus .1ex}{0.8em}

\usepackage[final]{microtype}   % better spacing/hyphenation
\emergencystretch=2em           % give TeX extra room before overfull boxes
\allowdisplaybreaks             % allow multi-line displays to break across columns

% Description labels small & bold
\setlist[description]{font=\footnotesize\bfseries, labelsep=0.6em, leftmargin=1.8em}

\geometry{letterpaper, margin=0.65in}
\setstretch{0.95}

\title{A Multiplicative EKF with Latent OU Acceleration for Drift-Robust Wave Kinematics: \\
Theory, Discretization, and Bias Compensation}
\author{Mikhail Grushinskiy}
\date{2025}

\begin{document}

% Full-width title + abstract at top of first page (portable; no cuted needed)
\twocolumn[
\maketitle
\begin{abstract}
We present a detailed mathematical development of a quaternion multiplicative EKF (MEKF) fused with an extended 
linear kinematic chain for ocean-wave motion estimation. 
The method augments attitude (with optional gyroscope and accelerometer bias) by the linear states velocity $\bm v$, displacement $\bm p$, 
and the integral of displacement $\bm S$, driven by a \emph{latent} world-frame acceleration $\bm a_w$ modeled as an Ornstein--Uhlenbeck (OU) process. The OU prior confers stationary variance and realistic temporal correlation for the latent acceleration, which slows the drift growth in integrated states compared to white-noise forcing. It does not strictly bound the variance of integrated displacement, but ensures covariance growth of displacement is polynomially slower and numerically manageable.
Biases are explicitly modeled: gyroscope bias as a random walk, accelerometer bias both as random walk and as systematic temperature-dependent drift. 
A pseudo-measurement is introduced on $\bm S$ to control double integral divergence. 
We derive continuous- and discrete-time process models, show Rodrigues and Analytic discretizations, and derive explicit Jacobians for accelerometer and magnetometer updates. 
Finally, we discuss tuning strategies and the prospect of adaptive parameter estimation.
\end{abstract}
\vspace{1ex}
]

% If you want a full-width ToC, do a short single-column page:
\onecolumn
\tableofcontents
\clearpage
\twocolumn

% === CORE SECTIONS ===
\input{w3d-intro.tex-part}
\input{w3d-state.tex-part}
\input{w3d-meas.tex-part}
\input{w3d-proc-cont.tex-part}
\input{w3d-lti-discrete.tex-part}
%\input{w3d-proc.tex-part}
\input{w3d-analytic-coeff.tex-part}
%\input{w3d-tupdate.tex-part}
\input{w3d-kalm-up.tex-part}
\input{w3d-init.tex-part}

% === ANALYTIC DISCRETIZATION AND COVARIANCE STRUCTURE ===
%\input{w3d-analytic.tex-part}
%\input{w3d-analytic-lap.tex-part}

%\input{w3d-cross-cov.tex-part}
%\input{w3d-riccati.tex-part}

% === JACOBIANS ===
%\input{w3d-jacobian.tex-part}
%\input{w3d-jac-attitude.tex-part}
%\input{w3d-jac-meas.tex-part}

% === ORIENTATION HANDLING ===
%\input{w3d-quat.tex-part}

% === OU APPENDICES ===
%\input{w3d-ou-q.tex-part}

%\input{w3d-lyap.tex-part}

%\section{Symbolical Derivations}
%\input{w3d-phi-symb.tex-part}
%\input{w3d-qd-symb.tex-part}
%\input{w3d-qd-symb-lap.tex-part}
%\input{w3d-phi-A-symb.tex-part}
%\input{w3d-qd-A-symb.tex-part}
%\input{w3d-proc-symb-assembly.tex-part}

\input{fus-methods.tex-part}

%\input{sim-results.tex-part}
\input{sim-charts.tex-part}

% Balance final page columns (optional)
\balance

% === END ===
\end{document}
