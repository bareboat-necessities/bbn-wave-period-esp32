\documentclass[conference]{IEEEtran}

% Core packages
\usepackage{iftex}
\ifPDFTeX
  \usepackage[utf8]{inputenc}
  \usepackage[T1]{fontenc}
\fi

\usepackage{amsthm}
\usepackage{amsmath, bm}
\usepackage{newtxtext,newtxmath} % Times-like text + math (IEEE-style)
\usepackage{textcomp}

\interdisplaylinepenalty=2500 % allow page breaks in multiline equations
\usepackage{pgf}
\usepackage{pgfplots}
\usepackage{xcolor}
\usepackage{float}
\usepackage{mathrsfs}
\usepackage{graphicx}
\usepackage{placeins} % for \FloatBarrier
\usepackage{array, booktabs, makecell, adjustbox}
\usepackage{mathtools}
\usepackage{nccmath, relsize}
\usepackage{enumitem}

\usepackage{balance} % balance final page columns
\usepackage{flafter} % figures/tables won’t appear before they’re definedap
\usepackage{siunitx}
\usepackage[final]{microtype}   % better spacing/hyphenation
\usepackage{etoolbox}

%\let\Bbbk\relax
%\usepackage{amssymb}

\usepackage{cite}
\usepackage[breaklinks=true,hidelinks]{hyperref}
\usepackage{bookmark}
\usepackage[nameinlink,capitalise,noabbrev]{cleveref}

\sisetup{
  detect-weight = true,
  detect-family = true,
  per-mode = symbol,
}

% Description labels small & bold
\setlist[description]{font=\footnotesize\bfseries, labelsep=0.6em, leftmargin=1.8em}

% Make math/siunitx safe in PDF bookmarks
\pdfstringdefDisableCommands{%
  \def\SI#1#2{#1 #2}%
  \def\bm#1{#1}%
  \def\Skew#1{[#1]_\text{x}}%
  \def\Id{I}%
  \def\pct#1{#1\%}%
  \def\circ{deg}%
  \def\degree{deg}% siunitx \degree if it leaks into bookmarks
  \def\propto{∝}%
  \def\sigma{sigma}%
  \def\tau{tau}%
  \def\Phi{Phi}%
  \def\omega{omega}%
  \def\_{}%
  \def\cdot{}%
  \def\;{ }%
}

\AtBeginEnvironment{equation}{\small}
\AtBeginEnvironment{align}{\small}

% Needed by PGF outputs that reference \mathdefault
\providecommand{\mathdefault}[1]{#1}

\setcounter{topnumber}{5}
\setcounter{dbltopnumber}{5}
\renewcommand{\topfraction}{0.95}
\renewcommand{\dbltopfraction}{0.95}
\renewcommand{\textfraction}{0.05}
\renewcommand{\floatpagefraction}{0.9}
\renewcommand{\dblfloatpagefraction}{0.9}

% For including PGF
\newcommand{\IncludePGF}[1]{\resizebox{\linewidth}{!}{\input{#1}}}

% Unicode punctuation mappings (pdfLaTeX only)
\ifPDFTeX
  \DeclareUnicodeCharacter{2013}{\textendash}        % –
  \DeclareUnicodeCharacter{2014}{\textemdash}        % —
  \DeclareUnicodeCharacter{2018}{\textquoteleft}     % ‘
  \DeclareUnicodeCharacter{2019}{\textquoteright}    % ’
  \DeclareUnicodeCharacter{201C}{\textquotedblleft}  % “
  \DeclareUnicodeCharacter{201D}{\textquotedblright} % ”
  \DeclareUnicodeCharacter{2212}{\textminus}         % − (math minus)
\fi

% Theorems: small header/body
\newtheoremstyle{bodysmall}
  {3pt}{3pt}{\normalfont\small}{0pt}{\bfseries\small}{.}{0.5em}{}
\theoremstyle{bodysmall}
\newtheorem{lemma}{Lemma}
\newtheorem{theorem}{Theorem}
\newtheorem*{theorem*}{Theorem}

% Shortcuts
\newcommand{\Skew}[1]{\left[#1\right]_\times}
\newcommand{\Id}{\mathbf{I}}
\newcommand{\pct}[1]{#1\%}
\newcommand{\meanstd}[2]{#1 \pm #2}
% or
% \newcommand{\meanstd}[2]{#1\,{\scriptsize$\pm$}\,#2}
\newcommand{\ci}[2]{#1\,(\,#2\,\text{CI}\,)}

% Itô integral
\newcommand{\Ito}{\mathrm{It\hat{o}}}
\newcommand{\ItoInt}[2]{\int_0^{#1} #2\, dW_s^{\Ito}}

\DeclareMathOperator{\proj}{\mathcal{P}}
\DeclareMathOperator{\clip}{clip}
\DeclareMathOperator{\wrap}{wrap}
\DeclareMathOperator{\sgn}{sgn}
\DeclareMathOperator{\blkdiag}{blkdiag}
\DeclareMathOperator{\diag}{diag}
\DeclareMathOperator{\sym}{sym}

% Two-column helper (no layout change)
\newcommand{\fitcol}[1]{%
  \adjustbox{max width=\columnwidth}{\ensuremath{\displaystyle #1}}%
}

\makeatletter
\@ifpackageloaded{cite}{%
  \renewcommand{\citepunct}{,\penalty\citepunctpenalty\,}%
  \renewcommand{\citedash}{--}%
}{}
\makeatother

% Allow multi-line displays to break across columns
\emergencystretch=2em
\allowdisplaybreaks

% Title & authors (IEEEtran conference style)
\title{A Multiplicative EKF with Latent OU Acceleration for Drift-Robust Wave Kinematics:\\
Theory, Discretization, and Bias Compensation (draft)}

\author{%
\IEEEauthorblockN{Mikhail S. Grushinskiy}
\IEEEauthorblockA{Independent Researcher, USA, Fair Lawn\\
Bareboat Necessities GitHub Project\\
Email: mgrouch@users.github.com}%
}

\pgfplotsset{compat=1.18}

\begin{document}
\maketitle

\begin{abstract}
We present a detailed mathematical development of a quaternion multiplicative extended Kalman filter fused with an extended
linear kinematic chain for ocean-wave motion estimation. The method augments attitude, with gyroscope and
accelerometer bias, by linear velocity, displacement, and the integral of displacement, driven by a latent world-frame
acceleration modeled as an Ornstein--Uhlenbeck process. The Ornstein--Uhlenbeck prior provides stationary variance and
realistic temporal correlation for the latent acceleration, which slows the growth of drift in integrated states compared
with white-noise forcing. Biases are explicitly modeled: gyroscope bias as a
random walk and accelerometer bias as both random walk and systematic temperature-dependent drift. A pseudo-measurement is
introduced on the integral of displacement to control double-integral divergence. We derive continuous-time and discrete-time
process models, present Rodrigues and analytic discretizations, and give explicit Jacobians for accelerometer and
magnetometer updates. We design and present an additional Kalman filter for wave direction estimation. 
Finally, we discuss tuning strategies and adaptive parameter estimation using frequency trackers.
\end{abstract}

\begin{IEEEkeywords}
extended Kalman filter, inertial navigation, IMU, marine motion reference unit, MRU, Ornstein--Uhlenbeck process, heave reference system, HRS, heave sensor,
wave height estimation, wave measurement buoy
\end{IEEEkeywords}

% === CORE SECTIONS ===
\input{w3d-intro.tex-part}
\input{w3d-state.tex-part}
\input{w3d-meas.tex-part}
\input{w3d-proc-cont.tex-part}
\input{w3d-lti-discrete.tex-part}
%\input{w3d-proc.tex-part}
\input{w3d-analytic-coeff.tex-part}
%\input{w3d-tupdate.tex-part}
\input{w3d-kalm-up.tex-part}
\input{w3d-init.tex-part}

% === ANALYTIC DISCRETIZATION AND COVARIANCE STRUCTURE ===
%\input{w3d-analytic.tex-part}
%\input{w3d-analytic-lap.tex-part}

%\input{w3d-cross-cov.tex-part}
%\input{w3d-riccati.tex-part}

% === JACOBIANS ===
%\input{w3d-jacobian.tex-part}
%\input{w3d-jac-attitude.tex-part}
%\input{w3d-jac-meas.tex-part}

% === ORIENTATION HANDLING ===
%\input{w3d-quat.tex-part}

% === OU APPENDICES ===
%\input{w3d-ou-q.tex-part}

\input{fus-methods.tex-part}

%\FloatBarrier              % flush any earlier floats

%\input{sim-results.tex-part}
\input{sim-charts.tex-part}

\FloatBarrier

\section*{Acknowledgments}
This work was prepared with the supervised assistance of artificial intelligence tools, including ChatGPT and DeepSeek, which were employed to aid in code generation, mathematical drafting, and editorial preparation. All conceptual development, validation of results, and final revisions were conducted independently by the author.

%\nocite{*}
\bibliographystyle{IEEEtran}
\bibliography{w3d}

% Balance final page columns (optional, IEEE-style)
\balance

\newpage
\section*{Apendix}
Stability proof for adaptation and symbolical derivations for discretization coefficients matrices.
\input{w3d-lyap.tex-part}

%\section*{Symbolical Derivations}
%\input{w3d-phi-symb.tex-part}
%\input{w3d-qd-symb.tex-part}
%\input{w3d-qd-symb-lap.tex-part}
%\input{w3d-phi-A-symb.tex-part}
%\input{w3d-qd-A-symb.tex-part}
%\input{w3d-proc-symb-assembly.tex-part}

\end{document}
