\documentclass[11pt]{article}
\usepackage{amsmath, amssymb, bm, geometry}
\geometry{margin=1in}

\title{SeaMetrics: Online Estimator of Ocean Wave Spectral Metrics}
\author{Mikhail Grushinskiy}
\date{2025}

\begin{document}
\maketitle

\section{Overview}

\texttt{SeaMetrics} is a C++17 implementation of an online estimator of ocean wave
spectral metrics, operating on vertical acceleration $a_z(t)$ from an IMU and an externally
tracked instantaneous angular frequency $\omega(t)$. 

By demodulating the acceleration signal, normalizing to displacement,
and tracking exponential running averages, \texttt{SeaMetrics} produces
a wide range of oceanographic metrics in real time.

\section{Input Signals}

\begin{itemize}
  \item Vertical acceleration $a_z(t)$ [m/s$^2$]
  \item Instantaneous frequency estimate $\omega_{\text{inst}}(t)$ [rad/s]
\end{itemize}

\section{Spectral Moments}

The central quantities are the (normalized) spectral moments:
\begin{align}
M_0 &= \langle P_{\eta} \rangle, \\
M_1 &= \langle P_{\eta} \, \omega \rangle, \\
M_2 &= \langle P_{\eta} \, \omega^2 \rangle, \\
M_3 &= \langle P_{\eta} \, \omega^3 \rangle \quad \text{(optional)}, \\
M_4 &= \langle P_{\eta} \, \omega^4 \rangle \quad \text{(optional)}, \\
M_{-1} &= \langle P_{\eta} \, \omega^{-1} \rangle \quad \text{(optional)},
\end{align}
where $P_{\eta}$ is the estimated displacement envelope power.

\section{Central Moments and Shape (optional)}

Using $M_k$, central moments can be formed:
\begin{align}
\mu_2 &= \tfrac{M_2}{M_0} - \left(\tfrac{M_1}{M_0}\right)^2, \\
\mu_3 &= \tfrac{M_3}{M_0} - 3\mu \tfrac{M_2}{M_0} + 2\mu^3, \\
\mu_4 &= \tfrac{M_4}{M_0} - 4\mu \tfrac{M_3}{M_0} + 6\mu^2 \tfrac{M_2}{M_0} - 3\mu^4,
\end{align}
with $\mu = M_1/M_0$.

From these, spectral shape diagnostics are available:
\begin{align}
\text{Skewness} &= \frac{\mu_3}{\mu_2^{3/2}}, \\
\text{Kurtosis} &= \frac{\mu_4}{\mu_2^2}, \\
\text{Excess Kurtosis} &= \frac{\mu_4}{\mu_2^2} - 3.
\end{align}

\section{Frequency Metrics}

\begin{align}
\bar{\omega} &= \frac{M_1}{M_0}, 
\qquad \bar{f} = \frac{\bar{\omega}}{2\pi}, \\
\text{RBW} &= \frac{\sqrt{\mu_2}}{\bar{\omega}}, \\
\text{RBW}_{\phi} &= \frac{\sqrt{\mathrm{Var}(\Delta \phi / \Delta t)}}{\bar{\omega}}.
\end{align}

\section{Regularity Metrics}

\begin{align}
R_{\text{spec}} &= \exp(-\beta \cdot \text{RBW}), \\
R_{\text{phase}} &= \left\|\left\langle \frac{z}{|z|} \right\rangle\right\|, \\
\nu &= \sqrt{\frac{M_0 M_2}{M_1^2} - 1}.
\end{align}

\section{Period Summaries}

\begin{align}
T_z &= \sqrt{2\pi^2 \frac{M_0}{M_2}}, \\
T_1 &= \frac{2\pi M_0}{M_1}, \\
T_{m02} &= 2\pi \sqrt{\frac{M_0}{M_2}}, \\
T_e &= \frac{2\pi M_{-1}}{M_0}, \\
T_{m0,-1} &= \frac{M_0}{M_{-1}}, \\
T_{m1,-1} &= \frac{M_1}{M_{-1}}.
\end{align}

The upcrossing rate is
\begin{equation}
\nu_{\uparrow} = \frac{1}{2\pi} \sqrt{\frac{M_2}{M_0}},
\end{equation}
and expected wave counts and confidence intervals are available via Poisson/Garwood formulas.

\section{Heights and Steepness}

\begin{align}
\sigma &= \sqrt{M_0}, \\
H_s &= 2\sqrt{2} \, \sigma, \\
\text{RMS disp.} &= \sigma, \\
\text{Steepness} &= \frac{H_s}{L_0}, \qquad L_0 = \frac{g T_z^2}{2\pi}.
\end{align}

\section{Probabilities and Extremes}

\begin{itemize}
  \item Rayleigh exceedance:
  \[
  P(H_c > h) = \exp\!\left(-\frac{h^2}{2\sigma^2}\right).
  \]
  \item Tayfun nonlinear exceedance (approx.):
  \[
  P(H_c > h) \approx \exp\!\left(-\frac{h^2}{2\sigma^2}\right) \cdot
  \exp\!\left(\Lambda \frac{h^3}{\sigma^3}\right).
  \]
  \item Crest heights: $H_{1/10}, H_{1/100}$.
\end{itemize}

\section{Bandwidth Metrics}

\begin{align}
\varepsilon_{\text{CLH}} &= \sqrt{1 - \frac{M_1^2}{M_0 M_2}}, \\
\varepsilon_{\text{Goda}} &= \sqrt{\frac{M_0 M_2}{M_1^2} - 1}, \\
\varepsilon_{\text{Kuik}} &= \frac{\sqrt{M_0 M_2 - M_1^2}}{M_1}, \\
\varepsilon_{\text{LH}} &= \sqrt{\frac{M_0 M_2}{M_1^2} - 1}.
\end{align}

\section{Energy and Power}

\begin{align}
T_{e,\text{flux}} &= \frac{M_{-1}}{M_0}, \\
P &= \frac{\rho g^2}{64 \pi} H_s^2 T_e.
\end{align}

\section{Bias-Corrected Metrics}

To reduce bias from frequency jitter, Jensen’s inequality correction is applied:
\[
E\!\left[\frac{1}{\omega^n}\right] \approx 
\frac{1}{\bar{\omega}^n}\Big(1 + c_n \frac{\sigma_\omega^2}{\bar{\omega}^2}\Big),
\]
with coefficients $c_1=1, c_2=3, c_3=6, c_4=10$.  
Bias-corrected versions of all major metrics (moments, periods, bandwidths, extremes, power) are provided.

\end{document}
