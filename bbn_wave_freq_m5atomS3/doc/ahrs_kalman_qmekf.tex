\documentclass[12pt]{article}
\usepackage{amsmath,amssymb,amsthm,fullpage}
\usepackage{authblk}
\usepackage{hyperref}
\usepackage{cite}

\title{A Quaternion‐Multiplicative Extended Kalman Filter for Attitude Estimation:\\
Mathematical Formulation and Analysis}
\author{}
\date{}

\begin{document}
\maketitle

\begin{abstract}
This paper presents the theory and analysis of a quaternion‐based multiplicative extended Kalman filter (Q‐MEKF) for rigid‐body attitude estimation using rate gyro, accelerometer, and magnetometer measurements.  Following Lefferts \emph{et al.}~\cite{Lefferts1982} and Markley~\cite{Markley2003}, the filter represents attitude errors multiplicatively on the unit‐quaternion manifold, achieves second‐order accuracy in state propagation, and maintains covariance consistency via Joseph‐form updates.  Detailed derivations of the state‐space model, measurement Jacobians, noise covariance selection, and numerical stability considerations are provided.
\end{abstract}

\section{Introduction}
Attitude estimation for vehicles and spacecraft relies on fusing inertial sensor data in a stable, consistent filter.  The quaternion multiplicative extended Kalman filter (Q‐MEKF) from Lefferts \emph{et al.}~\cite{Lefferts1982} avoids singularities of Euler angles by operating on quaternions and propagates small attitude errors additively in the tangent space~\cite{Markley2003}.  This work analyzes its mathematical structure, linearization, and covariance properties.

\section{State and Error Representation}
The true attitude is represented by a unit quaternion \(q\in\mathbb{H}\), evolving under body‐fixed angular rate \(\omega\).  The filter maintains a reference quaternion \(\hat q\) and a small error vector \(\delta\theta\in\mathbb{R}^3\) such that
\[
q = \hat q \otimes \delta q,\quad
\delta q \approx \begin{bmatrix}1\\\tfrac12\,\delta\theta\end{bmatrix},\ \|\delta\theta\|\ll1.
\]
If gyro bias \(b\) is estimated, the error state is \(\mathbf{x}=[\delta\theta^\top,\;b^\top]^\top\in\mathbb{R}^6\).

\section{Process Model}
The true quaternion obeys
\[
\dot q = \tfrac12\,q\otimes\begin{bmatrix}0\\\omega\end{bmatrix}.
\]
Subtracting bias, the measured rate is \(\omega_m = \omega + \nu_g - b\).  The reference quaternion propagates as
\[
\hat q_{k+1} = \hat q_k \otimes \exp\!\Bigl(\tfrac12(\omega_m)\Delta t\Bigr),
\]
and the error evolves linearly:
\[
\delta\theta_{k+1} \approx \delta\theta_k - (\omega_m)\,\Delta t - b\,\Delta t + w_k,
\]
where \(w_k\sim\mathcal{N}(0,Q)\) accounts for gyro noise and bias random‐walk.

\section{Measurement Model}
Normalized accelerometer and magnetometer readings provide observations of gravity \(\mathbf{g}\) and local field \(\mathbf{m}\) in body frame:
\[
y = \begin{bmatrix}a_m\\m_m\end{bmatrix}
\approx \begin{bmatrix}\hat R^\top\mathbf{g}\\\hat R^\top\mathbf{m}\end{bmatrix} + v,\quad v\sim\mathcal{N}(0,R).
\]
Here \(\hat R\) is the rotation matrix corresponding to \(\hat q\).  Linearizing about the reference yields
\[
y - \hat y \;\approx\; H\,\delta x + v,
\]
with Jacobian
\[
H = \begin{bmatrix}
-[\hat R^\top\mathbf{g}]_\times & \mathbf{0}_{3\times3}\\
-[\hat R^\top\mathbf{m}]_\times & \mathbf{0}_{3\times3}
\end{bmatrix},
\]
where \([\,\cdot\,]_\times\) denotes the skew‐symmetric matrix.

\section{Kalman Filter Equations}
\subsection{Time Update}
\begin{align*}
\hat x^-_k &= F\,\hat x_{k-1},\\
P^-_k &= F\,P_{k-1}F^\top + Q,
\end{align*}
with 
\[
F = \begin{bmatrix}
I_3 - [\omega_m]_\times\Delta t & -I_3\,\Delta t\\
0_{3\times3} & I_3
\end{bmatrix}.
\]

\subsection{Measurement Update}
Innovation:
\[
\nu_k = y_k - \hat y_k,\quad
S_k = H\,P^-_k H^\top + R.
\]
Kalman gain and Joseph‐form covariance:
\begin{align*}
K_k &= P^-_k H^\top S_k^{-1},\\
\hat x_k &= \hat x^-_k + K_k\,\nu_k,\\
P_k &= (I-K_kH)\,P^-_k\,(I-K_kH)^\top + K_k\,R\,K_k^\top.
\end{align*}
After update, the reference quaternion is corrected:
\[
\hat q_k = \hat q^-_k \otimes \begin{bmatrix}1\\\tfrac12\,\delta\theta_k\end{bmatrix}, 
\quad \delta\theta_k \leftarrow 0.
\]

\section{Noise Covariance Selection}
Process noise \(Q = \mathrm{diag}(\sigma_g^2\Delta t^2,\,\sigma_b^2\Delta t)\) reflects gyro white noise and bias instability.  Measurement noise \(R = \mathrm{diag}(\sigma_a^2,\sigma_m^2)\) follows sensor datasheet values.

\section{Numerical Stability}
\begin{itemize}
  \item \textbf{Joseph‐form update} preserves symmetry and positive definiteness of \(P\) in finite precision~\cite{Maybeck1979}.
  \item \(\hat q\) is renormalized after each propagation and correction.
  \item Small‐angle approximation is valid for \(\|\delta\theta\|\ll1\).
\end{itemize}

\section{Discussion}
The Q‐MEKF combines robustness of multiplicative error representation with the optimality of Kalman filtering.  Unlike additive‐error EKFs, it preserves the quaternion manifold and avoids singularities.

\section*{References}
\begin{thebibliography}{9}
\bibitem{Lefferts1982}
E.~J. Lefferts, F.~L. Markley, and M.~D. Shuster, “Kalman filtering for spacecraft attitude estimation,” \emph{J. Guidance, Control, and Dynamics}, vol.~5, no.~5, pp. 417–429, 1982.

\bibitem{Markley2003}
F.~L. Markley, “Attitude error representations for Kalman filtering,” \emph{J. Guidance, Control, and Dynamics}, vol.~26, no.~2, pp. 311–317, 2003.

\bibitem{Maybeck1979}
P.~S. Maybeck, \emph{Stochastic Models, Estimation, and Control}, vol.~1. Academic Press, 1979.

\end{thebibliography}

\end{document}
