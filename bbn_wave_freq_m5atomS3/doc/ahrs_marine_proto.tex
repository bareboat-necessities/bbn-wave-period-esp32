\documentclass[11pt,letterpaper]{article}
\usepackage{amsmath,amssymb,authblk,fullpage}
\usepackage{siunitx}
\usepackage{hyperref}
\usepackage{cite}

\title{Design and Implementation of a Microcontroller-Based Marine AHRS with Real-Time Wave Motion Estimation}
\author{Mikhail Grushinskiy}
\affil{Independent Researcher, 2025}

\begin{document}
\maketitle

\begin{abstract}
A compact Attitude and Heading Reference System (AHRS) enhanced with real-time wave motion estimation has been developed on an ESP32-S3 (M5AtomS3).  The system integrates: (1) a Mahony or quaternion EKF for three-axis orientation; (2) zero-crossing, Aranovskiy, or adaptive-notch filters for instantaneous wave frequency; (3) a trochoidal-model Kalman filter for heave estimation and bias correction; and (4) complementary or Kalman-based methods for wave direction.  This paper presents the underlying mathematical models, filter structures, and real-time implementation details.
\end{abstract}

\section{Introduction}
Small marine platforms benefit from accurate estimates of vessel attitude and wave-induced motions.  An embedded system was implemented using an ESP32-S3 microcontroller and an onboard IMU (MPU6886/BMI270).  The objective is simultaneous estimation of orientation, wave frequency, vertical displacement (heave), and wave propagation direction in real time.

\section{Attitude Estimation}
Body-frame gyroscope measurements \(\boldsymbol\omega_b\) and accelerometer readings \(\mathbf{a}_b\) are fused to obtain the rotation quaternion \(\mathbf{q}_{b\to n}\).

\subsection{Mahony Complementary Filter}
Angular rate integration is corrected by the gravity vector error:
\[
\dot{\hat{\mathbf q}}
= \tfrac12\,\hat{\mathbf q}\!\otimes\!\begin{bmatrix}0\\\boldsymbol\omega_b\end{bmatrix}
- K_P\,\hat{\mathbf q}\!\otimes\!(\hat{\mathbf v}_g\times\mathbf{a}_b)
- K_I\!\int\!(\hat{\mathbf v}_g\times\mathbf{a}_b)\,dt,
\]
where \(\hat{\mathbf v}_g\) is the estimated gravity direction.  Proportional and integral gains \(K_P,K_I\) determine filter bandwidth and steady-state error .

\subsection{Quaternion MEKF}
A multiplicative extended Kalman filter augments the quaternion with gyro bias \(\mathbf{b}\) and enforces the unit‐norm constraint via a linearized measurement update .

\section{Wave Frequency Estimation}
Accurate angular frequency \(\omega = 2\pi f\) is required by subsequent filters.  Three estimators are supported:

\subsection{Schmitt-Trigger Zero-Crossing}
The normalized vertical acceleration \(s(t)=a_z/A\) is thresholded at \(\pm h\).  Rising‐edge timestamps \(t_i\) yield period estimates
\[
T_k = \frac{2\,(t_k - t_{k-N_c})}{N_c - 1},\quad f_k = 1/T_k,
\]
following the method of Slita \emph{et al.} .

\subsection{Aranovskiy Recursive Filter}
A recursive estimator minimizes phase error of \(s(t)\approx\cos(2\pi\hat f t)\) using a gradient/RLS approach .

\subsection{Adaptive Notch Kalman Filter (KalmANF)}
A single-parameter adaptive notch is tuned in a Kalman framework to track \(\omega\) .

\section{Heave Estimation}
After gravity removal by the AHRS, vertical acceleration \(a_z\) drives a 5-state Kalman filter under the trochoidal wave model \(a=-\omega^2y\) .  The state vector
\(\mathbf{x}=[z,y,v,a,\hat a]^\top\) evolves according to a Taylor-expanded transition matrix \(F\).  Measurements
\(\mathbf{z}=[0,\;a_{\rm meas}]^\top\) enforce zero-mean displacement integral and observe acceleration plus bias.  The Joseph form update preserves numerical stability .

\section{Wave Direction Estimation}
Horizontal accelerations \((a_x,a_y)\) at known \(\omega\) permit two schemes:

\subsection{Kalman Amplitude Filter}
Amplitude vector \(\mathbf{A}\) follows a random walk with observations \(\mathbf{z}=\mathbf{A}\cos(\omega t)+\nu\), yielding \(\hat{\mathbf A}\) and unit direction \(\mathbf{d}=\hat{\mathbf A}/\|\hat{\mathbf A}\|\).

\subsection{Complementary Proxy Filter}
Proxy \(p=(\pm\|\mathbf{a}_h\|)\,\dot a_z\) is smoothed via
\[
P_k=(1-\alpha)P_{k-1}+\alpha\,p_k,
\]
with thresholding for direction decision.

\section{Rolling Min–Max Filter}
Wave height and mean heave derive from a sliding‐window min–max filter using Lemire’s monotonic wedge algorithm, with enhancements by Balster.  This achieves amortized \(O(1)\) per‐sample complexity .

\section{Real-Time Data Flow}
At each 200 Hz IMU interrupt:
\begin{enumerate}
  \item Acquire and spike‐filter \(\mathbf{a}_b,\boldsymbol\omega_b\).
  \item Update AHRS quaternion.
  \item Rotate accelerations to navigation frame.
  \item Estimate \(\omega\) via selected frequency filter.
  \item Execute heave Kalman filter.
  \item Execute direction filter.
  \item Update min–max for wave height and mean.
  \item Every 125 ms, emit NMEA‐0183 sentences for heave, frequency, direction, and attitude.
\end{enumerate}

All computations complete within \(<5\) ms, ensuring real-time performance on the M5AtomS3.

\begin{thebibliography}{10}

\bibitem{Sharkh2014}
Sharkh, S., Hendijanizadeh, M., Moshrefi-Torbati, M., \& Abusara, M. (2014, August). 
A novel Kalman filter based technique for calculating the time history of vertical displacement of a boat from measured acceleration. 
\textit{Marine Engineering Frontiers (MEF)}, \textit{2}.

\bibitem{Mahony2008}
R.~Mahony, T.~Hamel, and J.~P. Pflimlin, “Nonlinear complementary filters on the special orthogonal group,” \emph{IEEE Trans. Autom. Control}, vol.~53, no.~5, pp. 1203–1218, 2008.

\bibitem{Markley2003}
F.~L. Markley, “Attitude error representations for Kalman filtering,” \emph{J. Guid., Control, Dyn.}, vol.~26, no.~2, pp. 311–317, 2003.

\bibitem{Bobtsov2013}
A.~A. Bobtsov, N.~A. Nikolaev, O.~V. Slita, A.~S. Borgul, and S.~V. Aranovskiy, “The new algorithm of sinusoidal signal frequency estimation,” in \emph{11th IFAC Int. Workshop on Adaptation and Learning in Control and Signal Processing}, 2013, pp. 99–104.

\bibitem{Ali2023}
R.~Ali and T.~van Waterschoot, “A frequency tracker based on a Kalman filter update of a single parameter adaptive notch filter (KalmANF),” in \emph{Proc. 26th Int. Conf. on Digital Audio Effects (DAFx23)}, 2023.

\bibitem{Fenton1988}
J.~D. Fenton, “The numerical solution of steady water‐wave problems,” \emph{Computers \& Geosciences}, vol.~14, no.~3, pp. 357–368, 1988.

\bibitem{Gerstner1802}
F.~J. von Gerstner, “Theory of water waves,” \emph{Annalen der Physik}, vol.~9, no.~6, pp. 412–445, 1802.

\bibitem{Maybeck1979}
P.~S. Maybeck, \emph{Stochastic Models, Estimation, and Control}, vol.~1. Academic Press, 1979.

\bibitem{Yee2000}
R.~Yee and G.~S.~R. Murthy, “Kalman filtering of discretely sampled attitude data,” \emph{AIAA J. Guid., Control, Dyn.}, vol.~23, no.~4, pp. 664–666, 2000.

\bibitem{Lemire2006}
D.~T. Lemire, “Streaming maximum–minimum filter using no more than three comparisons per element,” \emph{arXiv preprint arXiv:cs/0610046}, 2006.

\bibitem{Balster2016}
E.~Balster, “STL compatible monotonic wedge for fast rolling min/max,” GitHub repository, 2016. \url{https://github.com/EvanBalster/STL_mono_wedge}

\end{thebibliography}

\end{document}
