\documentclass[12pt,letterpaper]{article}
\usepackage{amsmath,amssymb,fullpage}
\usepackage{hyperref}
\usepackage{cite}
\usepackage{mathabx}
\usepackage{mathtools}
\usepackage{physics}

\title{Implementation-Exact Analysis of the Mahony Complementary AHRS Algorithm}
\date{}

\begin{document}
\maketitle

\begin{abstract}
The Mahony filter is widely used in drones, robotics, and VR due to its robustness and low computational cost. The Mahony complementary filter fuses gyroscope and accelerometer data via proportional–integral feedback on the quaternion manifold.  This document derives every step exactly as implemented in the reference C code: accelerometer normalization, discrete PI feedback, fast inverse‐sqrt, quaternion integration, integral‐windup logic, and Euler‐angle extraction.  Numerical stability and computational cost are also analyzed.
\end{abstract}

\section{Introduction}
Inertial attitude estimation commonly blends high-frequency gyro integration with low-frequency accelerometer corrections~\cite{Mahony2008}.  The Mahony algorithm applies PI feedback directly to the quaternion rate, avoiding singularities and maintaining unit norm at minimal computational cost.

\section{Notation and State}
Let the estimated attitude be the unit quaternion 
\[
\hat q = \begin{bmatrix}\hat q_0\\\hat q_1\\\hat q_2\\\hat q_3\end{bmatrix},
\]
and raw gyro measurement 
\[
\boldsymbol\omega_m = (g_x,g_y,g_z),
\]
corrupted by bias $\mathbf b$ and noise.  Feedback gains are stored as 
\[
\texttt{twoKp} = 2K_P,\quad \texttt{twoKi} = 2K_I.
\]
Integral error accumulators are 
\(\mathbf I=(I_x,I_y,I_z)\), initially zero.

\section{Accelerometer Normalization}
On each update, read $(a_x,a_y,a_z)$.  If $(a_x,a_y,a_z)\neq(0,0,0)$, compute
\[
\text{recipNorm} = \mathrm{invSqrt}(a_x^2+a_y^2+a_z^2),
\quad
(a_x,a_y,a_z)\leftarrow (a_x,a_y,a_z)\,\text{recipNorm}.
\]
If the vector is zero, skip feedback to avoid NaNs.

\section{Feedback Computation}
Compute the estimated gravity direction in body frame:
\[
\begin{aligned}
\hat v_x &= 2(\hat q_1\hat q_3 - \hat q_0\hat q_2),\\
\hat v_y &= 2(\hat q_0\hat q_1 + \hat q_2\hat q_3),\\
\hat v_z &= 2(\hat q_0^2 - \tfrac12 + \hat q_3^2).
\end{aligned}
\]
Form the error vector via cross-product:
\[
(e_x,e_y,e_z)
= (a_y\,\hat v_z - a_z\,\hat v_y,\;
  a_z\,\hat v_x - a_x\,\hat v_z,\;
  a_x\,\hat v_y - a_y\,\hat v_x).
\]
Proportional and integral feedback modify the rate:
\[
\begin{aligned}
\mathbf I &\leftarrow
\begin{cases}
\mathbf I + \texttt{twoKi}\, \mathbf e\,\Delta t, &\text{if }\texttt{twoKi}>0,\\
(0,0,0), &\text{otherwise},
\end{cases}\\
(g_x,g_y,g_z) &\leftarrow (g_x,g_y,g_z)
+ \texttt{twoKp}\,\mathbf e
+ \mathbf I.
\end{aligned}
\]

\section{Discrete Quaternion Update}
Pre-multiply feedback rates:
\[
(g_x',g_y',g_z') = \tfrac12\,\Delta t\,(g_x,g_y,g_z).
\]
Denote the old quaternion $(q_0,q_1,q_2,q_3)$.  Then update:
\[
\begin{aligned}
q_0 &\leftarrow q_0 - q_1\,g_x' - q_2\,g_y' - q_3\,g_z',\\
q_1 &\leftarrow q_1 + q_0\,g_x' + q_2\,g_z' - q_3\,g_y',\\
q_2 &\leftarrow q_2 + q_0\,g_y' - q_1\,g_z' + q_3\,g_x',\\
q_3 &\leftarrow q_3 + q_0\,g_z' + q_1\,g_y' - q_2\,g_x'.
\end{aligned}
\]
Normalize:
\[
\text{recipNorm} = \mathrm{invSqrt}(q_0^2+q_1^2+q_2^2+q_3^2),\quad
(q_0,\dots,q_3)\leftarrow(q_0,\dots,q_3)\,\text{recipNorm}.
\]

\section{Attitude Extraction}
The Euler angles (degrees) are computed as:
\[
\begin{aligned}
\text{pitch} &=
\sin^{-1}(-2\,q_1q_3 + 2\,q_0q_2)\times\mathrm{RAD\_TO\_DEG},\\
\text{roll}  &=
\atan2(2\,q_2q_3 + 2\,q_0q_1,\;
       -2\,q_1^2 -2\,q_2^2 +1)\times\mathrm{RAD\_TO\_DEG},\\
\text{yaw}   &=
\atan2(2(q_1q_2+q_0q_3),\;
       q_0^2+q_1^2 - q_2^2 - q_3^2)\times\mathrm{RAD\_TO\_DEG}.
\end{aligned}
\]

\section{Fast Inverse-Sqrt}
The code uses the “magic‐number” Newton step:
\[
y = \mathrm{invSqrt}(x)
\quad\Longleftrightarrow\quad
\begin{cases}
i = 0x5f375a86 - (i\,\gg1),\\
y \leftarrow y\,[1.5 - 0.5\,x\,y^2].
\end{cases}
\]
This achieves one iteration of Newton–Raphson in constant time~\cite{Quake2002}.

\section{Numerical Stability}
\begin{itemize}
  \item Quaternion is renormalized every cycle.
  \item Integral terms are zeroed when \texttt{twoKi}=0 to prevent windup.
  \item All divisions by zero are avoided via checks and the fast‐invSqrt guard.
\end{itemize}

\paragraph{Representation Singularities}  
While the quaternion state remains nonsingular, Euler-angle outputs exhibit gimbal lock at $\mathrm{pitch} = \pm90^\circ$. Near this configuration:  
\begin{itemize}  
  \item The $\mathrm{atan2}$ terms in roll/yaw become numerically ill-conditioned.  
  \item Small accelerometer noise can cause large jumps in extracted roll/yaw.  
\end{itemize}  
Mitigations include: (1) thresholding pitch away from $\pm90^\circ$, (2) using quaternions for control, or (3) switching to rotation matrices.

\section{Computational Cost}
Each update executes:
\begin{itemize}
  \item One invSqrt and vector normalization (\(\approx10\) flops).
  \item Cross products and PI updates (\(\approx15\) flops).
  \item Four-component quaternion update (\(\approx20\) flops).
  \item Three \(\atan2\), one \(\asin\) for Euler (\(\mathcal O(1)\) but costly).
\end{itemize}
Total is constant per step, suitable for microcontrollers.

\begin{thebibliography}{9}
\bibitem{Mahony2008}
R.~Mahony, T.~Hamel, and J.~P. Pflimlin, “Nonlinear complementary filters on the special orthogonal group,” \emph{IEEE Trans. Autom. Control}, vol.~53, no.~5, pp. 1203–1218, 2008.

\bibitem{Madgwick2010}
S.~O.~H. Madgwick, “An efficient orientation filter for inertial and inertial/magnetic sensor arrays,” 2010.

\bibitem{Quake2002}
J.~W. Davies, “Fast inverse square root,” in \emph{Graphics Gems IV}, 2002.

\end{thebibliography}

\end{document}
