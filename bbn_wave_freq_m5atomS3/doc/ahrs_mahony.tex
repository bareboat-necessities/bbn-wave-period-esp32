\documentclass[12pt]{article}
\usepackage{amsmath,amssymb,fullpage}
\usepackage{hyperref}
\usepackage{cite}

\title{Mathematical Analysis of the Mahony Complementary AHRS Algorithm}
\date{}

\begin{document}
\maketitle

\begin{abstract}
The Mahony complementary filter provides an efficient attitude estimation scheme by fusing gyroscope and accelerometer data via proportional–integral feedback on the quaternion manifold.  This paper presents a detailed derivation of its error dynamics, gain selection criteria, numerical stability properties, and computational complexity.  Comparisons to related complementary and Kalman filters are drawn.
\end{abstract}

\section{Introduction}
Attitude estimation from inertial sensors can be achieved by complementary filters combining high‐frequency gyro integration with low‐frequency accelerometer corrections~\cite{Mahony2008}.  The Mahony algorithm implements PID‐style feedback directly on the quaternion representation, avoiding singularities and preserving unit norm.

\section{Quaternion Kinematics}
The true attitude quaternion \(q\) evolves under body‐fixed angular rate \(\boldsymbol\omega=(\omega_x,\omega_y,\omega_z)\) as
\begin{equation}
\dot q \;=\; \tfrac12\,q \otimes \begin{bmatrix}0\\\boldsymbol\omega\end{bmatrix},
\label{eq:quat_kin}
\end{equation}
where \(\otimes\) denotes quaternion multiplication.

\section{Mahony Filter Formulation}
Define the estimate quaternion \(\hat q=(\hat q_0,\hat{\mathbf q}_v)\) and raw gyro measurement \(\boldsymbol\omega_m=\boldsymbol\omega+\mathbf{b}+\boldsymbol\nu_g\), with bias \(\mathbf{b}\) and noise \(\boldsymbol\nu_g\).  Gravity measured by accelerometer is \(\mathbf a_m = R(\hat q)^\top\mathbf g + \boldsymbol\nu_a\), where \(R(\hat q)\) is the rotation matrix.

\subsection{Error Definition}
The attitude error quaternion \(\delta q\) satisfies
\[
q = \hat q \otimes \delta q,
\quad
\delta q \approx \begin{bmatrix}1\\\tfrac12\,\delta\boldsymbol\theta\end{bmatrix},\quad
\delta\boldsymbol\theta\in\mathbb R^3,\ \|\delta\boldsymbol\theta\|\ll1.
\]

\subsection{Feedback Terms}
Let the estimated gravity direction be
\[
\hat{\mathbf v}_g
= \begin{bmatrix}
2(\hat q_1\hat q_3 - \hat q_0\hat q_2)\\
2(\hat q_0\hat q_1 + \hat q_2\hat q_3)\\
\hat q_0^2 - \hat q_1^2 - \hat q_2^2 + \hat q_3^2
\end{bmatrix}.
\]
The error vector
\[
\mathbf e \;=\; \mathbf a_m \times \hat{\mathbf v}_g
\]
drives proportional and integral corrections:
\[
\boldsymbol\omega_c
= \boldsymbol\omega_m
+ K_P\,\mathbf e
+ K_I\!\int_0^t\mathbf e(\tau)\,d\tau.
\]

\subsection{Filter Equations}
The update law becomes
\begin{align}
\dot{\hat q} &= \tfrac12\,\hat q \otimes \begin{bmatrix}0\\\boldsymbol\omega_c\end{bmatrix},\label{eq:propagate}\\
\dot{\mathbf z}_I &=  \mathbf e,\quad
\mathbf z_I(0)=\mathbf0,\label{eq:integral}
\end{align}
with gains \(K_P=2\,\kappa_P\), \(K_I=2\,\kappa_I\).  After integration, \(\hat q\) is renormalized.

\section{Error Dynamics}
Linearizing about \(\delta\boldsymbol\theta=\mathbf0\) yields
\[
\dot{\delta\boldsymbol\theta}
= -(\boldsymbol\omega_m - K_P\,\mathbf e - K_I\,\mathbf z_I)\,\Delta t + \boldsymbol\nu,
\]
and
\[
\dot{\mathbf z}_I = \mathbf e.
\]
Substituting \(\mathbf e \approx \delta\boldsymbol\theta\) gives a second‐order system:
\[
\ddot{\delta\boldsymbol\theta} + K_P\,\dot{\delta\boldsymbol\theta} + K_I\,\delta\boldsymbol\theta = \boldsymbol\nu_{\rm eff},
\]
which is stable for \(K_P,K_I>0\).

\section{Gain Selection}
The characteristic equation \(s^2 + K_P\,s + K_I=0\) implies natural frequency \(\omega_n=\sqrt{K_I}\) and damping \(\zeta=K_P/(2\sqrt{K_I})\).  Typical tuning sets \(\zeta\approx0.7\) and \(\omega_n\) near the cutoff between gyro and accelerometer bandwidths.

\section{Numerical Stability}
\begin{itemize}
  \item \textbf{Quaternion Normalization:} prevents drift off the unit sphere.
  \item \textbf{Integral Windup Avoidance:} integral reset when \(K_I=0\).
  \item \textbf{Low‐Rate Sensitivity:} small \(\Delta t\) ensures accurate integration of (\ref{eq:propagate}).
\end{itemize}

\section{Computational Complexity}
Each update requires:
\begin{itemize}
  \item Vector cross‐products (\(O(1)\) operations).
  \item Floating‐point multiplies and adds: \(\approx50\) flops.
  \item Inverse‐sqrt for normalization: one fast approximate \(O(1)\) routine~\cite{Quake2002}.
\end{itemize}
Overall cost per step is constant, suitable for real‐time execution on microcontrollers.

\section{Comparison to Related Methods}
\begin{itemize}
  \item \textbf{Madgwick Filter} adds magnetometer correction and uses gradient descent rather than PI feedback~\cite{Madgwick2010}.
  \item \textbf{Multiplicative EKF} yields statistically optimal estimates at higher computational cost~\cite{Lefferts1982,Markley2003}.
\end{itemize}

\section*{References}
\begin{thebibliography}{9}
\bibitem{Mahony2008}
R.~Mahony, T.~Hamel, and J.~P. Pflimlin, “Nonlinear complementary filters on the special orthogonal group,” \emph{IEEE Trans. Autom. Control}, vol.~53, no.~5, pp. 1203–1218, 2008.

\bibitem{Lefferts1982}
E.~J. Lefferts, F.~L. Markley, and M.~D. Shuster, “Kalman filtering for spacecraft attitude estimation,” \emph{Journal of Guidance, Control, and Dynamics}, vol.~5, no.~5, pp. 417–429, 1982.

\bibitem{Markley2003}
F.~L. Markley, “Attitude error representations for Kalman filtering,” \emph{Journal of Guidance, Control, and Dynamics}, vol.~26, no.~2, pp. 311–317, 2003.

\bibitem{Madgwick2010}
S.~O.~H. Madgwick, “An efficient orientation filter for inertial and inertial/magnetic sensor arrays,” \emph{Report x-io and University of Bristol (UK)}, 2010.

\bibitem{Quake2002}
J.~W. Davies, “Fast inverse square root,” \emph{Graphic Gems IV}, pp. 489–492, 2002.

\end{thebibliography}

\end{document}
