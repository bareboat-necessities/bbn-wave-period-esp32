\documentclass[12pt,letterpaper]{article}
\usepackage{amsmath,amssymb}
\usepackage{graphicx}
\usepackage{authblk}
\usepackage{fullpage}
\usepackage{physics}
\usepackage{siunitx}
\usepackage{bm}
\usepackage{cite}

\title{Heave Estimation from Vertical Acceleration using a Kalman Filter with Drift and Bias Compensation}
\author{Mikhail Grushinskiy}
\affil{Independent Researcher, 2025}

\begin{document}
\maketitle

\begin{abstract}
This paper presents a Kalman filter-based approach for estimating vertical displacement (heave) from vertical acceleration measurements. The method incorporates double integration, bias estimation, and drift correction via pseudo-measurements. A Schmitt trigger-like logic is employed to detect wave zero-crossings and correct state trajectories based on ocean wave physics. The algorithm is suitable for real-time applications such as wave-following buoys or onboard motion sensing.
\end{abstract}

\section{Introduction}
Estimating heave from vertical accelerometers is challenging due to integration drift, sensor bias, and wave irregularities. This paper proposes a solution using a Kalman filter that augments the state space with bias and drift components and corrects for mean-zero vertical displacement using a soft pseudo-measurement and a soft zero-crossing correction strategy. The idea of using soft pseudo-measurement for a drift correction was presented in \cite{Sharkh2014}. This paper improves correction further by modelling bias dynamics via a state variable and by introducing
a soft zero crossing correction which respects the physics of the process.

\section{State-Space Model}

The continuous-time vertical acceleration \( a(t) \) is sampled at time steps \( k \) with period \( T \). The state vector \( \mathbf{x}_k \in \mathbb{R}^4 \) is defined as:
\begin{equation}
\mathbf{x}_k = \begin{bmatrix}
z_k \\
y_k \\
v_k \\
\hat{a}_k
\end{bmatrix}
\end{equation}
where:
\begin{itemize}
  \item \( z_k \): Third integral of acceleration (integrated displacement)
  \item \( y_k \): Heave (vertical displacement)
  \item \( v_k \): Vertical velocity
  \item \( \hat{a}_k \): Estimated accelerometer bias
\end{itemize}



\subsection{Process Model Equations}

The discrete-time process model equations are:

\begin{align}
v_k &= v_{k-1} + a T - \hat{a}_{k-1} T \\
y_k &= y_{k-1} + v_{k-1} T + \frac{1}{2}a T^2 - \frac{1}{2}\hat{a}_{k-1} T^2 \\
z_k &= z_{k-1} + y_{k-1} T + \frac{1}{2}v_{k-1} T^2 + \frac{1}{6}a T^3 - \frac{1}{6}\hat{a}_{k-1} T^3 \\
\hat{a}_k &= \hat{a}_{k-1}
\end{align}

where $T$ is the sampling interval and $a$ is the measured acceleration, $a - \hat{a}$ is the true vertical acceleration, and the bias $\hat{a}$ is modeled as a constant.


\subsection{Matrix Formulation}

The process model in matrix form is:
\begin{equation}
\mathbf{x}_k = \mathbf{F}_k \mathbf{x}_{k-1} + \mathbf{B}_k a_k + \mathbf{w}_k
\end{equation}
where \( a_k \) is measured vertical acceleration and \( \mathbf{w}_k \sim \mathcal{N}(0, \mathbf{Q}) \) is the process noise with covariance \(  \bm{Q} \). The matrices \( \mathbf{F}_k \) (state transition) and \( \mathbf{B}_k \) (input matrix) are time step dependent:
\begin{equation}
\mathbf{F}_k =
\begin{bmatrix}
1 & T & \frac{1}{2}T^2 & -\frac{1}{6}T^3 \\
0 & 1 & T & -\frac{1}{2}T^2 \\
0 & 0 & 1 & -T \\
0 & 0 & 0 & 1
\end{bmatrix}
\end{equation}
\begin{equation}
\mathbf{B}_k =
\begin{bmatrix}
\frac{1}{6}T^3 \\
\frac{1}{2}T^2 \\
T \\
0
\end{bmatrix}
\end{equation}

\subsection{Measurement Model}
To correct for drift, the integral of displacement \( z_k \) is treated as a pseudo-measurement with expected value zero:
\[
z_k^\text{meas} = 0 + v_k, \quad v_k \sim \mathcal{N}(0, R)
\]
The measurement matrix:
\begin{equation}
\mathbf{H} = \begin{bmatrix} 1 & 0 & 0 & 0 \end{bmatrix}
\end{equation}



% ===== PREDICT STEP =====
\subsection{Predict Step}
\begin{equation}
\bm{x}_k^- = \bm{F}\bm{x}_{k-1} + \bm{B}a_k
\end{equation}

\begin{equation}
\bm{F} = 
\begin{bmatrix}
1      & T     & \frac{1}{2}T^2 & -\frac{1}{6}T^3 \\
0      & 1     & T              & -\frac{1}{2}T^2 \\
0      & 0     & 1              & -T              \\
0      & 0     & 0              & 1
\end{bmatrix}, \quad
\bm{B} = 
\begin{bmatrix}
\frac{1}{6}T^3 \\
\frac{1}{2}T^2 \\
T \\
0
\end{bmatrix}
\end{equation}

\begin{equation}
\bm{P}_k^- = \bm{F}\bm{P}_{k-1}\bm{F}^\top + \bm{Q}, \quad 
\bm{Q} = \mathrm{diag}(q_0, q_1, q_2, q_3)
\end{equation}

% ===== STANDARD UPDATE =====
\subsection{Standard Update}
Measurement: $z=0$ with $\bm{H} = [1\;0\;0\;0]$

\begin{align}
\tilde{y} &= 0 - \bm{H}\bm{x}_k^- \\
S &= \bm{H}\bm{P}_k^-\bm{H}^\top + R \\
\bm{K} &= \bm{P}_k^- \bm{H}^\top S^{-1} \\
\bm{x}_k^+ &= \bm{x}_k^- + \bm{K}\tilde{y} \\
\bm{P}_k^+ &= (\bm{I}-\bm{K}\bm{H})\bm{P}_k^-(\bm{I}-\bm{K}\bm{H})^\top + \bm{K}R\bm{K}^\top
\end{align}

% ===== ZERO-CROSSING CORRECTION =====
\subsection{Zero-Crossing Correction}
Trigger when $|a_k| > 0.04\,\text{m/s}^2$ and $|v| > 0.6\,\text{m/s}$:

\begin{equation}
\bm{H}_c = \begin{bmatrix}0&1&0&0\\0&0&1&0\end{bmatrix}, \quad
\bm{z}_c = \begin{bmatrix}(1-\alpha)y\\v+\alpha(\tilde{v}-v)\end{bmatrix}
\end{equation}

\begin{equation}
\tilde{v} = \mathrm{sgn}(v)\min\left(\sqrt{v^2 + (\pi y/\Delta t_\mathrm{zero})^2}, 3.0\right)
\end{equation}

\begin{align}
\bm{y}_c &= \bm{z}_c - \bm{H}_c\bm{x}_k^+ \\
\bm{S}_c &= \bm{H}_c\bm{P}_k^+\bm{H}_c^\top + \mathrm{diag}(R_y, R_v) \\
\bm{K}_c &= \bm{P}_k^+\bm{H}_c^\top \bm{S}_c^{-1} \quad \text{(if $\bm{y}_c^\top\bm{S}_c^{-1}\bm{y}_c < 13.0$)} \\
\bm{x}_k &\leftarrow \bm{x}_k + \bm{K}_c\bm{y}_c \\
\bm{P}_k &\leftarrow (\bm{I}-\bm{K}_c\bm{H}_c)\bm{P}_k(\bm{I}-\bm{K}_c\bm{H}_c)^\top + \bm{K}_c\bm{S}_c\bm{K}_c^\top
\end{align}

% ===== STABILIZATION =====
\subsection{Stabilization}
\begin{itemize}
\item Symmetry: $\bm{P} \leftarrow \frac{1}{2}(\bm{P}+\bm{P}^\top)$
\item PD enforcement: While $\bm{P} \not\succ 0$, add $\epsilon\bm{I}$
\item Numerical safety: $S^{-1}$ clamped at $10^{-12}$
\end{itemize}


\section{Schmitt Trigger-Based Correction}
To further constrain drift and improve estimation during wave oscillations, a Schmitt trigger-based logic is implemented. It detects zero-crossings of the vertical acceleration with hysteresis and debounce conditions.

Let \( a_k \) be the filtered acceleration and \( v_k \) the estimated velocity. A crossing event is detected when:
\[
a_k > \theta_{+}, \quad |v_k| > v_\text{thresh}, \quad \Delta t > t_\text{debounce}
\]
or
\[
a_k < \theta_{-}, \quad |v_k| > v_\text{thresh}, \quad \Delta t > t_\text{debounce}
\]

When a zero-crossing is detected, a soft correction is applied using an extended observation:
\[
\mathbf{H}_z =
\begin{bmatrix}
0 & 1 & 0 & 0 \\
0 & 0 & 1 & 0
\end{bmatrix}, \quad
\mathbf{z}_\text{corr} =
\begin{bmatrix}
(1 - \gamma)y_k \\
v_k + \gamma(\hat{v}_\text{target} - v_k)
\end{bmatrix}
\]
where \( \gamma \in [0,1] \) controls the correction strength and \( \hat{v}_\text{target} = \sqrt{v_k^2 + (\omega y_k)^2} \) based on estimated frequency.

\section{Implementation Notes}

This algorithm is implemented in C++ and uses the Eigen library. It is designed to run in embedded environments and includes:
\begin{itemize}
  \item Numerical conditioning of covariance matrices
  \item Mahalanobis distance gating
  \item Time-varying Schmitt trigger detection logic
  \item Optional theoretical process noise estimation from IMU specs
\end{itemize}

\section{Conclusion}
The proposed Kalman filter formulation provides robust heave estimation from vertical acceleration even in the presence of sensor drift and bias. The inclusion of pseudo-measurements and zero-crossing detection improves long-term stability and physical consistency.

\section*{Acknowledgements}
Implementation and theoretical design by Mikhail Grushinskiy, 2025.

\begin{thebibliography}{4}

\bibitem{Gerstner1809} 
F.~J. Gerstner, ``Theorie der Wellen,'' 
\emph{Annalen der Physik}, vol.~32, pp.~412–445, 1809.  
\emph{(English translation: }``Theory of Waves,'' \emph{Annual Reports of the Prague Polytechnic Institute}, 1847.)

\bibitem{Clamond2007} 
D.~Clamond and M.~Dutykh, ``Practical analytic approximation of trochoidal waves,'' 
\emph{Applied Ocean Research}, vol.~29, no.~4, pp.~213–220, 2007.

\bibitem{Sharkh2014}
Sharkh, S., Hendijanizadeh, M., Moshrefi-Torbati, M., \& Abusara, M. (2014, August). 
A novel Kalman filter based technique for calculating the time history of vertical displacement of a boat from measured acceleration. 
\textit{Marine Engineering Frontiers (MEF)}, \textit{2}.

\bibitem{Fenton1988}
J.~D.~Fenton, “The numerical solution of steady water wave problems,” \emph{Computers \& Geosciences}, vol.~14, no.~3, pp.~357–368, 1988.

\end{thebibliography}


\end{document}
