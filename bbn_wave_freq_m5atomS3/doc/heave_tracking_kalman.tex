\documentclass[11pt,letterpaper]{article}
\usepackage{amsmath,amssymb}
\usepackage{graphicx}
\usepackage{authblk}
\usepackage{fullpage}
\usepackage{physics}
\usepackage{siunitx}
\usepackage{bm}
\usepackage{cite}

\title{Heave Estimation from Vertical Acceleration using a Kalman Filter with Drift and Bias Compensation}
\author{Mikhail Grushinskiy}
\affil{Independent Researcher, 2025}

\setlength{\parskip}{1em}
\setlength{\parindent}{0pt}

\begin{document}
\maketitle

\begin{abstract}
This article presents a novel implementation of a Kalman filter-based algorithm for estimating vertical displacement (heave) from noisy vertical acceleration measurements. The method compensates for accelerometer bias and low-frequency drift inherent in double integration, while maintaining physical constraints using both pseudo-measurements and zero-crossing detection inspired by Schmitt trigger logic. The filter design includes a dynamic model of vertical motion and a novel mechanisms with zero-crossing and feedback corrections, bias compensation making it well-suited for marine wave sensing, vertical heave motion of a vessel, and inertial-based sea state estimation.
\end{abstract}

\section{Introduction}

Estimating vertical displacement (heave) from acceleration measurements is a longstanding challenge in inertial navigation. Direct double integration of accelerometer signals often leads to unbounded error growth due to sensor noise, bias, and numerical drift. These effects are especially problematic in marine environments, where accurate wave-induced heave estimation is required, but GPS or pressure-based references may be unavailable or unreliable.

This work presents a method based on a Kalman filter that estimates heave by integrating vertical acceleration while simultaneously estimating and compensating for accelerometer bias. The approach includes a physically sound mechanism for drift suppression and long-term stability, making it suitable for use in embedded real-time systems.

Vertical acceleration can be obtained from an IMU (Inertial Motion Unit) using an Attitude and Heading Reference System (AHRS), which estimates the orientation of the sensor relative to the Earth. AHRS algorithms use data from the accelerometer, gyroscope, and sometimes magnetometer to compute a rotation from the sensor frame to a global frame. 

Common AHRS algorithms include:

\begin{itemize}
\item \textbf{Madgwick filter} – a computationally efficient gradient-descent-based algorithm suitable for real-time systems.
\item \textbf{Mahony filter} – a complementary filter with integral feedback for robust orientation tracking.
\item \textbf{Extended Kalman Filter (EKF)} – a more complex approach that provides accurate orientation by modeling sensor noise and system dynamics.
\item \textbf{Complementary filters} – lightweight filters that combine high-pass and low-pass filtered signals from gyroscopes and accelerometers.
\end{itemize}

The core features of the proposed method are:

\begin{itemize}
    \item \textbf{Bias-aware kinematic model:} The state includes displacement, velocity, and accelerometer bias, enabling continuous bias correction within the filtering framework.
    
    \item \textbf{Drift mitigation using pseudo-measurements:} A third integral of acceleration --- the integral of displacement --- is maintained as part of the state vector. A pseudo-measurement enforces this term to be zero, which corresponds to constraining the long-term average displacement around sea level. This suppresses drift that would otherwise accumulate through direct numerical integration.
    
    \item \textbf{Zero-crossing feedback:} A Schmitt trigger-inspired mechanism detects zero-crossings in acceleration where vertical velocity is near its extreme. These events indicate turning points in the wave cycle. At these points, the filter applies a soft correction to align displacement and velocity with physically plausible wave dynamics.
    
    \item \textbf{Numerically stable covariance update:} The Kalman update uses the Joseph form, along with explicit symmetry and positive-definiteness enforcement, ensuring robust operation even under limited precision on embedded hardware.
\end{itemize}

The resulting filter is lightweight, real-time capable, and physically consistent. It is particularly well suited for applications such as wave buoy data analysis, ship motion monitoring, and low-power marine sensor nodes operating without absolute vertical references.

\paragraph{Physical Basis for Zero Integral of Displacement.}
The idea of using displacement integral as a soft pseudo-measurement for a drift correction was previously presented in \cite{Sharkh2014}. While it does solve the issue of long term drift with noisy IMU it's not enough to provide good convergence times. The method in the article presented here improves correction further by modelling bias dynamics via a state variable and by introducing
a soft zero-crossing correction which respects the physics of the process.

In ocean wave motion, the vertical displacement \( y(t) \) of a floating object such as a buoy or vessel oscillates around a fixed reference level, typically mean sea level. This implies that, over time, the average displacement is zero. Mathematically, this behavior can be expressed as
\begin{equation}
\lim_{T \to \infty} \frac{1}{T} \int_0^T y(t) \, dt = 0,
\end{equation}
which means that the integral of displacement over a sufficiently long time horizon tends toward zero:
\begin{equation}
\int_0^T y(t) \, dt \approx 0 \quad \text{for large } T.
\end{equation}
This is a physically grounded property of wave-induced vertical motion: although the object moves up and down due to wave energy, it does not exhibit a net upward or downward trend over time. By including this third integral of acceleration---the integral of displacement---as a state in the Kalman filter and softly constraining it to remain near zero, the algorithm imposes a natural long-term anchoring condition. This correction counteracts drift caused by accelerometer bias and low-frequency integration errors, leading to more stable and physically consistent displacement estimates.

\paragraph{Zero-Crossing Correction for Displacement and Velocity.}
In marine applications, the vertical motion of a floating platform or vessel naturally oscillates around mean sea level with no persistent offset. To better align with physical wave behavior, the filter uses a zero-crossing detection mechanism based on a Schmitt trigger. This system identifies points where acceleration crosses zero while velocity is near a peak, which corresponds to wave crests and troughs. These are used as soft correction points to help re-align the filter’s velocity and displacement estimates with real wave dynamics. This is conceptually similar to Zero Velocity Updates (ZUPT) used in step-counting and pedestrian navigation apps, where the foot’s contact with the ground resets velocity to zero. The difference is that in waves zero crossing of acceleration means that vertical velocity is not zero but it's at extreme.

Zero-crossing correction is applied when the vertical acceleration crosses zero and the vertical velocity magnitude exceeds a predefined threshold. These conditions typically correspond to turning points in the wave cycle—either crests or troughs—where the vertical displacement is at an extremum and the velocity is near its maximum. To maintain physically consistent behavior, the filter estimates what the velocity \emph{should be} at this point using the principle of energy conservation, assuming approximately harmonic motion.

In this framework, the total mechanical energy (kinetic plus potential) is conserved. The corrected velocity is computed using:

\begin{equation}
v_{\text{corrected}} = \sqrt{v^2 + (\omega y)^2},
\end{equation}

where \( v \) is the current vertical velocity estimate, \( y \) is the current vertical displacement (heave), and \( \omega \) is the estimated angular frequency of the wave. The angular frequency is inferred from the timing between zero-crossings of acceleration, assuming the time interval corresponds to half a wave period. Specifically,

\begin{equation}
\omega = \frac{\pi}{\Delta t_{z}},
\end{equation}

where \( \Delta t_{z} \) is the time since the last zero-crossing. This approximation is physically grounded in linear wave theory, where the relationship between displacement and velocity is sinusoidal and governed by wave frequency.

After estimating \( v_{\text{corrected}} \), the filter softly adjusts both \( y \) and \( v \) toward their corrected values using a configurable gain factor. The direction of velocity is preserved to maintain motion continuity. To ensure robustness, the correction is only applied if it passes a Mahalanobis distance check, verifying statistical consistency with the filter's uncertainty model. This correction mechanism reduces drift, maintains phase alignment with true wave motion, and supports long-term accuracy in the absence of absolute references.

\paragraph{Including accelerometer bias}
as part of the state vector is critical for accurate and stable heave estimation. Even a small constant bias in the vertical acceleration signal, when integrated twice, causes displacement to drift unbounded over time. While using the third integral of acceleration (i.e., the integral of displacement) and enforcing it to be approximately zero helps suppress long-term drift, this alone is not enough. Without explicitly modeling the bias, the Kalman filter cannot attribute the drift to its true source. Instead, it misattributes the error to other states like velocity or displacement, leading to unrealistic estimates. By estimating bias directly, the filter can learn and compensate for it continuously, maintaining physical consistency and reducing long-term error.

\paragraph{Numerical stability}
is vital when implementing the filter on embedded or resource-constrained systems. Rounding errors, particularly in the covariance matrix, can lead to loss of symmetry or positive-definiteness, causing the filter to diverge or behave erratically especially in long runs. To address this, the filter uses the Joseph stabilized form of the Kalman update, which is more robust against numerical errors. It also explicitly enforces symmetry and positive-definiteness of covariance matrices using Cholesky decomposition and small regularization terms when needed. These safeguards ensure that the filter remains stable and reliable even on 32-bit microcontrollers with limited floating-point precision.

% ===== Kalman Filter =====
\section{Kalman Filter Equations (Linear, Discrete-Time)}

We consider a linear discrete-time system governed by the following process and measurement models:

\paragraph{State transition model:}
\begin{equation}
    \mathbf{x}_k = \mathbf{F}_k \, \mathbf{x}_{k-1} + \mathbf{B}_k \, \mathbf{u}_k + \mathbf{w}_k
\end{equation}
\begin{equation}
    \mathbf{w}_k \sim \mathcal{N}(\mathbf{0}, \mathbf{Q}_k)
\end{equation}

\paragraph{Measurement model:}
\begin{equation}
    \mathbf{z}^{\text{meas}}_k = \mathbf{H}_k \, \mathbf{x}_k + \boldsymbol{\nu}_k
\end{equation}
\begin{equation}
    \boldsymbol{\nu}_k \sim \mathcal{N}(\mathbf{0}, \mathbf{R}_k)
\end{equation}

\paragraph{Variables:}
\begin{itemize}
    \item \( \mathbf{x}_k \in \mathbb{R}^n \): state vector
    \item \( \mathbf{u}_k \in \mathbb{R}^m \): control input
    \item \( \mathbf{z}^{\text{meas}}_k \in \mathbb{R}^p \): measurement vector
    \item \( \mathbf{F}_k \in \mathbb{R}^{n \times n} \): state transition matrix
    \item \( \mathbf{B}_k \in \mathbb{R}^{n \times m} \): control input matrix
    \item \( \mathbf{H}_k \in \mathbb{R}^{p \times n} \): observation matrix
    \item \( \mathbf{Q}_k \in \mathbb{R}^{n \times n} \): process noise covariance
    \item \( \mathbf{R}_k \in \mathbb{R}^{p \times p} \): measurement noise covariance
\end{itemize}

\subsection*{Matrix Dimensions}
\begin{itemize}
  \item $n$: Dimension of the state vector $\bm{x}_k$
  \item $m$: Dimension of the measurement vector $\bm{z}_k^{\text{meas}}$
  \item $p$: Dimension of the control input vector $\bm{u}_k$
\end{itemize}

\subsubsection*{Prediction Step}
\begin{align}
    \hat{\mathbf{x}}_{k|k-1} &= \mathbf{F}_k \, \hat{\mathbf{x}}_{k-1|k-1} + \mathbf{B}_k \, \mathbf{u}_k \\
    \mathbf{P}_{k|k-1} &= \mathbf{F}_k \, \mathbf{P}_{k-1|k-1} \, \mathbf{F}_k^\top + \mathbf{Q}_k
\end{align}

\textbf{Where:}
\begin{itemize}
  \item $\bm{x}_{k|k-1}$: Predicted state estimate at time $k$
  \item $\bm{x}_{k-1|k-1}$: Posterior state estimate at time $k-1$
  \item $\mathbf{F}_k$: State transition matrix
  \item $\mathbf{B}_k$: Control input matrix
  \item $\bm{u}_k$: Control input vector at time $k$
  \item $\mathbf{P}_{k|k-1}$: Predicted state covariance
  \item $\mathbf{P}_{k-1|k-1}$: Posterior state covariance at time $k-1$
  \item $\mathbf{Q}_k$: Process noise covariance matrix
\end{itemize}

\subsubsection*{Update Step}
\begin{align}
    \mathbf{y}_k &= \mathbf{z}^{\text{meas}}_k - \mathbf{H}_k \, \hat{\mathbf{x}}_{k|k-1} \\
    \mathbf{S}_k &= \mathbf{H}_k \, \mathbf{P}_{k|k-1} \, \mathbf{H}_k^\top + \mathbf{R}_k \\
    \mathbf{K}_k &= \mathbf{P}_{k|k-1} \, \mathbf{H}_k^\top \, \mathbf{S}_k^{-1} \\
    \hat{\mathbf{x}}_{k|k} &= \hat{\mathbf{x}}_{k|k-1} + \mathbf{K}_k \, \mathbf{y}_k \\
    \mathbf{P}_{k|k} &= (\mathbf{I} - \mathbf{K}_k \mathbf{H}_k) \, \mathbf{P}_{k|k-1} \, (\mathbf{I} - \mathbf{K}_k \mathbf{H}_k)^\top + \mathbf{K}_k \mathbf{R}_k \mathbf{K}_k^\top
\end{align}

\textbf{Where:}
\begin{itemize}
  \item $\bm{z}^{\text{meas}}_k$: Measurement vector at time $k$
  \item $\mathbf{H}_k$: Observation matrix
  \item $\bm{y}_k$: Innovation (residual), difference between measurement and prediction
  \item $\mathbf{S}_k$: Innovation covariance
  \item $\mathbf{R}_k$: Measurement noise covariance
  \item $\mathbf{K}_k$: Kalman gain
  \item $\bm{x}_{k|k}$: Updated (posterior) state estimate at time $k$
  \item $\mathbf{P}_{k|k}$: Updated (posterior) estimate covariance using the Joseph form
  \item $\mathbf{I}$: Identity matrix of appropriate size
\end{itemize}

\paragraph{Notes:}
Equation \( {\mathbf{P}}_{k|k} \) is the \textbf{Joseph stabilized form} of the covariance update, which maintains positive semi-definiteness and improves numerical stability.


% ===== Kalman Filter for Wave =====

\section{Kalman Model for Wave Dynamics}

\subsection*{State-Space Model}
The continuous-time vertical acceleration \( a = a(t) \) is sampled at time steps \( k \) with period \( T \). The state vector \( \mathbf{x}_k \in \mathbb{R}^4 \) is defined as:
\begin{equation}
\mathbf{x}_k = \begin{bmatrix}
z_k \\
y_k \\
v_k \\
\hat{a}_k
\end{bmatrix}
\end{equation}
where:
\begin{itemize}
  \item \( z_k \): Third integral of acceleration (integrated displacement)
  \item \( y_k \): Heave (vertical displacement)
  \item \( v_k \): Vertical velocity
  \item \( \hat{a}_k \): Estimated accelerometer bias
\end{itemize}


\subsection*{Process Model Equations}
The discrete-time process model equations are:

\begin{align}
v_k &= v_{k-1} + a T - \hat{a}_{k-1} T \\
y_k &= y_{k-1} + v_{k-1} T + \frac{1}{2}a T^2 - \frac{1}{2}\hat{a}_{k-1} T^2 \\
z_k &= z_{k-1} + y_{k-1} T + \frac{1}{2}v_{k-1} T^2 + \frac{1}{6}a T^3 - \frac{1}{6}\hat{a}_{k-1} T^3 \\
\hat{a}_k &= \hat{a}_{k-1}
\end{align}

where $T$ is the sampling interval and $a$ is the measured acceleration, $a - \hat{a}$ is the true vertical acceleration, and the bias $\hat{a}$ is modeled as a constant.

\subsection*{Process Model Matrix Formulation}
The matrices \( \mathbf{F}_k \) (state transition) and \( \mathbf{B}_k \) (input matrix) are time step dependent:
\begin{equation}
\mathbf{F}_k =
\begin{bmatrix}
1 & T & \frac{1}{2}T^2 & -\frac{1}{6}T^3 \\
0 & 1 & T & -\frac{1}{2}T^2 \\
0 & 0 & 1 & -T \\
0 & 0 & 0 & 1
\end{bmatrix}
\end{equation}
\begin{equation}
\mathbf{B}_k =
\begin{bmatrix}
\frac{1}{6}T^3 \\
\frac{1}{2}T^2 \\
T \\
0
\end{bmatrix}
\end{equation}

\subsection*{Standard Kalman Update}
To correct for drift, the integral of displacement \( z_k \) is treated as a pseudo-measurement with expected value zero:
\begin{equation}
z_k^\text{meas} = 0 + \nu_k, \quad \nu_k \sim \mathcal{N}(0, R)
\end{equation}
The measurement matrix:
\begin{equation}
\mathbf{H} = \begin{bmatrix} 1 & 0 & 0 & 0 \end{bmatrix}
\end{equation}

Process noise covariance matrix:
\begin{equation}
\bm{Q} = \mathrm{diag}(q_0, q_1, q_2, q_3)
\end{equation}

Measurement noise covariance:
\begin{equation}
\bm{R} = \mathrm{diag}(r_0)
\end{equation}

% ===== Zero-crossing Correction =====
\subsection{Zero-Crossing Correction}
Trigger when $|a_k| > 0.04\,\text{m/s}^2$ and $|v| > 0.6\,\text{m/s}$:

\begin{equation}
\bm{H}_c = \begin{bmatrix}0&1&0&0\\0&0&1&0\end{bmatrix}, \quad
\bm{z}_c = \begin{bmatrix}(1-\alpha)y\\v+\alpha(\tilde{v}-v)\end{bmatrix}
\end{equation}

\begin{equation}
\tilde{v} = \mathrm{sgn}(v)\min\left(\sqrt{v^2 + (\pi y/\Delta t_\mathrm{zero})^2}, 3.0\right)
\end{equation}

\begin{align}
\bm{y}_c &= \bm{z}_c - \bm{H}_c\bm{x}_k^+ \\
\bm{S}_c &= \bm{H}_c\bm{P}_k^+\bm{H}_c^\top + \mathrm{diag}(R_y, R_v) \\
\bm{K}_c &= \bm{P}_k^+\bm{H}_c^\top \bm{S}_c^{-1} \quad \text{(if $\bm{y}_c^\top\bm{S}_c^{-1}\bm{y}_c < 13.0$)} \\
\bm{x}_k &\leftarrow \bm{x}_k + \bm{K}_c\bm{y}_c \\
\bm{P}_k &\leftarrow (\bm{I}-\bm{K}_c\bm{H}_c)\bm{P}_k(\bm{I}-\bm{K}_c\bm{H}_c)^\top + \bm{K}_c\bm{S}_c\bm{K}_c^\top
\end{align}

% ===== STABILIZATION =====
\subsection{Stabilization}
\begin{itemize}
\item Symmetry: $\bm{P} \leftarrow \frac{1}{2}(\bm{P}+\bm{P}^\top)$
\item PD enforcement: While $\bm{P} \not\succ 0$, add $\epsilon\bm{I}$
\item Numerical safety: $S^{-1}$ clamped at $10^{-12}$
\end{itemize}


\section{Schmitt Trigger-Based Correction}
To further constrain drift and improve estimation during wave oscillations, a Schmitt trigger-based logic is implemented. It detects zero-crossings of the vertical acceleration with hysteresis and debounce conditions.

Let \( a_k \) be the filtered acceleration and \( v_k \) the estimated velocity. A crossing event is detected when:
\[
a_k > \theta_{+}, \quad |v_k| > v_\text{thresh}, \quad \Delta t > t_\text{debounce}
\]
or
\[
a_k < \theta_{-}, \quad |v_k| > v_\text{thresh}, \quad \Delta t > t_\text{debounce}
\]

When a zero-crossing is detected, a soft correction is applied using an extended observation:
\[
\mathbf{H}_z =
\begin{bmatrix}
0 & 1 & 0 & 0 \\
0 & 0 & 1 & 0
\end{bmatrix}, \quad
\mathbf{z}_\text{corr} =
\begin{bmatrix}
(1 - \gamma)y_k \\
v_k + \gamma(\hat{v}_\text{target} - v_k)
\end{bmatrix}
\]
where \( \gamma \in [0,1] \) controls the correction strength and \( \hat{v}_\text{target} = \sqrt{v_k^2 + (\omega y_k)^2} \) based on estimated frequency.

\section{Implementation Notes}

This algorithm is implemented in C++ and uses the Eigen library. It is designed to run in embedded environments and includes:
\begin{itemize}
  \item Numerical conditioning of covariance matrices
  \item Mahalanobis distance gating
  \item Time-varying Schmitt trigger detection logic
  \item Optional theoretical process noise estimation from IMU specs
\end{itemize}

\section{Conclusion}
The proposed Kalman filter formulation provides robust heave estimation from vertical acceleration even in the presence of sensor drift and bias. The inclusion of pseudo-measurements and zero-crossing detection improves long-term stability and physical consistency.

\section*{Acknowledgements}
Implementation and theoretical design by Mikhail Grushinskiy, 2025.

\begin{thebibliography}{4}

\bibitem{Gerstner1809} 
F.~J. Gerstner, ``Theorie der Wellen,'' 
\emph{Annalen der Physik}, vol.~32, pp.~412–445, 1809.  
\emph{(English translation: }``Theory of Waves,'' \emph{Annual Reports of the Prague Polytechnic Institute}, 1847.)

\bibitem{Clamond2007} 
D.~Clamond and M.~Dutykh, ``Practical analytic approximation of trochoidal waves,'' 
\emph{Applied Ocean Research}, vol.~29, no.~4, pp.~213–220, 2007.

\bibitem{Sharkh2014}
Sharkh, S., Hendijanizadeh, M., Moshrefi-Torbati, M., \& Abusara, M. (2014, August). 
A novel Kalman filter based technique for calculating the time history of vertical displacement of a boat from measured acceleration. 
\textit{Marine Engineering Frontiers (MEF)}, \textit{2}.

\bibitem{Fenton1988}
J.~D.~Fenton, “The numerical solution of steady water wave problems,” \emph{Computers \& Geosciences}, vol.~14, no.~3, pp.~357–368, 1988.

\end{thebibliography}


\end{document}
