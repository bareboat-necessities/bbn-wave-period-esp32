\documentclass[11pt]{article}
\usepackage{amsmath, amssymb, graphicx, geometry}
\geometry{margin=1in}

\title{WaveSpectrumEstimator: An Embedded Algorithm for Ocean Wave Spectrum Estimation from Acceleration Measurements}
\author{Mikhail Grushinskiy}
\date{2025}

\begin{document}
\maketitle

\begin{abstract}
This article describes the methodology implemented in the \texttt{WaveSpectrumEstimator}, 
a C++ embedded-friendly class for estimating ocean wave spectra from vertical acceleration measurements.
The estimator combines signal preprocessing, decimation, Goertzel spectral analysis, and spectrum-to-wave conversion. 
It supports significant wave height estimation, peak frequency extraction, and parametric fitting to the Pierson--Moskowitz spectrum.
\end{abstract}

\section{Introduction}
Estimating ocean wave characteristics from inertial sensors is a key task in marine engineering, autonomous navigation, 
and environmental monitoring. Conventional methods often rely on FFT-based spectral estimators, which can be computationally expensive for embedded devices. 

The \texttt{WaveSpectrumEstimator} implements a lightweight, embedded-friendly approach based on a decimated Goertzel transform and spectral conversion techniques, optimized for real-time processing on microcontrollers.

\section{Signal Processing Pipeline}
The input to the estimator is a raw acceleration signal $x_{\text{raw}}(n)$, sampled at frequency $f_{s,\text{raw}}$. The processing pipeline consists of four stages:

\subsection{Low-pass Filtering and Decimation}
A biquad low-pass filter is designed with cutoff
\[
f_c = 0.45 \cdot \frac{f_{s,\text{raw}}}{2 \cdot \text{decimFactor}},
\]
and applied using the transposed direct form II realization. The output is downsampled by a factor $D$, yielding the effective sampling rate
\[
f_s = \frac{f_{s,\text{raw}}}{D}.
\]

\subsection{Windowing and Block Formation}
Samples are stored in a circular buffer of length $N_\text{block}$. Each block is multiplied by either a Hann window
\[
w(n) = \tfrac{1}{2}\left(1 - \cos\left(\tfrac{2\pi n}{N_\text{block}-1}\right)\right),
\]
or a rectangular window, depending on configuration. The mean is subtracted to remove DC bias.

\subsection{Spectral Estimation via Goertzel Algorithm}
Instead of a full FFT, the Goertzel recursion is applied for each target frequency $f_i$:
\begin{align}
s_n &= x_n w_n + c_i s_{n-1} - s_{n-2}, \\
c_i &= 2\cos\!\left(\frac{2\pi f_i}{f_s}\right).
\end{align}
At the end of the block, the DFT bin is reconstructed:
\[
X(f_i) = (s_{N-1} - s_{N-2}\cos\omega_i) - j (s_{N-2}\sin\omega_i).
\]

The single-sided power spectral density (PSD) of acceleration is then
\[
S_{aa}(f_i) = \frac{2}{f_s \sum w_n^2} \, |X(f_i)|^2.
\]

\subsection{Acceleration-to-Displacement Spectrum Conversion}
Using the deep-water linear wave relation, displacement spectrum $S_{\eta\eta}(f)$ is obtained by
\[
S_{\eta\eta}(f) = \frac{S_{aa}(f)}{\omega^4}, \qquad \omega = 2\pi f.
\]
A high-pass cutoff $f_\text{hp}$ (default $0.06$ Hz) suppresses spurious low-frequency energy.

\section{Spectral Parameters}
\subsection{Significant Wave Height}
The zeroth spectral moment is
\[
m_0 = \int_0^\infty S_{\eta\eta}(f)\, df,
\]
approximated numerically. The significant wave height is
\[
H_s = 4\sqrt{m_0}.
\]

\subsection{Peak Frequency Estimation}
The peak frequency $f_p$ is identified as the frequency maximizing $S_{\eta\eta}(f)$. A parabolic interpolation in $\log$-spectral space improves resolution.

\subsection{Pierson--Moskowitz Spectrum Fit}
The observed spectrum is compared against the Pierson--Moskowitz model
\[
S_{PM}(f;\alpha, f_p) = \alpha g^2 (2\pi f)^{-5} \exp\left[-\beta\left(\frac{f_p}{f}\right)^4\right],
\]
with $\beta = 0.74$. The parameters $\alpha$ and $f_p$ are estimated by minimizing the squared log-spectral error
\[
\text{cost}(\alpha,f_p) = \sum_i \left(\log S_{\eta\eta}(f_i) - \log S_{PM}(f_i;\alpha,f_p)\right)^2.
\]
A grid search followed by local refinement yields the optimal fit.

\section{Implementation Notes}
\begin{itemize}
\item Embedded-friendly: uses fixed-size arrays and Eigen matrices.
\item Frequency grid constructed with logarithmic spacing at low frequencies and linear spacing at higher frequencies.
\item Includes warm-up logic: spectrum becomes valid only after at least one full block is collected.
\end{itemize}

\section{Conclusion}
The \texttt{WaveSpectrumEstimator} provides a computationally efficient and embedded-ready implementation of wave spectral estimation from acceleration data. It offers not only raw displacement spectra, but also higher-level oceanographic parameters such as significant wave height, peak frequency, and Pierson--Moskowitz spectrum fits.
This makes it suitable for low-power autonomous platforms and real-time wave monitoring systems.

\end{document}
