\documentclass[11pt]{article}

% ---------- Preamble: professional but simple ----------
\usepackage[T1]{fontenc}
\usepackage[utf8]{inputenc}
\usepackage{lmodern}
\usepackage[margin=1in]{geometry}
\usepackage{microtype}
\usepackage{setspace}
\setstretch{1.05}

\usepackage{amsmath,amssymb,amsthm,mathtools}
\usepackage{bm}
\usepackage{siunitx}
\usepackage{booktabs,array}
\usepackage{graphicx}
\usepackage[hidelinks]{hyperref}
\usepackage{cleveref}

\newcommand{\vect}[1]{\bm{#1}}
\newcommand{\mat}[1]{\bm{#1}}
\newcommand{\quat}[1]{\mathbf{#1}}
\newcommand{\R}{\mathbb{R}}
\newcommand{\E}{\mathbb{E}}
\newcommand{\T}{\mathsf{T}}
\newcommand{\I}{\mat{I}}
% Renamed to avoid conflict with TeX primitive \skew
\newcommand{\skx}[1]{\left[ #1 \right]_\times}

\title{A Multiplicative EKF with Latent Ornstein--Uhlenbeck Acceleration\\
for Drift-Robust Wave Kinematics: Full Mathematical Specification,\\
Analytic Discretization, and Bias/PSD Safeguards}
\author{Mikhail Grushinskiy}
\date{2025}

\begin{document}
\maketitle

\begin{abstract}
We present a complete mathematical specification of a quaternion
right-multiplicative Extended Kalman Filter (MEKF) coupled to an extended
linear navigation chain for marine wave kinematics. The state comprises
attitude error, gyroscope bias, world-frame velocity and displacement, the
integral of displacement, a latent world acceleration driven by an
Ornstein--Uhlenbeck (OU) process, and accelerometer bias with optional
temperature dependence. The OU prior endows the latent acceleration with
stationary variance and finite correlation time, reducing low-frequency
drift growth in integrated position while preserving responsiveness to
swell dynamics. We derive (i) the continuous-time model; (ii) analytic
discrete transitions for attitude (Rodrigues small-angle form) and the
OU-driven linear chain; (iii) closed-form discrete process noise via OU
kernel double-integrals; (iv) the continuous-to-discrete Riccati mapping
for \(Q_d\) using the Van Loan method; (v) full measurement Jacobians under
right-multiplicative linearization; and (vi) numerical safeguards:
Joseph-form updates, explicit symmetry enforcement, positive semidefinite
(PSD) projection, and innovation conditioning. All expressions match the
algorithmic structure of the accompanying C++ implementation
(\textit{Kalman3D\_Wave} with analytic \texttt{PhiAxis4x1\_analytic} and
\texttt{QdAxis4x1\_analytic}, Kronecker OU coupling, and blockwise
covariance propagation). The document is self-contained with sequential
equation numbering and ready for archival compilation.
\end{abstract}

\section{State and Coordinates}
\label{sec:state}
The estimator runs in a world NED frame. The extended state is
\begin{equation}
  \vect{x} \;=\;
  \begin{bmatrix}
    \delta\vect{\theta} & \vect{b}_g & \vect{v} & \vect{p} & \vect{S} & \vect{a}_w & \vect{b}_a
  \end{bmatrix}^{\T}
  \in \R^{N_X}, \quad N_X = 21,
  \label{eq:state}
\end{equation}
where
\(\delta\vect{\theta}\in\R^3\) is the small attitude-error (right-multiplicative MEKF),
\(\vect{b}_g\in\R^3\) gyroscope bias,
\(\vect{v},\vect{p}\in\R^3\) world-frame velocity/position,
\(\vect{S}\in\R^3\) integral of displacement \( \dot{\vect{S}}=\vect{p} \),
\(\vect{a}_w\in\R^3\) latent world acceleration (OU),
and \(\vect{b}_a\in\R^3\) accelerometer bias (optionally temperature dependent).
Gravity in NED is \(\vect{g}_w = [0,0,g]^\T\) with \(g=\SI{9.80665}{m/s^2}\).

\section{Attitude Representation (Right-Multiplicative)}
\label{sec:attitude}
Let \(\quat{q}\) be the world\(\to\)body quaternion and
\(\delta\quat{q}(\delta\vect{\theta})\) the small-angle quaternion:
\begin{equation}
  \quat{q}^+ \;=\; \quat{q}\otimes \delta\quat{q}\!\left(\tfrac12\,\delta\vect{\theta}\right), 
  \qquad
  \delta\quat{q}(\vect{r}) \;=\;
  \begin{bmatrix}
    \cos(\|\vect{r}\|/2)\\
    \frac{\sin(\|\vect{r}\|/2)}{\|\vect{r}\|}\,\vect{r}
  \end{bmatrix},
  \label{eq:quat-update}
\end{equation}
with standard Maclaurin guards for \(\|\vect{r}\|\ll 1\).
The skew operator for \(\vect{x}=[x\,y\,z]^\T\) is
\begin{equation}
  \skx{\vect{x}} \;=\;
  \begin{bmatrix}
    0 & -z & y\\ z & 0 & -x\\ -y & x & 0
  \end{bmatrix}.
  \label{eq:skew}
\end{equation}

\section{Continuous-Time Process Model}
\label{sec:ct-model}
\subsection{Attitude and Bias}
Let \(\vect{\omega}_b\) be the measured body angular rate and
\(\vect{\omega} \coloneqq \vect{\omega}_b - \vect{b}_g\).
\begin{align}
  \dot{\quat{q}} &= \tfrac12\, \quat{q} \otimes \begin{bmatrix} 0 \\ \vect{\omega} \end{bmatrix},
  \label{eq:qdot}\\
  \dot{\delta\vect{\theta}} &= -\,\skx{\vect{\omega}}\,\delta\vect{\theta} - \delta\vect{b}_g + \vect{\nu}_g,
  \quad
  \dot{\vect{b}}_g \;=\; \vect{\xi}_g,
  \label{eq:theta-bg-ct}
\end{align}
with \(\vect{\nu}_g\sim\mathcal{N}(\vect{0},\,\Sigma_g)\) (gyro white noise in error eq.) and
\(\vect{\xi}_g\sim\mathcal{N}(\vect{0},\,q_{bg}\I_3)\) (bias random walk).

\subsection{Linear Kinematics and Latent OU Acceleration}
\begin{align}
  \dot{\vect{v}} &= \vect{a}_w, &
  \dot{\vect{p}} &= \vect{v}, &
  \dot{\vect{S}} &= \vect{p},
  \label{eq:lin-chain}\\
  \dot{\vect{a}}_w &= -\tfrac{1}{\tau}\,\vect{a}_w + \vect{w}_a, \quad
  \vect{w}_a \sim \mathcal{N}\!\bigl(\vect{0},\, \tfrac{2}{\tau}\,\Sigma_{aw}\bigr),
  \label{eq:ou-ct}\\
  \dot{\vect{b}}_a &= \vect{\xi}_{ba}, \quad \vect{\xi}_{ba} \sim \mathcal{N}(\vect{0},\, q_{ba}\I_3).
  \label{eq:ba-ct}
\end{align}
Here \(\tau>0\) is the OU correlation time and \(\Sigma_{aw}\) its stationary covariance
(allowing cross-axis correlation).

\section{Measurement Models and Jacobians}
\label{sec:meas}
Let \(\mat{R}_{wb}(\quat{q})\) be the world\(\to\)body rotation.

\subsection{Accelerometer}
\begin{align}
  \vect{f}_b &= \mat{R}_{wb}\bigl(\vect{a}_w - \vect{g}_w\bigr) + \vect{b}_a(T) + \vect{n}_a,
  \qquad \vect{n}_a \sim \mathcal{N}(\vect{0},\mat{R}_a),
  \label{eq:acc-meas}\\
  \vect{b}_a(T) &= \vect{b}_{a0} + \mat{K}_a \bigl(T - T_{\mathrm{ref}}\bigr).
  \label{eq:ba-temp}
\end{align}
Right-multiplicative attitude linearization (\Cref{app:att-jac}) gives
\begin{equation}
  \frac{\partial \vect{f}_b}{\partial \delta\vect{\theta}}
  \;=\; -\,\skx{\vect{f}_b^{\mathrm{pred}}}, \quad
  \frac{\partial \vect{f}_b}{\partial \vect{a}_w} \;=\; \mat{R}_{wb}, \quad
  \frac{\partial \vect{f}_b}{\partial \vect{b}_a} \;=\; \I_3.
  \label{eq:acc-jacs}
\end{equation}

\subsection{Magnetometer}
\begin{align}
  \vect{m}_b &= \mat{R}_{wb}\,\vect{B}_w + \vect{n}_m, \qquad
  \vect{n}_m \sim \mathcal{N}(\vect{0},\mat{R}_m),
  \label{eq:mag-meas}\\
  \frac{\partial \vect{m}_b}{\partial \delta\vect{\theta}}
  &= -\,\skx{\vect{m}_b^{\mathrm{pred}}}.
  \label{eq:mag-jac}
\end{align}

\subsection{Integral Pseudo-Measurement}
\begin{equation}
  \vect{z}_S \;=\; \vect{S} + \vect{n}_S, \qquad
  \vect{n}_S \sim \mathcal{N}(\vect{0},\mat{R}_S), \qquad
  \frac{\partial \vect{z}_S}{\partial \vect{S}} \;=\; \I_3.
  \label{eq:S-pseudo}
\end{equation}

\section{Discrete Time Update (Prediction)}
\label{sec:discrete-pred}

\subsection{Attitude Error/Bias Transition}
With step \(h>0\), the local attitude error transition and bias coupling are
\begin{equation}
  \Phi_{AA} \;=\;
  \begin{bmatrix}
    \I_3 - \skx{\vect{\omega}} h + \tfrac12 \skx{\vect{\omega}}^2 h^2 & -\I_3 h \\
    \mat{0} & \I_3
  \end{bmatrix},
  \label{eq:PhiAA}
\end{equation}
with standard small-angle guards or Rodrigues form for larger angles.

\subsection{OU-Driven Linear Chain (Per-Axis Analytic)}
For \(x \coloneqq h/\tau\), \(\alpha \coloneqq e^{-x}\). The one-axis chain
\(\vect{x}_L=[v,p,S,a]^\T\) has exact discrete transition
\begin{equation}
  \Phi_{\mathrm{axis}}(h,\tau) \;=\;
  \begin{bmatrix}
    1 & 0 & 0 & \phi_{va} \\
    h & 1 & 0 & \phi_{pa} \\
    \tfrac12 h^2 & h & 1 & \phi_{Sa} \\
    0 & 0 & 0 & \alpha
  \end{bmatrix},
  \quad
  \begin{aligned}
    \phi_{va} &\coloneqq \tau (1-\alpha),\\
    \phi_{pa} &\coloneqq \tau^2 \bigl(x + \alpha - 1\bigr),\\
    \phi_{Sa} &\coloneqq \tau^3 \!\left(\tfrac12 x^2 - x - (\alpha-1)\right).
  \end{aligned}
  \label{eq:Phi-axis}
\end{equation}
High-order Maclaurin for \(x\ll 1\) avoids cancellation:
\begin{equation}
  \phi_{va} = h - \tfrac12 \tfrac{h^2}{\tau} + \tfrac16 \tfrac{h^3}{\tau^2} - \tfrac1{24} \tfrac{h^4}{\tau^3} + \cdots,
  \quad
  \phi_{pa} = \tfrac12 h^2 - \tfrac16 \tfrac{h^3}{\tau} + \tfrac1{24} \tfrac{h^4}{\tau^2} - \cdots,
  \quad
  \phi_{Sa} = \tfrac16 h^3 - \tfrac1{24} \tfrac{h^4}{\tau} + \tfrac1{120} \tfrac{h^5}{\tau^2} - \cdots.
  \label{eq:Phi-axis-series}
\end{equation}

\subsection{Three-Axis Assembly and Complete \texorpdfstring{$\Phi$}{Phi}}
Let \(\Phi_{\mathrm{axis}}\) be as in \eqref{eq:Phi-axis}. Define the block
for the 3D linear subsystem in \([v,p,S,a_w]\) order per axis and gather via
Kronecker:
\begin{equation}
  \Phi_{LL}^{\mathrm{raw}} \;=\; \I_3 \otimes \Phi_{\mathrm{axis}}(h,\tau).
  \label{eq:PhiLL-raw}
\end{equation}
If the global state stacks \([\,\vect{v}~\vect{p}~\vect{S}~\vect{a}_w\,]\) by components,
apply a fixed permutation \(\mat{P}\) to map axis-interleaved blocks to the chosen
state ordering:
\begin{equation}
  \Phi_{LL} \;=\; \mat{P}\, \Phi_{LL}^{\mathrm{raw}} \, \mat{P}^{\T}.
  \label{eq:PhiLL}
\end{equation}
Accelerometer-bias is a random walk, so
\begin{equation}
  \Phi_{BB} \;=\; \I_3.
  \label{eq:PhiBB}
\end{equation}
Collecting blocks in the order \([\delta\vect{\theta},\vect{b}_g]\), \(\vect{b}_a\), and
\([\vect{v},\vect{p},\vect{S},\vect{a}_w]\), the full discrete transition is
\begin{equation}
  \Phi \;=\;
  \begin{bmatrix}
    \Phi_{AA} & \mat{0} & \mat{0} \\
    \mat{0}   & \Phi_{BB} & \mat{0} \\
    \mat{0}   & \mat{0}   & \Phi_{LL}
  \end{bmatrix}.
  \label{eq:Phi-full}
\end{equation}
(If additional small cross-terms are retained in implementation, they enter the
off-diagonal blocks; the C++ baseline uses the structure above, with cross-covariance
propagation carried by \(P\), see \Cref{sec:cov-prop}.)

\section{Discrete Process Noise \texorpdfstring{$Q_d$}{Qd}}
\label{sec:Qd}

\subsection{OU Kernel Double-Integral (Per Axis)}
For the one-axis chain \( [v,p,S,a]^\T \) driven by OU with stationary covariance
\(\sigma_a^2\), the discrete covariance is
\begin{equation}
  \mat{Q}_d^{(1)}(h,\tau,\sigma_a^2) \;=\; \frac{2\sigma_a^2}{\tau}
  \int_0^h\!\!\int_0^h e^{-|t-s|/\tau}\,\vect{f}(t)\,\vect{f}(s)^{\T}\,ds\,dt,
  \quad
  \vect{f}(t) \coloneqq \begin{bmatrix} 1 & t & \tfrac12 t^2 & \alpha(t) \end{bmatrix}^{\T},
  \label{eq:Qd-axis-integral}
\end{equation}
with \(\alpha(t)=e^{-t/\tau}\). By symmetry,
\(\int_0^h\!\!\int_0^h e^{-|t-s|/\tau}\cdot = 2\int_0^h \int_0^t e^{-(t-s)/\tau}\cdot \,ds\,dt\),
reducing all entries to combinations of polynomials times exponentials.
Closed forms are provided in \Cref{app:ou-kernel}. For numerical stability define
\(x\coloneqq h/\tau\), \(\alpha\coloneqq e^{-x}\), \(\text{em1}\coloneqq \alpha-1\),
\(\alpha_2\coloneqq e^{-2x}\).
Introduce helpers
\begin{equation}
  A_0 \coloneqq \tau(1-\alpha),\qquad
  B_0 \coloneqq \tfrac{\tau}{2}\,(1-\alpha_2),
  \label{eq:A0B0}
\end{equation}
and small-\(x\) series in \Cref{app:series-guards}.

\subsection{Three-Axis Correlated OU via Kronecker}
Let \(\Sigma_{aw}\in\R^{3\times 3}\) be the stationary covariance (possibly anisotropic,
correlated). Then
\begin{equation}
  \mat{Q}_{LL}^{\mathrm{raw}}
  \;=\; \Sigma_{aw} \otimes \mat{Q}_d^{(1)}(h,\tau,1),
  \qquad
  \mat{Q}_{LL} \;=\; \mat{P}\,\mat{Q}_{LL}^{\mathrm{raw}}\,\mat{P}^{\T}.
  \label{eq:Qll-kron}
\end{equation}
Finally, bias and attitude/bias driving spectra yield
\begin{equation}
  \mat{Q}_{AA} \;=\; \mathrm{diag}\!\bigl(\Sigma_g h,\; q_{bg} h\,\I_3\bigr),
  \qquad
  \mat{Q}_{BB} \;=\; q_{ba} h\,\I_3.
  \label{eq:QaaQbb}
\end{equation}
Assemble the global
\begin{equation}
  \mat{Q}_d \;=\; \mathrm{diag}\!\bigl(\mat{Q}_{AA},\; \mat{Q}_{BB},\; \mat{Q}_{LL}\bigr),
  \quad
  \mat{Q}_d \gets \tfrac12\bigl(\mat{Q}_d+\mat{Q}_d^{\T}\bigr),
  \label{eq:Qd-full}
\end{equation}
with PSD floor if needed (\Cref{sec:numerics}).

\subsection{Continuous Riccati \texorpdfstring{$\to$}{} Discrete \texorpdfstring{$Q_d$}{} (Van Loan)}
\label{sec:vanloan}
For \(\dot{\vect{x}}=\mat{A}\vect{x}+\mat{G}\vect{w}\), \(\E[\vect{w}\vect{w}^{\T}]=\mat{Q}_c\delta\),
the exact discretization satisfies
\begin{equation}
  \Phi \;=\; e^{\mat{A}h}, \qquad
  \mat{Q}_d \;=\; \int_0^h e^{\mat{A}\tau}\,\mat{G}\mat{Q}_c \mat{G}^{\T}\, e^{\mat{A}^{\T}\tau}\, d\tau.
  \label{eq:Qd-def}
\end{equation}
Define the Van Loan companion
\begin{equation}
  \mat{M} \;=\;
  \begin{bmatrix}
    -\mat{A} & \mat{G}\mat{Q}_c\mat{G}^{\T} \\
    \mat{0}  & \mat{A}^{\T}
  \end{bmatrix},\qquad
  e^{\mat{M}h} \;=\;
  \begin{bmatrix}
    \mat{M}_{11} & \mat{M}_{12}\\
    \mat{0}      & \mat{M}_{22}
  \end{bmatrix}.
  \label{eq:van-loan}
\end{equation}
Then \( \Phi = \mat{M}_{22}^{\T} \) and \( \mat{Q}_d = \mat{M}_{22}^{\T}\mat{M}_{12} \)
(see \Cref{app:van-loan-derivation}). For the OU chain we supply closed forms
(\Cref{app:ou-kernel}), which the implementation uses via
\texttt{PhiAxis4x1\_analytic} and \texttt{QdAxis4x1\_analytic}.

\section{Complete Process Jacobian \texorpdfstring{$\mat{A}=\partial f/\partial x$}{A} and \texorpdfstring{$\mat{G}$}{G}}
\label{sec:AG}
Ordering \(\vect{x}\) as in \eqref{eq:state}, the continuous Jacobian is
\begin{equation}
\small
\mat{A} \;=\;
\begin{bmatrix}
 -\skx{\vect{\omega}} & -\I_3 & \mat{0} & \mat{0} & \mat{0} & \mat{0} & \mat{0} \\
 \mat{0}               & \mat{0} & \mat{0} & \mat{0} & \mat{0} & \mat{0} & \mat{0} \\
 \mat{0} & \mat{0} & \mat{0} & \I_3 & \mat{0} & \I_3 & \mat{0} \\
 \mat{0} & \mat{0} & \mat{0} & \mat{0} & \I_3 & \mat{0} & \mat{0} \\
 \mat{0} & \mat{0} & \mat{0} & \mat{0} & \mat{0} & \mat{0} & \mat{0} \\
 \mat{0} & \mat{0} & \mat{0} & \mat{0} & \mat{0} & -\tfrac{1}{\tau}\I_3 & \mat{0} \\
 \mat{0} & \mat{0} & \mat{0} & \mat{0} & \mat{0} & \mat{0} & \mat{0}
\end{bmatrix}.
\label{eq:A-ct}
\end{equation}
The noise input matrix \(\mat{G}\) (shaping white drivers into \(\delta\dot{\vect{\theta}},\dot{\vect{b}}_g,\dot{\vect{a}}_w,\dot{\vect{b}}_a\)) is
\begin{equation}
\small
\mat{G} \;=\; \mathrm{blkdiag}\!\left(
\begin{bmatrix}\I_3\\ \mat{0}\end{bmatrix},\;
\I_3,\;
\begin{bmatrix}\mat{0}\\ \mat{0}\\ \mat{0}\\ \I_3\end{bmatrix},\;
\I_3
\right),
\quad
\mat{Q}_c \;=\; \mathrm{blkdiag}\!\bigl(\Sigma_g,\; q_{bg}\I_3,\; \tfrac{2}{\tau}\Sigma_{aw},\; q_{ba}\I_3\bigr).
\label{eq:GQc}
\end{equation}
These \(\mat{A},\mat{G},\mat{Q}_c\) reproduce the block \(Q_d\) of \Cref{sec:Qd} via
either Van Loan or the supplied OU closed forms.

\section{Measurement Update (Unified Form)}
\label{sec:meas-update}
For measurement \(\vect{z}=h(\vect{x})+\vect{n}\) with Jacobian \(\mat{C}\) and noise \(\mat{R}\):
\begin{align}
 \vect{r} &= \vect{z} - \vect{h}(\hat{\vect{x}}^-), \label{eq:innov}\\
 \mat{S}  &= \mat{C}\mat{P}^- \mat{C}^{\T} + \mat{R}, \label{eq:S}\\
 \mat{K}  &= \mat{P}^- \mat{C}^{\T} \mat{S}^{-1}, \label{eq:K}\\
 \hat{\vect{x}}^+ &= \hat{\vect{x}}^- + \mat{K}\,\vect{r}, \label{eq:x-update}\\
 \mat{P}^+ &= (\I - \mat{K}\mat{C})\mat{P}^- (\I - \mat{K}\mat{C})^{\T} + \mat{K}\mat{R}\mat{K}^{\T} \quad \text{(Joseph form)}.
 \label{eq:joseph}
\end{align}
After each update, apply quaternion correction and error reset per \eqref{eq:quat-update}.

\section{Covariance Propagation and Coupling}
\label{sec:cov-prop}
Prediction:
\begin{equation}
  \hat{\vect{x}}^- \;=\; \Phi\,\hat{\vect{x}}^+,
  \qquad
  \mat{P}^- \;=\; \Phi \mat{P}^+ \Phi^{\T} + \mat{Q}_d.
  \label{eq:pred}
\end{equation}
Partitioning \( \mat{P} \) conformably with \([\delta\vect{\theta},\vect{b}_g]\),
\(\vect{b}_a\), and \([\vect{v},\vect{p},\vect{S},\vect{a}_w]\), the block prediction is
\begin{align}
  P_{AA}^- &= \Phi_{AA} P_{AA}^+ \Phi_{AA}^{\T} + Q_{AA}, \\
  P_{LL}^- &= \Phi_{LL} P_{LL}^+ \Phi_{LL}^{\T} + Q_{LL}, \\
  P_{BB}^- &= P_{BB}^+ + Q_{BB}, \\
  P_{AL}^- &= \Phi_{AA} P_{AL}^+ \Phi_{LL}^{\T}, \quad
  P_{AB}^- \;=\; \Phi_{AA} P_{AB}^+, \quad
  P_{LB}^- \;=\; \Phi_{LL} P_{LB}^+.
  \label{eq:block-prop}
\end{align}
Cross-covariances maintain the physical projection of attitude uncertainty into
linear kinematics; the implementation preserves these blocks.

\section{Measurement Noise Structure}
\label{sec:meas-noise}
\begin{equation}
  \mat{R}_a = \mathrm{diag}(\sigma_{ax}^2,\sigma_{ay}^2,\sigma_{az}^2),\quad
  \mat{R}_m = \mathrm{diag}(\sigma_{mx}^2,\sigma_{my}^2,\sigma_{mz}^2),\quad
  \mat{R}_S = \mathrm{diag}(\sigma_{Sx}^2,\sigma_{Sy}^2,\sigma_{Sz}^2).
  \label{eq:R-blocks}
\end{equation}
Tuning balances short-term attitude correction (\(\mat{R}_a\)), heading anchoring (\(\mat{R}_m\)),
and long-term drift regularization (\(\mat{R}_S\)).

\section{Numerical Safeguards}
\label{sec:numerics}
\textbf{Symmetry and PSD:}
\begin{equation}
  \mat{P} \leftarrow \tfrac12(\mat{P}+\mat{P}^{\T}),\qquad
  \mat{P} \succeq 0 \text{ via eigenvalue floor } \lambda_i \leftarrow \max(\lambda_i,\varepsilon),
  \label{eq:psd}
\end{equation}
with \(\varepsilon\approx 10^{-12}\) for \(4\times4\) attitude block, \(10^{-16}\) for \(12\times12\) linear block.
\textbf{Innovation conditioning:} if \(\mat{S}\) is near-singular, add a small diagonal
\(\delta \I\) before inversion (LDLT solve), \(\delta \approx 10^{-6}(\|\mat{S}\|+1)\).
\textbf{Gating:} accept accelerometer when \(|\|\vect{a}_{\mathrm{meas}}\|-g|<2g\);
accept magnetometer when \(\|\vect{m}_{\mathrm{meas}}\|,\|\vect{m}_{\mathrm{pred}}\|>10^{-6}\).

\section{Initialization and Defaults}
\label{sec:init}
\textbf{Attitude from accel+mag:} form world triad from gravity and horizontal magnetic
component, then set \(\mat{R}_{wb}\) and \(\quat{q}\). Reference field vector \(\vect{B}_w\) from
declination/inclination if needed.\\
\textbf{Initial covariance (typical):}
\begin{align}
  P_{\theta\theta} &= 10^{-6}\I_3, \quad
  P_{b_g b_g} = 10^{-1}\I_3, \quad
  P_{vv} = 1^2\I_3, \quad
  P_{pp} = 20^2\I_3, \\
  P_{SS} &= 50^2\I_3, \quad
  P_{aa} = \Sigma_{aw}, \quad
  P_{b_a b_a} = 0.1^2\I_3.
  \label{eq:P0}
\end{align}
\textbf{Process noise (typical):}
\begin{equation}
  \Sigma_g = \sigma_g^2 \I_3,\quad q_{bg}\approx 10^{-12},\quad
  q_{ba}\approx 10^{-8},\quad
  \tau \approx \SI{2.3}{s},\quad
  \Sigma_{aw} \approx (2.4^2)\I_3.
  \label{eq:Q0}
\end{equation}
\textbf{Measurement noise (typical):}
\begin{equation}
  \mat{R}_a = \mathrm{diag}(\sigma_a^2),\quad
  \mat{R}_m = \mathrm{diag}(\sigma_m^2),\quad
  \mat{R}_S = 1.5^2 \I_3.
  \label{eq:R0}
\end{equation}

% ===================== Appendices =====================
\appendix

\section{Right-Multiplicative Attitude Jacobian}
\label{app:att-jac}
For \(h(\quat{q})=\mat{R}_{wb}(\quat{q})\vect{v}_w\),
\( \quat{q}^+=\quat{q}\otimes \delta\quat{q}(\tfrac12\delta\vect{\theta}) \),
\( \mat{R}(\delta\vect{\theta})\approx \I - \skx{\delta\vect{\theta}} \).
Then
\begin{equation}
  h(\quat{q}^+) \approx h(\quat{q}) - \mat{R}_{wb}\,\skx{\delta\vect{\theta}}\,\vect{v}_w
  \;=\; h(\quat{q}) + \bigl(-\skx{\mat{R}_{wb}\vect{v}_w}\bigr)\delta\vect{\theta},
  \quad
  \frac{\partial h}{\partial \delta\vect{\theta}} = -\,\skx{\mat{R}_{wb}\vect{v}_w}.
  \label{eq:att-jac-appendix}
\end{equation}
With \(\vect{v}_w = \vect{a}_w-\vect{g}_w\) and \(\vect{v}_w=\vect{B}_w\) one obtains
\eqref{eq:acc-jacs} and \eqref{eq:mag-jac}.

\section{Derivation of OU Integration Coefficients}
\label{app:ou-kernel}
We evaluate \eqref{eq:Qd-axis-integral} using
\(
\int_0^h\!\!\int_0^h e^{-|t-s|/\tau} p(t) q(s)\,ds\,dt
= 2\int_0^h \int_0^t e^{-(t-s)/\tau} p(t) q(s)\,ds\,dt
\)
for polynomials \(p,q\) and \(e^{-t/\tau}\).
Define \(x=h/\tau\), \(\alpha=e^{-x}\).
Representative entries (per-axis, unit \(\sigma_a^2=1\)):
\begin{align}
  [\mat{Q}_d^{(1)}]_{vv} &= \tfrac{2}{\tau}\!\left(
     \tfrac{h^3}{3} - \tfrac{h^4}{4\tau} + \tfrac{h^5}{10\tau^2}
     - \tfrac{h^3}{\tau^2}A_0 + \tfrac{1}{\tau^2}B_0
  \right), \label{eq:Qvv}\\
  [\mat{Q}_d^{(1)}]_{va} &= \tfrac{2}{\tau}\!\left(
     \tfrac{h^2}{2} - \tfrac{h^3}{3\tau} + \tfrac{h^4}{8\tau^2}
     - A_0 + \tfrac{B_0}{\tau}
  \right), \label{eq:Qva}\\
  [\mat{Q}_d^{(1)}]_{aa} &= \tfrac{2}{\tau}\, B_0,
  \qquad
  A_0=\tau(1-\alpha),\; B_0=\tfrac{\tau}{2}(1-\alpha^2),
  \label{eq:Qaa}
\end{align}
with analogous forms for \([pp], [SS], [pv], [pa], [Sa], [Sp], [Sv]\).
All coefficients arise from integrals of \(t^i s^j e^{-(t-s)/\tau}\) on the
simplex \(0\le s\le t\le h\); the implementation uses closed forms bundled
in \texttt{QdAxis4x1\_analytic} with small-\(x\) guards.

\section{Small-\texorpdfstring{$x=h/\tau$}{x} Series for Stability}
\label{app:series-guards}
For \(x\ll 1\), use Maclaurin expansions to avoid cancellation:
\begin{align}
  \alpha &= 1 - x + \tfrac{x^2}{2} - \tfrac{x^3}{6} + \tfrac{x^4}{24} - \cdots, \\
  A_0 &= \tau\!\left(x - \tfrac{x^2}{2} + \tfrac{x^3}{6} - \tfrac{x^4}{24} + \cdots\right),\quad
  B_0 = \tau\!\left(x - x^2 + \tfrac{2}{3}x^3 - \tfrac{1}{3}x^4 + \cdots\right).
  \label{eq:series}
\end{align}
Substitute into \eqref{eq:Phi-axis}--\eqref{eq:Qaa} as needed.

\section{Continuous\texorpdfstring{$\to$}{}Discrete Riccati Mapping (Van Loan)}
\label{app:van-loan-derivation}
From \eqref{eq:Qd-def}, stack the coupled ODEs
\(
\frac{d}{d\tau}\bigl[e^{-\mat{A}\tau}\mat{P}(\tau)e^{-\mat{A}^{\T}\tau}\bigr]
= e^{-\mat{A}\tau}\mat{G}\mat{Q}_c\mat{G}^{\T} e^{-\mat{A}^{\T}\tau}
\),
integrate on \([0,h]\), and use block-matrix exponentials of \(\mat{M}\) in \eqref{eq:van-loan}.
The identity \(e^{\mat{M}h}=\begin{bmatrix}\mat{M}_{11}&\mat{M}_{12}\\\mat{0}&\mat{M}_{22}\end{bmatrix}\)
implies \(\Phi=\mat{M}_{22}^{\T}\) and \(\mat{Q}_d=\Phi\,\mat{M}_{12}\).
For the OU chain, these reproduce the closed forms in \Cref{app:ou-kernel}.

\section{Full Measurement Jacobian Assembly}
\label{app:full-C}
Embed sensor blocks into \(m\times 21\) dense \(\mat{C}\):
\begin{align}
  \mat{C}_a &= \begin{bmatrix}
     -\skx{\vect{f}_b^{\mathrm{pred}}} & \mat{0} & \mat{0} & \mat{0} & \mat{0} & \mat{R}_{wb} & \I_3
  \end{bmatrix}, \label{eq:C-acc-full}\\
  \mat{C}_m &= \begin{bmatrix}
     -\skx{\vect{m}_b^{\mathrm{pred}}} & \mat{0} & \mat{0} & \mat{0} & \mat{0} & \mat{0} & \mat{0}
  \end{bmatrix}, \label{eq:C-mag-full}\\
  \mat{C}_S &= \begin{bmatrix}
     \mat{0} & \mat{0} & \mat{0} & \mat{0} & \I_3 & \mat{0} & \mat{0}
  \end{bmatrix}. \label{eq:C-S-full}
\end{align}
These match the C++ full-matrix assembly used in the Kalman gain and Joseph update.

\section*{Acknowledgments}
This document mirrors the implementation of \texttt{Kalman3D\_Wave} (vOct22 baseline) and
the analytic primitives \texttt{PhiAxis4x1\_analytic} and \texttt{QdAxis4x1\_analytic}, with
blockwise covariance propagation, OU Kronecker coupling, right-multiplicative attitude
linearization, Joseph updates, PSD projection, and temperature-dependent accelerometer bias.
All formulas are aligned one-to-one with the code paths used in the fusion stack.

\bibliographystyle{plain}
\begin{thebibliography}{9}
\bibitem{jazwinski}
A.~H. Jazwinski, \emph{Stochastic Processes and Filtering Theory}. Academic Press, 1970.
\bibitem{shuster}
M.~D. Shuster, ``A survey of attitude representations,'' \emph{J. Astronautical Sciences}, 1983.
\bibitem{vanloan}
C.~F. Van Loan, ``Computing Integrals Involving the Matrix Exponential,'' \emph{IEEE TAC}, 1978.
\end{thebibliography}

\end{document}
