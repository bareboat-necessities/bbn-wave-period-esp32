\documentclass[10pt,twocolumn]{article}

% ===================== Packages =====================
\usepackage[T1]{fontenc}
\usepackage{lmodern}
\usepackage{microtype}
\usepackage{mathtools,amssymb,amsfonts}
\usepackage{bm}
\usepackage{siunitx}
\usepackage{geometry}
\usepackage[hidelinks]{hyperref}
\usepackage{booktabs}
\usepackage{enumitem}
\usepackage{algorithm}
\usepackage{algpseudocode}

\geometry{margin=0.68in}
\setlength{\columnsep}{0.22in}
\setlist[itemize]{leftmargin=*,itemsep=2pt,topsep=2pt}
\setlist[enumerate]{leftmargin=*,itemsep=2pt,topsep=2pt}

% ===================== Macros =====================
\newcommand{\R}{\mathbb{R}}
\newcommand{\SO}{\mathrm{SO}}
\newcommand{\diag}{\mathrm{diag}}
\newcommand{\E}{\mathbb{E}}
\newcommand{\I}{\mathbf{I}}
\newcommand{\0}{\mathbf{0}}

% IMPORTANT: do NOT name this \skew (TeX primitive). Use \xskew instead.
\newcommand{\xskew}[1]{\left[#1\right]_{\times}}

\newcommand{\dth}{\delta\bm{\theta}}
\newcommand{\bg}{\mathbf{b}_g}
\newcommand{\ba}{\mathbf{b}_a}
\newcommand{\aw}{\mathbf{a}_w}
\newcommand{\gvec}{\mathbf{g}}
\newcommand{\RwB}{\mathbf{R}_{wb}}
\newcommand{\RbW}{\mathbf{R}_{bw}}
\newcommand{\eThree}{\mathbf{e}_3}

\newcommand{\pvec}{\mathbf{p}}
\newcommand{\vvec}{\mathbf{v}}
\newcommand{\w}{\bm{\omega}}
\newcommand{\accm}{\mathbf{a}_m}
\newcommand{\magm}{\mathbf{m}_m}

\newcommand{\Ts}{\Delta t}
\newcommand{\eps}{\varepsilon}
\newcommand{\norm}[1]{\left\lVert #1 \right\rVert}

% Avoid double subscripts: make time-indexed qref macros directly.
\newcommand{\qrefk}{\mathbf{q}_{\mathrm{ref},k}}
\newcommand{\qrefkp}{\mathbf{q}_{\mathrm{ref},k+1}}

% Avoid double subscripts on b_a:
\newcommand{\baterm}{\mathbf{b}_{a,\mathrm{term}}}

% ===================== Title =====================
\title{\vspace{-0.35in}\bfseries
Kalman3D\_Wave\_2: Broadband Oscillator Sea-State + Quaternion Error-State Attitude\\
Full Math, Discretization, and Implementation-Traceable Algorithms (Compiles in CI)
}
\author{Mikhail Grushinskiy}
\date{\vspace{-0.30in}}

\begin{document}
\maketitle
\vspace{-0.25in}

\begin{abstract}
This document describes the math and implementation of a marine IMU fusion filter that combines
(i) quaternion error-state attitude (qMEKF style, right-multiply corrections),
(ii) a broadband sea-motion model as a bank of damped oscillators producing wave displacement/velocity/acceleration in WORLD,
and (iii) robust tilt-compensated magnetometer yaw correction using a direction-only residual with 180\textdegree\ ambiguity handling.
It includes process models, closed-form oscillator transitions (under/critical/over-damped),
Simpson-rule noise integration, measurement Jacobians, NIS gating, Joseph covariance update,
warmup/bias gating policies, and an optional axis-independence stabilization.
\end{abstract}

% ==============================================================================
\section{Frames and Conventions}
\paragraph{Frames.}
WORLD is typically NED ($+Z$ down). BODY' is an un-heeled virtual body frame (heel may be applied externally; the implementation may also de-heel IMU vectors internally).

\paragraph{Stored quaternion.}
The filter stores a reference quaternion mapping WORLD $\rightarrow$ BODY'. Its DCM is
\[
\RwB=\mathbf{R}(\mathbf{q}_{\mathrm{ref}})\in\SO(3),\qquad \RbW=\RwB^\top.
\]

\paragraph{Error state and correction (right-multiply).}
The small attitude error is $\dth\in\R^3$. The correction convention is:
\[
\mathbf{q}_{\mathrm{ref}} \leftarrow \mathbf{q}_{\mathrm{ref}} \otimes \delta q,\qquad
\delta q = \exp\!\left(\frac{1}{2}\dth\right),\qquad
\dth \leftarrow \0.
\]
All Jacobians below assume this convention.

\paragraph{Per-sample update order.}
Typical order per IMU sample:
(1) time update, (2) accelerometer update, (3) magnetometer update (delayed), then stability steps (symmetrize/clamp/PSD/optional axis-independence).

% ==============================================================================
\section{State Vector and Dimension}
Let $K$ be the number of oscillator modes (\texttt{KMODES}). For each mode $k\in\{1,\dots,K\}$, the wave block stores $\pvec_k,\vvec_k\in\R^3$ in WORLD.

\subsection{State definition}
\[
\mathbf{x}=
\begin{bmatrix}
\dth \\
(\bg) \\
\pvec_1\\ \vvec_1\\
\vdots\\
\pvec_K\\ \vvec_K\\
(\ba)
\end{bmatrix}.
\]

\subsection{Dimension}
Let $g\in\{0,1\}$ indicate gyro bias enabled and $a\in\{0,1\}$ indicate accel bias enabled. Then
\[
N_X = (3 + 3g) + 6K + 3a.
\]
With defaults $K=3$, $g=1$, $a=1$, the state dimension is $27$.

\subsection{Implementation ordering (critical)}
Within each mode $k$, the implementation stores
\[
\mathbf{x}_k=[p_{k,x},p_{k,y},p_{k,z},\,v_{k,x},v_{k,y},v_{k,z}]^\top,
\]
i.e., \textbf{p(3) then v(3)}. Oscillator discretization blocks must be assembled to match this ordering.

% ==============================================================================
\section{Continuous-Time Process Models}

\subsection{Attitude error-state}
Let measured angular rate be $\w_m$ in BODY' and the estimated gyro bias be $\bg$. The bias-corrected rate is $\w=\w_m-\bg$.
A small-error model consistent with right-multiply correction is:
\[
\dot{\dth} = -\xskew{\w}\,\dth\ -\ \delta\bg\ -\ \mathbf{n}_g,\qquad
\dot{\delta\bg} = \mathbf{n}_{bg}.
\]
(If gyro bias is disabled, drop the bias terms.)

\subsection{Broadband wave oscillator bank (per axis)}
For each mode $k$ and each axis independently:
\[
\dot p_k = v_k,\qquad
\dot v_k = -\omega_k^2 p_k - 2\zeta_k\omega_k v_k + \xi_k(t),
\]
where $\xi_k(t)$ is white noise driving $v'$ with intensity $q_k$:
\[
\E[\xi_k(t)\xi_k(\tau)] = q_k\,\delta(t-\tau),\qquad q_k\ [\si{m^2/s^5}].
\]

\subsection{Wave acceleration in WORLD}
Wave acceleration reconstructed from the oscillator bank is
\[
\aw = \sum_{k=1}^{K}\left(-\omega_k^2\pvec_k - 2\zeta_k\omega_k\vvec_k\right).
\]

\subsection{Accelerometer bias random walk (optional)}
If accel bias is enabled and updates are allowed:
\[
\dot{\delta\ba}=\mathbf{n}_{ba}.
\]

% ==============================================================================
\section{Discrete-Time Propagation}

\subsection{Quaternion propagation}
With $\w=\w_m-\bg$,
\[
\qrefkp = \qrefk \otimes \exp\!\left(\frac{1}{2}\w\,\Ts\right).
\]

\subsection{Base block discretization and ``exact-ish'' noise integration}
Define $\Omega=\xskew{\w}$ and
\[
\mathbf{R}(t)=\exp(-\Omega t),\qquad
\mathbf{B}(t) = -\int_0^t \mathbf{R}(\tau)\,d\tau.
\]
Let $\mathbf{Q}_g$ be gyro noise PSD and $\mathbf{Q}_{bg}$ gyro-bias RW PSD (both diagonal in the implementation). Then
\[
\mathbf{Q}_{\theta\theta} =
\int_0^{\Ts}\mathbf{R}(t)\mathbf{Q}_g\mathbf{R}(t)^\top dt
+\int_0^{\Ts}\mathbf{B}(t)\mathbf{Q}_{bg}\mathbf{B}(t)^\top dt,
\]
\[
\mathbf{Q}_{bb} = \mathbf{Q}_{bg}\Ts,\qquad
\mathbf{Q}_{\theta b} = \Big(\int_0^{\Ts}\mathbf{B}(t)\,dt\Big)\mathbf{Q}_{bg}.
\]
Simpson rule is used:
\[
\int_0^{\Ts} f(t)\,dt \approx \frac{\Ts}{6}\bigl(f(0)+4f(\Ts/2)+f(\Ts)\bigr).
\]

\subsection{Oscillator transition Phi (closed form)}
For one axis (drop subscripts), define $a=\zeta\omega$.

\paragraph{Critical damping ($\zeta=1$).}
\[
\Phi(t)=e^{-a t}
\begin{bmatrix}
1+a t & t\\
-\omega^2 t & 1-a t
\end{bmatrix}.
\]

\paragraph{Underdamped ($\zeta<1$).}
Let $\omega_d=\omega\sqrt{1-\zeta^2}$:
\[
\Phi(t)=e^{-a t}
\begin{bmatrix}
\cos(\omega_d t) + \frac{a}{\omega_d}\sin(\omega_d t) & \frac{1}{\omega_d}\sin(\omega_d t)\\
-\frac{\omega^2}{\omega_d}\sin(\omega_d t) & \cos(\omega_d t) - \frac{a}{\omega_d}\sin(\omega_d t)
\end{bmatrix}.
\]

\paragraph{Overdamped ($\zeta>1$).}
Let $s=\sqrt{\zeta^2-1}$ and
\[
r_1=-\omega(\zeta-s),\qquad r_2=-\omega(\zeta+s).
\]
One equivalent form is:
\[
\Phi(t)=\frac{1}{r_2-r_1}
\begin{bmatrix}
r_2 e^{r_1 t} - r_1 e^{r_2 t} & e^{r_2 t}-e^{r_1 t}\\
r_1 r_2 (e^{r_1 t}-e^{r_2 t}) & r_2 e^{r_2 t}-r_1 e^{r_1 t}
\end{bmatrix}.
\]

\subsection{Oscillator Qd by Simpson integration}
For $\dot{\mathbf{s}}=\mathbf{A}\mathbf{s}+\mathbf{G}\xi$ with $\mathbf{G}=[0,1]^\top$:
\[
\mathbf{Q}_d = q\int_0^{\Ts} \Phi(t)\mathbf{G}\mathbf{G}^\top\Phi(t)^\top dt.
\]
Let $\mathbf{u}(t)=\Phi(t)\mathbf{G}$ (the second column of $\Phi$). Then
\[
\mathbf{Q}_d\approx q\cdot\frac{\Ts}{6}\left(u_0u_0^\top+4u_mu_m^\top+u_1u_1^\top\right),
\]
where $u_0=\mathbf{u}(0)$, $u_m=\mathbf{u}(\Ts/2)$, $u_1=\mathbf{u}(\Ts)$.

% ==============================================================================
\section{Accelerometer Measurement Update}

\subsection{Measurement model}
The accelerometer measures specific force in BODY' (after de-heel).
Let WORLD gravity be $\gvec=[0,0,g]^\top$ with $+Z$ down (NED). The predicted mean is
\[
\hat{\accm} = \RwB(\aw-\gvec) + \mathbf{a}_{\mathrm{lever}} + \baterm.
\]
Residual:
\[
\mathbf{r}_a = \accm - \hat{\accm}.
\]

\subsection{Jacobian blocks}
Let $\mathbf{f}_{\mathrm{cog}}=\RwB(\aw-\gvec)$. Using the variation rule
\[
\delta(\RwB\mathbf{u})=-\xskew{\RwB\mathbf{u}}\,\dth,
\]
the attitude Jacobian is
\[
\mathbf{J}_\theta = \frac{\partial\hat{\accm}}{\partial\dth} = -\xskew{\mathbf{f}_{\mathrm{cog}}}.
\]
From the definition of $\aw$:
\[
\frac{\partial \hat{\accm}}{\partial \pvec_k}=\RwB(-\omega_k^2\I),\qquad
\frac{\partial \hat{\accm}}{\partial \vvec_k}=\RwB(-2\zeta_k\omega_k\I).
\]
If accel bias updates are enabled: $\mathbf{J}_{ba}=\I$. If bias updates are disabled, the bias mean may be omitted but its covariance is treated as nuisance noise.

\subsection{Innovation covariance, marginalization, and NIS}
Let $\mathbf{R}_{acc}$ be diagonal accelerometer measurement covariance. Innovation covariance:
\[
\mathbf{S}=\mathbf{R}_{acc}+\mathbf{J}\mathbf{P}\mathbf{J}^\top.
\]
If the wave block is disabled, add missing wave-accel uncertainty:
\[
\mathbf{S}\leftarrow \mathbf{S}+\RwB\,\Sigma_{\aw}^{(\mathrm{disabled})}\,\RwB^\top.
\]
If accel-bias updates are disabled, add $\mathbf{P}_{ba,ba}$ to $\mathbf{S}$ as nuisance:
\[
\mathbf{S}\leftarrow \mathbf{S}+\mathbf{P}_{ba,ba}.
\]
Compute NIS and reject if too large:
\[
\mathrm{NIS}=\mathbf{r}_a^\top \mathbf{S}^{-1}\mathbf{r}_a.
\]

\subsection{Gain, state update, Joseph covariance update}
Gain and state update:
\[
\mathbf{K}=\mathbf{P}\mathbf{J}^\top\mathbf{S}^{-1},\qquad
\mathbf{x}\leftarrow \mathbf{x}+\mathbf{K}\mathbf{r}_a.
\]
Covariance update (Joseph-equivalent form used in the implementation):
\[
\mathbf{P}\leftarrow \mathbf{P}
- \mathbf{K}(\mathbf{P}\mathbf{J}^\top)
- (\mathbf{K}(\mathbf{P}\mathbf{J}^\top))^\top
+ \mathbf{K}\mathbf{S}\mathbf{K}^\top.
\]
Then apply quaternion correction:
\[
\mathbf{q}_{\mathrm{ref}} \leftarrow \mathbf{q}_{\mathrm{ref}} \otimes \exp\!\left(\frac{1}{2}\dth\right),\qquad
\dth\leftarrow \0.
\]

% ==============================================================================
\section{Magnetometer Update}
\subsection{Horizontal direction residual}
Let $\mathbf{B}_W$ be the world magnetic reference. Predicted field in BODY':
\[
\hat{\mathbf{b}}=\RwB\mathbf{B}_W.
\]
Predicted down direction in BODY':
\[
\mathbf{d}=\RwB\eThree,\quad \eThree=[0,0,1]^\top.
\]
Horizontal projection:
\[
\mathbf{P}_h=\I-\mathbf{d}\mathbf{d}^\top.
\]
Project and normalize:
\[
\mathbf{z}=\frac{\mathbf{P}_h\magm}{\norm{\mathbf{P}_h\magm}},\qquad
\mathbf{h}=\frac{\mathbf{P}_h\hat{\mathbf{b}}}{\norm{\mathbf{P}_h\hat{\mathbf{b}}}}.
\]
Resolve 180\textdegree\ ambiguity: if $\mathbf{z}^\top\mathbf{h}<0$, set $\mathbf{h}\leftarrow-\mathbf{h}$.
Residual: $\mathbf{r}_m=\mathbf{z}-\mathbf{h}$.

\subsection{Finite-difference Jacobian (robust and convention-matched)}
Compute $\mathbf{J}_\theta$ by symmetric finite differences consistent with right-multiply correction:
\[
\mathbf{J}_\theta[:,i]\approx
\frac{\mathbf{h}(\mathbf{q}_{\mathrm{ref}}\otimes\exp(\tfrac12\eps \mathbf{e}_i))-\mathbf{h}(\mathbf{q}_{\mathrm{ref}}\otimes\exp(-\tfrac12\eps \mathbf{e}_i))}{2\eps},
\]
where $\mathbf{h}(\cdot)$ recomputes the predicted direction and enforces the same sign branch as nominal.

\subsection{Direction noise scaling and base-only update}
A practical direction noise variance is
\[
\sigma_{\mathrm{dir}}^2 \approx \frac{\tfrac13 \mathrm{tr}(\mathbf{R}_{mag})}{\norm{\mathbf{P}_h\magm}^2}.
\]
Then $\mathbf{S}\approx \sigma_{\mathrm{dir}}^2\I + \mathbf{J}_\theta \mathbf{P}_{\theta\theta}\mathbf{J}_\theta^\top$, NIS gate, gain, and Joseph-form update.
To prevent coupling into wave/bias states, force the gain rows for wave and accel bias to zero (base-only update), then apply quaternion correction.

% ==============================================================================
\section{Warmup and Staged Enabling}
Typical staged policy:
\begin{itemize}
\item Warmup ON: wave block disabled and wave covariance reset small; lever arm disabled; accel-bias updates may be disabled; gyro-bias learning gated by stationarity.
\item Exit warmup: after motion/time thresholds, enable wave block and restore nominal measurement noise.
\item Mag delay: start magnetometer update only after a configured delay.
\end{itemize}

% ==============================================================================
\section{Axis-Independent Covariance Approximation}
If enabled, enforce axis-independence by zeroing cross-axis blocks of $\mathbf{P}$.
Let $\mathcal{I}_x,\mathcal{I}_y,\mathcal{I}_z$ contain all states belonging to each axis (attitude component, optional gyro bias component, all mode $(p,v)$ components, optional accel bias component).
After propagation and after each measurement update:
\begin{enumerate}
\item Set $\mathbf{P}_{\mathcal{I}_a,\mathcal{I}_b}\leftarrow \0$ for $a\neq b$.
\item PSD-project each axis submatrix $\mathbf{P}_{\mathcal{I}_a,\mathcal{I}_a}$, clamp diagonal, symmetrize.
\end{enumerate}
This is a stability approximation and can break statistical optimality when true dynamics couple axes.

% ==============================================================================
\section{Algorithms (Implementation-Traceable)}
\begin{algorithm}[t]
\caption{Per-sample update loop}
\begin{algorithmic}[1]
\Require IMU sample $(\w_m,\accm)$ at period $\Ts$, temperature $T$; optional mag sample $\magm$
\State De-heel IMU vectors if needed
\State TimeUpdate$(\w_m,\Ts)$
\If{warmup\_mode}
  \State WarmupUpdate$(\accm,\w_m,\Ts)$
\EndIf
\State AccelUpdate$(\accm,T)$
\If{with\_mag and time $>$ mag\_delay and mag sample available}
  \State MagUpdate$(\magm)$
\EndIf
\State Symmetrize $\mathbf{P}$; clamp diagonal; PSD projection (as configured)
\If{axis\_independence}
  \State EnforceAxisIndependence$(\mathbf{P})$
\EndIf
\end{algorithmic}
\end{algorithm}

\end{document}
