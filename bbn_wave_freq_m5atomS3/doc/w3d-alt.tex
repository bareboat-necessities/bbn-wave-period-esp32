\documentclass[10pt,twocolumn]{article}

% ===================== Packages =====================
\usepackage[T1]{fontenc}
\usepackage{lmodern}
\usepackage{microtype}
\usepackage{mathtools,amssymb,amsfonts}
\usepackage{bm}
\usepackage{physics}      % \norm{}, \qty{}
\usepackage{siunitx}
\usepackage{geometry}
\usepackage{hyperref}
\usepackage{booktabs}
\usepackage{enumitem}

\geometry{margin=0.68in}
\setlength{\columnsep}{0.22in}
\setlist[itemize]{leftmargin=*,itemsep=2pt,topsep=2pt}
\setlist[enumerate]{leftmargin=*,itemsep=2pt,topsep=2pt}

% ===================== Macros =====================
\newcommand{\R}{\mathbb{R}}
\newcommand{\SO}{\mathrm{SO}}
\newcommand{\skew}[1]{\left[#1\right]_{\times}}
\newcommand{\diag}{\mathrm{diag}}
\newcommand{\E}{\mathbb{E}}
\newcommand{\I}{\mathbf{I}}
\newcommand{\0}{\mathbf{0}}

\newcommand{\qref}{\mathbf{q}_{\text{ref}}}
\newcommand{\dth}{\delta\bm{\theta}}
\newcommand{\bg}{\mathbf{b}_g}
\newcommand{\ba}{\mathbf{b}_a}
\newcommand{\aw}{\mathbf{a}_w}
\newcommand{\gvec}{\mathbf{g}}
\newcommand{\RwB}{\mathbf{R}_{wb}}
\newcommand{\RbW}{\mathbf{R}_{bw}}
\newcommand{\eThree}{\mathbf{e}_3}

\newcommand{\pvec}{\mathbf{p}}
\newcommand{\vvec}{\mathbf{v}}
\newcommand{\w}{\bm{\omega}}
\newcommand{\accm}{\mathbf{a}_m}
\newcommand{\magm}{\mathbf{m}_m}

\newcommand{\Hs}{H_s}
\newcommand{\Ts}{\Delta t}

\title{\vspace{-0.35in}\bfseries
Kalman3D\_Wave\_2: A Broadband Wave Oscillator + Quaternion Error-State Filter\\
Implementation-Matched Mathematics and Design Notes
}
\author{Mikhail Grushinskiy}
\date{\vspace{-0.3in}}

\begin{document}
\maketitle
\vspace{-0.25in}

\begin{abstract}
This note documents the math and implementation details of a marine IMU fusion filter that combines:
(i) a quaternion error-state attitude filter (qMEKF style), (ii) a broadband sea-motion model implemented as a bank of damped oscillators producing wave displacement/velocity/acceleration in WORLD, and (iii) robust tilt-compensated magnetometer yaw correction using a direction-only residual.
It also describes the staged warmup policy (wave block disabled and bias learning gated), innovation gating (NIS), and an optional stabilization approximation that forces axis-independent covariances by removing cross-axis blocks in the covariance matrix.
\end{abstract}

% ==============================================================================
\section{Frames, State Convention, and Update Order}
\paragraph{Frames.}
WORLD is NED by default: $+Z$ points \emph{down}. BODY' is a \emph{virtual} body frame with heel removed (the physical heel can be re-applied externally).

\paragraph{Reference quaternion convention.}
The filter stores
\[
\qref:\; \text{WORLD}\rightarrow \text{BODY'}.
\]
Thus the direction-cosine matrix is
\[
\RwB = \mathbf{R}(\qref)\in\SO(3),\qquad \RbW = \RwB^\top.
\]

\paragraph{Error state and correction.}
The small attitude error is $\dth\in\R^3$. The implementation uses \emph{right-multiplication} correction:
\[
\qref \leftarrow \qref \otimes \delta q,\quad \delta q = \exp\!\left(\frac{1}{2}\dth\right),
\qquad \dth \leftarrow \0.
\]
For small angles, $\delta q \approx \begin{bmatrix}1 & \tfrac12\dth^\top\end{bmatrix}^\top$.

\paragraph{One-step processing order (per IMU sample).}
\begin{enumerate}
\item Time update: propagate $\qref$, wave states, and covariance.
\item Accelerometer measurement update (and optional warmup logic/bias gating).
\item Magnetometer update (delayed, robust direction-only, typically base-only update).
\end{enumerate}

% ==============================================================================
\section{State Vector and Dimension}
Let $K$ be \texttt{KMODES}. For each mode $k\in\{1,\dots,K\}$, the wave block stores $\pvec_k,\vvec_k\in\R^3$ in WORLD.

\paragraph{State definition.}
The error-state vector is
\[
\mathbf{x}=
\begin{bmatrix}
\dth \\
(\bg) \\
\pvec_1\\ \vvec_1\\
\vdots\\
\pvec_K\\ \vvec_K\\
(\ba)
\end{bmatrix}.
\]
$\bg,\ba$ are optional (compile-time toggles).

\paragraph{Dimension.}
\[
N_X = (\texttt{with\_gyro\_bias ? }6:3) + 6K + (\texttt{with\_accel\_bias ? }3:0).
\]
With defaults $K=3$, gyro bias on, accel bias on: $N_X=27$.

\paragraph{State ordering (implementation-relevant).}
Within each mode block, the implementation stores \emph{p first, then v}:
\[
\text{mode }k:\quad
[\;p_{k,x},p_{k,y},p_{k,z},\,v_{k,x},v_{k,y},v_{k,z}\;]^\top.
\]
This matters when assembling $6\times6$ discretizations from per-axis $2\times2$ blocks.

% ==============================================================================
\section{Broadband Wave Oscillator Model}
\subsection{Continuous-time dynamics (per axis)}
For each mode $k$ and each axis independently:
\begin{align}
\dot p_k &= v_k,\\
\dot v_k &= -\omega_k^2 p_k - 2\zeta_k\omega_k v_k + \xi_k(t),
\end{align}
where $\xi_k(t)$ is white noise driving \emph{acceleration} in the $v$ equation (units $\mathrm{m/s^3}$).
A mode has parameters $(\omega_k,\zeta_k,q_k)$, where
\[
\E[\xi_k(t)\xi_k(\tau)] = q_k \delta(t-\tau),\qquad q_k\;\;[\mathrm{m^2/s^5}].
\]

\subsection{Wave acceleration reconstruction}
The WORLD wave acceleration produced by the oscillator bank is
\begin{equation}
\aw = \sum_{k=1}^{K}\bigl(-\omega_k^2\pvec_k - 2\zeta_k\omega_k\vvec_k\bigr).
\label{eq:aw}
\end{equation}

\subsection{Broadband parameterization}
A ``broadband'' tuning selects $\omega_k$ in a log-spread around a base frequency $f_0$,
with weights $w_k$ such that the total displacement variance matches the significant height $\Hs$ via
\[
\sigma_{\text{disp}} \approx \Hs/4,\qquad
\sum_k \Var(\pvec_k)\approx \sigma_{\text{disp}}^2.
\]
A common steady-state relationship for a driven oscillator implies a mapping from displacement variance to the driving intensity $q_k$ (implementation uses a closed-form steady-state moment relationship).
The same steady-state moments can also produce a diagonal \emph{disabled-wave} acceleration covariance $\Sigma_{\aw}^{(\text{disabled})}$.

% ==============================================================================
\section{Discrete-Time Propagation}
\subsection{Attitude propagation}
Given measured gyro $\w_m$ and estimated bias $\bg$, the bias-corrected rate is
\[
\w = \w_m - \bg.
\]
Update:
\[
\qref \leftarrow \qref \otimes \exp\!\left(\frac{1}{2}\w\,\Ts\right).
\]
The base error-state transition is the standard small-error linearization; importantly,
\[
\dth_{k+1} \approx \mathbf{F}_{\theta\theta}\dth_k - \I\,\Ts\,\delta\bg_k + \cdots
\]
(when gyro bias is enabled).

\subsection{Oscillator discretization}
For each mode $k$ and axis $a\in\{x,y,z\}$, define $\mathbf{s}_{k,a}=[p_{k,a}, v_{k,a}]^\top$.
The implementation computes a $2\times2$ transition $\Phi_{k,a}(\Ts)$ in closed form for under/critical/over-damped cases and a discrete covariance $Q_{d,k,a}(\Ts)$.

\paragraph{Noise covariance by integration (Simpson).}
Let $\mathbf{u}(t)$ be the second column of $\Phi(t)$ for the continuous-time system (the impulse response from $\xi$ into state).
Then
\[
Q_d = q\int_0^{\Ts} \mathbf{u}(t)\mathbf{u}(t)^\top dt
\approx q\cdot\frac{\Ts}{6}\bigl(f(0)+4f(\Ts/2)+f(\Ts)\bigr),
\]
where $f(t)=\mathbf{u}(t)\mathbf{u}(t)^\top$.

\paragraph{$6\times6$ assembly matched to state ordering.}
The mode state is ordered as $(\pvec_k,\vvec_k)$, so the correct assembly is
\[
\Phi_{6,k}=
\begin{bmatrix}
A & B\\
C & D
\end{bmatrix},
\quad
A=\diag(\Phi_{00}^x,\Phi_{00}^y,\Phi_{00}^z),\;\;
B=\diag(\Phi_{01}^x,\Phi_{01}^y,\Phi_{01}^z),
\]
and similarly $C=\diag(\Phi_{10}^a)$, $D=\diag(\Phi_{11}^a)$.
$Q_{d,6,k}$ is assembled the same way from $Q_{d,k,a}$ entries:
\[
Q_{d,6,k}=
\begin{bmatrix}
Q_{00} & Q_{01}\\
Q_{10} & Q_{11}
\end{bmatrix},
\quad
Q_{00}=\diag(Q_{00}^x,Q_{00}^y,Q_{00}^z),\;\;
Q_{01}=\diag(Q_{01}^x,Q_{01}^y,Q_{01}^z),\;\;Q_{10}=Q_{01}^\top.
\]

\subsection{Covariance propagation (block form)}
Let the full covariance be $\mathbf{P}\in\R^{N_X\times N_X}$. Propagation uses the usual
\[
\mathbf{P}\leftarrow \mathbf{F}\mathbf{P}\mathbf{F}^\top+\mathbf{Q}_d,
\]
implemented by propagating the base block with its own $\mathbf{F}_{AA},\mathbf{Q}_{AA}$, each wave mode block with $\Phi_{6,k},Q_{d,6,k}$, and cross-blocks with the appropriate left/right multiplications.

% ==============================================================================
\section{Accelerometer Measurement Update}
\subsection{Measurement model}
The accelerometer measures specific force in BODY' (after de-heel):
\[
\accm = \RwB(\aw - \gvec) + \text{lever-arm} + \text{bias/temperature} + \eta_a,
\]
where $\gvec=[0,0,g]^\top$ in WORLD ($+Z$ down). The predicted mean (with lever-arm/bias terms) is $\hat{\accm}$.
Residual:
\[
\mathbf{r}_a = \accm - \hat{\accm}.
\]

\subsection{Key Jacobians}
Let $\mathbf{f}_{\text{cog}} = \RwB(\aw - \gvec)$ (the COG specific force in BODY').
Using $\delta(\RwB \uvec) = -\skew{\RwB\uvec}\,\dth$, the attitude Jacobian is
\[
\mathbf{J}_{\theta} = -\skew{\mathbf{f}_{\text{cog}}}.
\]
Wave Jacobians follow from \eqref{eq:aw}:
\[
\frac{\partial \aw}{\partial \pvec_k} = -\omega_k^2 \I,\qquad
\frac{\partial \aw}{\partial \vvec_k} = -2\zeta_k\omega_k \I,
\]
so in BODY':
\[
\frac{\partial \hat{\accm}}{\partial \pvec_k} = \RwB(-\omega_k^2 \I),\qquad
\frac{\partial \hat{\accm}}{\partial \vvec_k} = \RwB(-2\zeta_k\omega_k \I).
\]
If accel bias updates are enabled, $\partial \hat{\accm}/\partial \ba = \I$; if disabled, $\ba$ is treated as nuisance (mean may omit it while its covariance inflates innovation).

\subsection{Innovation covariance and disabled-wave marginalization}
Let $\mathbf{R}_{acc}$ be accelerometer measurement noise.
Innovation covariance:
\[
\mathbf{S} = \mathbf{R}_{acc} + \mathbf{J}\mathbf{P}\mathbf{J}^\top,
\]
with additional term when the wave block is disabled:
\[
\mathbf{S}\leftarrow \mathbf{S} + \RwB\,\Sigma_{\aw}^{(\text{disabled})}\,\RwB^\top.
\]
This approximates the missing wave acceleration as extra measurement noise.

\subsection{NIS gate and Joseph-form covariance update}
Compute NIS:
\[
\text{NIS}=\mathbf{r}_a^\top \mathbf{S}^{-1}\mathbf{r}_a.
\]
If NIS is too large, reject the update.
Otherwise gain:
\[
\mathbf{K}=\mathbf{P}\mathbf{J}^\top\mathbf{S}^{-1}.
\]
State update $\mathbf{x}\leftarrow\mathbf{x}+\mathbf{K}\mathbf{r}_a$, then covariance update in Joseph form:
\[
\mathbf{P}\leftarrow \mathbf{P} - \mathbf{K}(\mathbf{P}\mathbf{J}^\top)
                        - (\mathbf{K}(\mathbf{P}\mathbf{J}^\top))^\top
                        + \mathbf{K}\mathbf{S}\mathbf{K}^\top,
\]
followed by quaternion correction from $\dth$ and clearing $\dth$.

% ==============================================================================
\section{Magnetometer Update: Tilt-Compensated Direction-Only}
The magnetometer is used to correct yaw without relying on magnitude.
Let $\mathbf{B}_W$ be the world magnetic reference (fixed or learned). Predicted field in BODY':
\[
\hat{\mathbf{b}} = \RwB \mathbf{B}_W.
\]
Predicted down direction in BODY':
\[
\mathbf{d} = \RwB \eThree,\qquad \eThree=[0,0,1]^\top.
\]
Horizontal projector:
\[
\mathbf{P}_h = \I - \mathbf{d}\mathbf{d}^\top.
\]
Project and normalize:
\[
\mathbf{z}=\frac{\mathbf{P}_h \magm}{\norm{\mathbf{P}_h \magm}},\qquad
\mathbf{h}=\frac{\mathbf{P}_h \hat{\mathbf{b}}}{\norm{\mathbf{P}_h \hat{\mathbf{b}}}}.
\]
Resolve $180^\circ$ ambiguity:
if $\mathbf{z}^\top\mathbf{h}<0$, set $\mathbf{h}\leftarrow-\mathbf{h}$.
Residual:
\[
\mathbf{r}_m = \mathbf{z}-\mathbf{h}.
\]

\paragraph{Jacobian.}
A reliable implementation strategy is finite-difference w.r.t. $\dth$ using the same correction convention
$\qref \leftarrow \qref \otimes \exp(\tfrac12\dth)$,
\emph{while keeping the same branch choice} (i.e., enforce $\mathbf{h}$ sign to match nominal).

\paragraph{Direction-domain noise.}
Because only direction is used, the measurement noise is scaled by the projected magnitude:
\[
\mathbf{S} \approx \sigma_{\text{dir}}^2 \I + \mathbf{J}_{\theta}\mathbf{P}_{\theta\theta}\mathbf{J}_{\theta}^\top,
\qquad
\sigma_{\text{dir}}^2 \propto \frac{\sigma_m^2}{\norm{\mathbf{P}_h \magm}^2}.
\]
NIS gating is applied as for the accelerometer.

\paragraph{Base-only update (recommended).}
To prevent magnetometer updates from coupling into wave/bias states via cross-covariances, the update may be restricted to the base attitude block (and optionally gyro bias). In matrix terms, only the corresponding rows of $\mathbf{K}$ are allowed to be nonzero.

% ==============================================================================
\section{Warmup Policy and Bias Gating}
The implementation supports a staged warmup mode:
\begin{itemize}
\item \textbf{Warmup ON:} wave block disabled, wave states/cov reset small; accel-bias updates disabled (optional); lever-arm disabled; gyro-bias may be learned only when stationarity tests pass.
\item \textbf{Warmup exit:} detect sufficient motion (time/distance heuristics from integrated non-gravity acceleration); then enable wave block and restore nominal settings.
\item \textbf{Mag delay:} magnetometer updates start after a configurable delay to avoid early transients.
\end{itemize}

% ==============================================================================
\section{Axis-Independent Covariance Approximation}
Sometimes the filter is more stable if cross-axis covariances are forcibly removed. Define axis index sets
\[
\mathcal{I}_x,\;\mathcal{I}_y,\;\mathcal{I}_z
\]
containing the indices for $(\dth_a, (b_{g,a}), p_{1,a},v_{1,a},\dots,p_{K,a},v_{K,a}, (b_{a,a}))$ for axis $a\in\{x,y,z\}$.

After each propagation and measurement update:
\begin{enumerate}
\item \textbf{Zero cross-axis blocks:} $\mathbf{P}_{\mathcal{I}_a,\mathcal{I}_b}\leftarrow \0$ for $a\neq b$.
\item \textbf{PSD per axis:} symmetrize each $\mathbf{P}^{(a)}=\mathbf{P}_{\mathcal{I}_a,\mathcal{I}_a}$, clamp diagonal, project to PSD, write back.
\item Symmetrize the full $\mathbf{P}$ again.
\end{enumerate}
This enforces
\[
\mathbf{P}\approx \diag\bigl(\mathbf{P}^{(x)},\mathbf{P}^{(y)},\mathbf{P}^{(z)}\bigr).
\]
\textbf{Important:} this is an approximation. The measurement residuals are still 3D, but cross-axis correlations are discarded afterward.

% ==============================================================================
\section{Implementation Checklist (What Must Match Code)}
This section exists because most ``docs'' fail by drifting from the code.

\begin{itemize}
\item \textbf{Quaternion convention:} $\qref$ is WORLD$\rightarrow$BODY' and correction is \emph{right-multiply}.
\item \textbf{WORLD is NED:} gravity is $\gvec=[0,0,g]^\top$ with $+Z$ down.
\item \textbf{Wave state ordering:} $(\pvec_k,\vvec_k)$ with p(3) then v(3). Your $\Phi_6,Q_{d,6}$ assembly must reflect this (A/B/C/D diagonal blocks).
\item \textbf{Wave acceleration:} $\aw$ is computed in WORLD as \eqref{eq:aw}, then rotated into BODY' only inside measurement prediction.
\item \textbf{Warmup behavior:} wave disabled implies innovation inflation by $\Sigma_{\aw}^{(\text{disabled})}$, and wave rows in gains are frozen/zeroed.
\item \textbf{Mag update:} direction-only, tilt-compensated using projector $\mathbf{P}_h=\I-\mathbf{d}\mathbf{d}^\top$, with $180^\circ$ ambiguity handling.
\item \textbf{Stability steps:} symmetrize, clamp diagonal, PSD projection (local blocks), and (optionally) axis-independence enforcement.
\end{itemize}

% ==============================================================================
\section{Typical Parameters and Scaling (Quick Reference)}
\begin{table}[h]
\centering
\begin{tabular}{@{}ll@{}}
\toprule
Symbol / knob & Meaning \\ \midrule
$K$ & number of oscillator modes (\texttt{KMODES}) \\
$\omega_k,\zeta_k$ & mode frequency/damping \\
$q_k$ & acceleration-drive intensity into $v'$ \\
$\mathbf{R}_{acc}$ & accel measurement covariance (BODY') \\
$\mathbf{R}_{mag}$ & mag measurement covariance (BODY') \\
$\Sigma_{\aw}^{(\text{disabled})}$ & missing wave-accel covariance (WORLD) \\
NIS gate & reject outlier measurements \\
\bottomrule
\end{tabular}
\vspace{-2pt}
\end{table}

% ==============================================================================
\section{What This Filter Is (and Is Not)}
\paragraph{What it is.}
A physically interpretable, low-state-count sea-motion model that estimates heave and wave acceleration by maintaining a driven oscillator bank, while simultaneously running an attitude filter with bias states and robust yaw correction.

\paragraph{What it is not.}
Not a full INS with position integration and GPS constraints. Also, if axis-independence is enabled, it is no longer statistically optimal---it is a stability approximation.

\vspace{2pt}
\hrule
\vspace{4pt}
\small
\noindent
If you want, I can extend this into a \emph{paper-style} description with: (i) a full block-matrix derivation of $\mathbf{F},\mathbf{Q}_d$ for the base error-state, (ii) explicit forms of the LTI oscillator $\Phi(t)$ for each damping regime, and (iii) a complete ``algorithm'' listing with pseudocode that is line-by-line traceable to the implementation.

\end{document}
