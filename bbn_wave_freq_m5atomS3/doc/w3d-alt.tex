\documentclass[10pt,twocolumn]{article}

% -------------------- Packages --------------------
\usepackage[T1]{fontenc}
\usepackage{lmodern}
\usepackage{microtype}
\usepackage{mathtools,amssymb,amsfonts}
\usepackage{bm}
\usepackage{geometry}
\usepackage{hyperref}

\geometry{margin=0.7in}
\setlength{\columnsep}{0.22in}

% -------------------- Macros --------------------
\newcommand{\R}{\mathbb{R}}
\newcommand{\SO}{\mathrm{SO}}
\newcommand{\skew}[1]{\left[#1\right]_\times}
\newcommand{\diag}{\mathrm{diag}}
\newcommand{\E}{\mathbb{E}}
\newcommand{\I}{\mathbf{I}}
\newcommand{\0}{\mathbf{0}}
\newcommand{\q}{\mathbf{q}}
\newcommand{\w}{\bm{\omega}}
\newcommand{\ba}{\mathbf{b}_a}
\newcommand{\bg}{\mathbf{b}_g}
\newcommand{\dth}{\delta\bm{\theta}}
\newcommand{\aw}{\mathbf{a}_w}
\newcommand{\gvec}{\mathbf{g}}
\newcommand{\uvec}{\mathbf{u}}
\newcommand{\vvec}{\mathbf{v}}
\newcommand{\pvec}{\mathbf{p}}
\newcommand{\norm}[1]{\left\lVert #1 \right\rVert}

\title{\vspace{-0.3in}Kalman3D\_Wave\_2: Broadband Wave Oscillator + Error-State Attitude Filter\\
with Axis-Independent Covariance Enforcement}
\author{Mikhail Grushinskiy}
\date{\vspace{-0.25in}}

\begin{document}
\maketitle
\vspace{-0.25in}

\begin{abstract}
This document describes a two-stage marine IMU fusion method that estimates attitude (via an error-state quaternion filter) and linear sea-motion (via a bank of damped harmonic oscillators). The filter supports optional gyro bias and accelerometer bias states, staged warmup behavior, and a robust magnetometer update that corrects yaw using a tilt-compensated direction-only residual. A practical stabilization modification is included: all cross-axis covariances are removed so that the covariance becomes block-diagonal across the X/Y/Z axis groups.
\end{abstract}

\vspace{-0.15in}
\section{Frames and Conventions}
We use a world frame \emph{W} (typically NED, $+Z$ down) and a virtual body frame \emph{B'} that is ``un-heeled'' (the physical heel is handled separately). The filter stores a reference quaternion
\[
\q_{\text{ref}} \;:\; \text{WORLD} \rightarrow \text{BODY'}.
\]
The user-facing orientation (BODY'$\rightarrow$WORLD) is $\q = \q_{\text{ref}}^{*}$ (conjugate). Error-state attitude uses a small rotation vector $\dth\in\R^3$ applied by right-multiplication:
\[
\q_{\text{ref}} \leftarrow \q_{\text{ref}}\otimes \exp(\tfrac12\dth), \qquad \dth \leftarrow \0.
\]

\section{State Vector}
Let $K$ be the number of oscillator modes (\texttt{KMODES}). For each mode $k$ we maintain position and velocity in WORLD, $\pvec_k,\vvec_k\in\R^3$.

\subsection{State Definition}
The error-state vector is
\[
\mathbf{x} \;=\;
\begin{bmatrix}
\dth \\
(\bg)\; \\
\pvec_1 \\ \vvec_1 \\
\vdots \\
\pvec_K \\ \vvec_K \\
(\ba)
\end{bmatrix}
\]
where $\bg\in\R^3$ is optional gyro bias and $\ba\in\R^3$ is optional accelerometer bias.

\subsection{Dimension}
\[
N_X \;=\; (\,\text{gyro bias? }6:3\,)\;+\;6K\;+\;(\,\text{acc bias? }3:0\,).
\]
With defaults $K=3$, gyro bias on, accel bias on: $N_X=6+18+3=27$.

\section{Process Model}
\subsection{Attitude Propagation}
Given measured body angular rate $\w_m$ (BODY') and gyro bias estimate $\bg$, the bias-corrected rate is
\[
\w = \w_m - \bg.
\]
The reference quaternion is propagated by
\[
\q_{\text{ref}} \leftarrow \q_{\text{ref}} \otimes \exp\!\left(\tfrac12 \w \Delta t\right).
\]
The attitude error-state transition is approximated by the rotation matrix induced by $\w$ over $\Delta t$, and gyro-bias enters as a first-order term (when enabled):
\[
\dth_{k+1} \approx \mathbf{F}_{\theta\theta}\dth_k \;+\; \mathbf{F}_{\theta b_g}\bg_k \;+\; \cdots
\]
with the common small-angle structure $\mathbf{F}_{\theta b_g}\approx -\I\,\Delta t$.

\subsection{Broadband Wave Oscillator Bank}
For each mode $k$ and each axis independently, the continuous dynamics are
\begin{align}
\dot{p}_{k} &= v_{k},\\
\dot{v}_{k} &= -\omega_k^2 p_{k} - 2\zeta_k\omega_k v_{k} + \xi_k(t),
\end{align}
where $\xi_k(t)$ is white acceleration-noise driving $v_k$ (units $\mathrm{m/s^3}$). In vector form per axis this is a linear time-invariant (LTI) oscillator.

The \emph{wave acceleration in WORLD} is reconstructed from the oscillator states:
\begin{equation}
\aw \;=\; \sum_{k=1}^{K}\left(-\omega_k^2 \pvec_k - 2\zeta_k\omega_k \vvec_k\right).
\label{eq:aw}
\end{equation}

\subsection{Discretization}
Each axis uses a $2\times2$ state transition $\Phi(\Delta t)$ (exact/closed-form for damped oscillators) and a discrete noise covariance $Q_d(\Delta t)$ computed by numerical integration (Simpson rule). The 3D mode block is assembled as a $6\times6$ block consistent with the state order $(p_x,p_y,p_z,v_x,v_y,v_z)$:
\[
\Phi_{6}=
\begin{bmatrix}
A & B\\
C & D
\end{bmatrix},\quad
A=\diag(\phi_{00}^x,\phi_{00}^y,\phi_{00}^z),\;\;
B=\diag(\phi_{01}^x,\phi_{01}^y,\phi_{01}^z),
\]
(and similarly for $C,D$). $Q_{d,6}$ is assembled analogously from per-axis $Q_d$.

\subsection{Bias Random Walks}
When enabled:
\[
\bg_{k+1} = \bg_k + \eta_g,\quad
\ba_{k+1} = \ba_k + \eta_a,
\]
with diagonal process covariances $Q_{b_g}\Delta t$ and $Q_{b_a}\Delta t$.

\section{Accelerometer Measurement Model}
The accelerometer measures \emph{specific force} in BODY' (including gravity in the raw IMU convention). Let $\mathbf{R}_{wb}\in\SO(3)$ map WORLD$\rightarrow$BODY' (this is $\q_{\text{ref}}$ as a rotation matrix). Let world gravity be
\[
\gvec = \begin{bmatrix}0&0&g\end{bmatrix}^\top \quad (\text{WORLD, } +Z\text{ down}).
\]
The predicted specific force at the IMU (ignoring optional lever-arm terms for brevity) is
\begin{equation}
\hat{\mathbf{f}} \;=\; \mathbf{R}_{wb}\left(\aw - \gvec\right) + \ba_{\text{term}},
\label{eq:acc_pred}
\end{equation}
where $\ba_{\text{term}}$ may include temperature compensation and staged enabling logic.

The residual is
\[
\mathbf{r}_a = \mathbf{z}_a - \hat{\mathbf{f}}.
\]
The attitude Jacobian uses the standard perturbation $\delta(\mathbf{R}_{wb}\uvec) = -\skew{\mathbf{R}_{wb}\uvec}\dth$, giving
\[
\mathbf{J}_{\theta} = -\skew{\mathbf{f}_{\text{cog}}},\qquad \mathbf{J}_{b_a}=\I,
\]
and wave Jacobians follow from \eqref{eq:aw}.

\subsection{Disabled Wave Marginalization}
If the wave block is disabled (e.g., warmup), the missing wave acceleration uncertainty is injected into the innovation covariance:
\[
\mathbf{S} \leftarrow \mathbf{S} + \mathbf{R}_{wb}\,\Sigma_{\aw}^{(\text{disabled})}\,\mathbf{R}_{wb}^\top,
\]
with diagonal $\Sigma_{\aw}^{(\text{disabled})}$ derived from steady-state oscillator moments.

\subsection{Joseph Covariance Update}
After a measurement update with gain $\mathbf{K}$ and innovation covariance $\mathbf{S}$, the covariance is updated in Joseph form to reduce numerical asymmetry:
\[
\mathbf{P} \leftarrow \mathbf{P} - \mathbf{K}\mathbf{P}\mathbf{H}^\top
                 - (\mathbf{K}\mathbf{P}\mathbf{H}^\top)^\top
                 + \mathbf{K}\mathbf{S}\mathbf{K}^\top.
\]

\section{Magnetometer Update (Direction-Only)}
The magnetometer update uses only the \emph{horizontal direction} of the magnetic field to avoid sensitivity to magnitude errors. Let $\mathbf{b}_W$ be the world magnetic reference (learned or configured). The predicted field in BODY' is
\[
\hat{\mathbf{b}} = \mathbf{R}_{wb}\mathbf{b}_W.
\]
Let $\mathbf{d}=\mathbf{R}_{wb}\mathbf{e}_3$ be the predicted down direction in BODY' ($\mathbf{e}_3=[0,0,1]^\top$ in WORLD). Define the horizontal projection
\[
\mathbf{P}_h = \I - \mathbf{d}\mathbf{d}^\top.
\]
Project and normalize:
\[
\mathbf{z} = \frac{\mathbf{P}_h \mathbf{m}}{\|\mathbf{P}_h\mathbf{m}\|},\qquad
\mathbf{h} = \frac{\mathbf{P}_h \hat{\mathbf{b}}}{\|\mathbf{P}_h\hat{\mathbf{b}}\|}.
\]
Resolve the $180^\circ$ ambiguity by flipping $\mathbf{h}$ if needed to minimize residual:
\[
\text{if }\mathbf{z}^\top\mathbf{h}<0 \;\Rightarrow\; \mathbf{h}\leftarrow -\mathbf{h}.
\]
Residual:
\[
\mathbf{r}_m = \mathbf{z}-\mathbf{h}.
\]
A practical Jacobian is computed by finite differences with the same correction convention
$\q_{\text{ref}}\leftarrow \q_{\text{ref}}\otimes \exp(\tfrac12\dth)$.
The magnetometer update is applied to the \emph{base attitude block only} (no wave/bias updates via cross-covariances).

\section{Warmup and Staged Enabling}
In warmup:
\begin{itemize}\itemsep2pt
\item wave block disabled and its state/covariance is reset to small values,
\item accelerometer-bias updates can be frozen,
\item gyro-bias can be learned only under stationarity gating.
\end{itemize}
The filter exits warmup when sufficient motion is detected (time/distance thresholds), then enables the wave block and restores nominal measurement noise.

\section{Axis-Independent Covariance Enforcement}
To remove all cross-axis covariances, define axis index sets
\[
\mathcal{I}_x,\;\mathcal{I}_y,\;\mathcal{I}_z
\]
containing the state indices for $(\dth_x, (b_{g,x}), p_{1,x},v_{1,x},\ldots,p_{K,x},v_{K,x}, (b_{a,x}))$,
and similarly for $y,z$.

After each propagation and each measurement update:
\begin{enumerate}\itemsep2pt
\item Zero all cross-axis covariance blocks:
\[
\mathbf{P}_{\mathcal{I}_a,\mathcal{I}_b} \leftarrow \0 \quad \forall a\neq b.
\]
\item For each axis $a\in\{x,y,z\}$, extract the axis submatrix
$\mathbf{P}^{(a)} = \mathbf{P}_{\mathcal{I}_a,\mathcal{I}_a}$,
symmetrize it, clamp its diagonal, and project it to PSD.
\item Write back each $\mathbf{P}^{(a)}$ and symmetrize the full $\mathbf{P}$.
\end{enumerate}
This makes the covariance block-diagonal across axes:
\[
\mathbf{P} \approx \diag\!\left(\mathbf{P}^{(x)},\mathbf{P}^{(y)},\mathbf{P}^{(z)}\right),
\]
which is a stabilizing approximation when cross-axis correlations are numerically problematic.

\section{Notes}
\begin{itemize}\itemsep2pt
\item Axis-independent covariance is an approximation: the 3D measurement residual is still computed in full, but cross-axis correlations are discarded afterward.
\item Innovation gates (e.g., NIS thresholds) are recommended for both accel and mag updates.
\end{itemize}

\vspace{-0.1in}
\hrule
\vspace{0.05in}
\small
\noindent
This document is intended as a practical mathematical reference for the implementation and can be expanded with specific tuning rules for $(\omega_k,\zeta_k,q_k)$ and the adaptive tuner if desired.

\end{document}
