\documentclass[12pt]{book}
\usepackage[utf8]{inputenc}
\usepackage{titlesec}
\usepackage{titletoc}
\usepackage{amsmath, amssymb} % For mathematical symbols
\usepackage{geometry} % For adjusting page margins
\geometry{a4paper, margin=1in}

% Formatting for Chapter titles
\titleformat{\chapter}[display]
{\normalfont\huge\bfseries}{\chaptertitlename\ \thechapter}{20pt}{\Huge}
\titlespacing*{\chapter}{0pt}{-30pt}{40pt}

% Formatting for Section titles
\titleformat{\section}
{\normalfont\Large\bfseries}{\thesection}{1em}{}

% Formatting for the Table of Contents in the TOC itself
\usepackage[titles]{tocloft}
\renewcommand{\cftchapfont}{\normalfont\bfseries}
\renewcommand{\cftchapleader}{\cftdotfill{\cftdotsep}}
\renewcommand{\cftchappagefont}{\normalfont}

\begin{document}

\frontmatter
\tableofcontents
\mainmatter

% The actual Table of Contents entries
\chapter*{Contents}

% Part I: Foundations
\part{Foundations}
\chapter{Introduction to the Oceanographic System}
\section{The Oceans: Dimensions, Structure, and Importance}
\section{The Multi-Disciplinary Nature of Oceanography}
\section{The Role of Mathematics and Theoretical Models}
\section{Observational Methods and Data Analysis}

\chapter{Essential Mathematical Tools}
\section{Scalars, Vectors, and Tensors in Oceanography}
\section{Calculus Review: Derivatives, Integrals, and Theorems (Gradient, Divergence, Curl)}
\section{Coordinate Systems: Cartesian, Cylindrical, and Spherical}
\section{Complex Numbers and Fourier Series}
\section{Introduction to Ordinary and Partial Differential Equations (ODEs/PDEs)}
\section{Basic Probability and Statistics for Oceanographic Data}

\chapter{Fluid Dynamics Preliminaries}
\section{Continuum Hypothesis and Fluid Properties}
\section{Kinematics: Lagrangian vs. Eulerian Descriptions, Material Derivative}
\section{Conservation Laws: Mass, Momentum, and Energy}
\section{Concepts of Vorticity and Circulation}

% Part II: Governing Equations and Simplifications
\part{Governing Equations and Simplifications}
\chapter{The Navier-Stokes Equations for Seawater}
\section{Derivation of the Mass Conservation (Continuity) Equation}
\section{Derivation of the Momentum Conservation Equations}
\section{Equation of State for Seawater}
\section{The Full System of Equations and Its Closure Problem}

\chapter{Simplifications and Scaling}
\section{The Boussinesq Approximation}
\section{Hydrostatic Balance}
\section{The Concept of Turbulence and Reynolds Averaging}
\section{The Reynolds-Averaged Navier-Stokes (RANS) Equations}
\section{Non-Dimensional Numbers: Rossby, Reynolds, Froude, Ekman}

% Part III: Oceanic Phenomena and Their Mathematical Descriptions
\part{Oceanic Phenomena and Their Mathematical Descriptions}
\chapter{Geostrophic Flow and Thermal Wind}
\section{The Coriolis Force on a Rotating Earth}
\section{Scaling for Large-Scale Flow: Geostrophic Balance}
\section{The Thermal Wind Equations}
\section{Calculating Geostrophic Currents from Hydrography}

\chapter{Wind-Driven Circulation and Ekman Theory}
\section{The Surface Wind Stress}
\section{The Classical Ekman Spiral Solution}
\section{Ekman Transport and Ekman Pumping}
\section{Application to Ocean Gyres and Upwelling/Downwelling}

\chapter{Ocean Waves}
\section{Linear Theory and the Perturbation Method}
\section{Surface Gravity Waves: Dispersion and Group Velocity}
\section{Internal Waves: Theory and Observations}
\section{Long Waves: Tsunamis and Tides}
\section{Nonlinear Waves: Solitons and Stokes Drift}

\chapter{Instabilities and Turbulence}
\section{Concept of Dynamic Instability}
\section{Baroclinic and Barotropic Instabilities}
\section{Kelvin-Helmholtz Instability}
\section{Homogeneous vs. Inhomogeneous Turbulence}
\section{Mixing and Eddy Diffusion Parameterizations}

\chapter{The Ocean's General Circulation}
\section{The Sverdrup Theory for Interior Flow}
\section{Western Boundary Currents: The Stommel and Munk Models}
\section{Abyssal Circulation and Thermohaline Processes}
\section{Conceptual Box Models and the Overturning Circulation}

% Part IV: Specialized Topics and Numerical Methods
\part{Specialized Topics and Numerical Methods}
\chapter{Coastal and Estuarine Dynamics}
\section{Shallow Water Equations}
\section{Tides in Coastal Seas and Amphidromic Systems}
\section{Estuarine Circulation and Salt Wedge Dynamics}
\section{Coastal Upwelling and River Plumes}

\chapter{Introduction to Numerical Modeling}
\section{Why Numerical Models? The Discretization Process}
\section{Finite Difference Methods}
\section{Finite Volume and Finite Element Methods}
\section{Grids: Arakawa C-Grid, Terrain-Following, and Isopycnal}
\section{Overview of Ocean General Circulation Models (OGCMs)}

\chapter{Data Analysis and Inverse Methods}
\section{Time Series Analysis: Autocorrelation, Spectral Analysis}
\section{Spatial Analysis: Objective Analysis, Empirical Orthogonal Functions (EOFs)}
\section{The Inverse Problem: Estimating Unobserved Quantities and Fluxes}
\section{Data Assimilation: Combining Models and Observations}

\chapter{Coupled Systems and Climate}
\section{Ocean-Atmosphere Coupling (e.g., El Niño-Southern Oscillation - ENSO)}
\section{Sea Ice Dynamics and Thermodynamics}
\section{The Ocean's Role in the Global Carbon Cycle}
\section{Climate Sensitivity and Ocean Feedbacks}

% Backmatter
\begin{appendix}
\chapter{Physical Constants and Properties of Seawater}
\chapter{Mathematical Formulas and Theorems}
\chapter{Standard Derivations}
\end{appendix}

\end{document}
