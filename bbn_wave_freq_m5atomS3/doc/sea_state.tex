\documentclass[11pt]{article}
\usepackage{amsmath, amssymb, geometry, graphicx}
\geometry{margin=1in}

\title{SeaStateRegularity: An Online Estimator of Ocean Wave Regularity from Vertical Acceleration}
\author{Mikhail Grushinskiy}
\date{2025}

\begin{document}
\maketitle

\begin{abstract}
We present \texttt{SeaStateRegularity}, an embedded-oriented algorithm for real-time estimation of ocean wave regularity from vertical acceleration signals. 
The method combines demodulation, spectral moment analysis, phase coherence tracking, and adaptive output shaping to provide indicators of wave narrowness, coherence, and approximate significant wave height. 
The approach is lightweight, robust to broadband and nonlinear waves, and well-suited for microcontroller deployment.
\end{abstract}

\section{Introduction}
Ocean waves exhibit varying degrees of spectral and phase regularity, which are relevant for navigation, energy harvesting, and environmental monitoring. 
Traditional spectral estimation methods (FFT, multitaper) are often too costly for embedded hardware. 
The proposed method uses recursive exponential averaging and analytic demodulation to extract regularity metrics from vertical acceleration in real time.

\section{Methodology}
The estimator is updated with each incoming acceleration measurement $a_z(t)$ and an instantaneous angular frequency estimate $\omega_{\text{inst}}(t)$. 
The core steps are described below.

\subsection{Demodulation of Vertical Acceleration}
Acceleration is demodulated into an approximate complex displacement:
\begin{align}
\phi(t) &= \int \omega_{\text{inst}}(t)\, dt, \\
z_{\text{real}}(t) &= a_z(t)\cos\phi(t), \\
z_{\text{imag}}(t) &= a_z(t)\sin\phi(t).
\end{align}
An exponential moving average (EMA) smooths $z_{\text{real}}$ and $z_{\text{imag}}$, with time constant $\tau_{\text{env}}$. 
This produces a narrowband analytic-like representation of displacement.

\subsection{Spectral Moment Estimation}
The displacement power is corrected by the squared frequency:
\begin{align}
x_{\text{disp}} &= \frac{z_{\text{real}} + i z_{\text{imag}}}{\omega^2}, \\
P_{\text{disp}} &= |x_{\text{disp}}|^2.
\end{align}
Spectral moments are recursively updated with EMA time constant $\tau_{\text{mom}}$:
\begin{align}
M_0 &\approx \langle P_{\text{disp}} \rangle, \\
M_1 &\approx \langle P_{\text{disp}} \cdot \omega \rangle, \\
M_2 &\approx \langle P_{\text{disp}} \cdot \omega^2 \rangle.
\end{align}

\subsection{Spectral Regularity}
A narrowness parameter $\nu$ is computed as
\begin{equation}
\nu = \sqrt{\frac{M_0 M_2}{M_1^2} - 1},
\end{equation}
which quantifies the effective bandwidth of the spectrum. 
The spectral regularity is then
\begin{equation}
R_{\text{spec}} = \exp(-\nu).
\end{equation}

\subsection{Phase Coherence Regularity}
Normalized complex displacement is used to track phase consistency:
\begin{align}
u &= \frac{z_{\text{real}} + i z_{\text{imag}}}{|z_{\text{real}} + i z_{\text{imag}}|}.
\end{align}
An EMA with time constant $\tau_{\text{coh}}$ updates its mean, yielding
\begin{equation}
R_{\text{phase}} = \left| \langle u \rangle \right|,
\end{equation}
which ranges from $0$ (incoherent phases) to $1$ (perfect coherence).

\subsection{Combined Regularity and Output Shaping}
A safe regularity is taken as
\[
R_{\text{safe}} = \max(R_{\text{spec}}, R_{\text{phase}}).
\]
The final output $R_{\text{out}}$ applies:
\begin{itemize}
    \item Reduction for broadband (moderately irregular) waves:
    \[
    R_{\text{target}} = R_{\text{safe}} - \Delta R_{\text{broadband}},
    \]
    with reduction proportional to $P_{\text{disp}}$ below a threshold.
    \item Boost for large, nonlinear waves:
    \[
    R_{\text{target}} = R_{\text{safe}} + \Delta R_{\text{boost}},
    \]
    when $P_{\text{disp}}$ exceeds a high threshold.
\end{itemize}
Finally, $R_{\text{out}}$ is smoothed with time constant $\tau_{\text{out}}$.

\subsection{Significant Wave Height Estimate}
An approximate significant wave height is computed as
\begin{equation}
H_s \approx 2\sqrt{M_0}\, f(R_{\text{out}}),
\end{equation}
where $f(R)$ is a height factor mapping regularity to amplitude correction. 
This accounts for underestimation in irregular seas.

\section{Implementation Notes}
\begin{itemize}
    \item All smoothing is exponential, ensuring $\mathcal{O}(1)$ updates per sample.
    \item Initialization uses NaN markers to allow seeding on the first sample.
    \item Frequency smoothing with $\tau_{\omega}$ reduces noise in $\omega_{\text{inst}}$.
    \item Broadband suppression and nonlinear boosting improve robustness for JONSWAP-like and steep wave conditions.
\end{itemize}

\section{Conclusion}
The \texttt{SeaStateRegularity} algorithm provides a real-time, embedded-friendly estimation of wave spectral and phase regularity, as well as an approximate wave height. Its recursive structure avoids heavy spectral transforms, making it suitable for low-power microcontrollers while retaining consistency with classical ocean wave theory.

\end{document}

