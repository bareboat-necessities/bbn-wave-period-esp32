\documentclass[11pt]{article}
\usepackage{amsmath,amssymb,amsthm}
\usepackage{siunitx}
\usepackage{graphicx}
\usepackage{hyperref}
\usepackage{geometry}
\usepackage{tikz}
\usepackage{booktabs}
\usepackage{cite}
\geometry{margin=1in}

\title{Gerstner (Trochoidal) Waves: \\
Exact Lagrangian Nonlinear Gravity Waves}
\author{}
\date{\today}

\begin{document}
\maketitle

\begin{abstract}
Gerstner waves, also called \emph{trochoidal waves}, are a remarkable exact nonlinear solution of the free-surface Euler equations for deep water. First derived by Franz Josef von Gerstner in 1802, they describe periodic progressive waves with circular particle trajectories and a free surface shaped as an inverted trochoid. Unlike irrotational Stokes waves, Gerstner waves are rotational, possessing finite depth-dependent vorticity. This article reviews their historical context, derivation from the Euler equations, Lagrangian description, velocity, acceleration, pressure, and energy properties. We discuss wave height, wavelength, speed, wave number, energy flux, rotational nature, nonlinear character, mechanical analogies, shallow-water limitations, and breaking criteria. While no longer used as realistic models of ocean waves due to their rotational structure, Gerstner waves remain important as one of the few exact nonlinear solutions of hydrodynamics, and they continue to be used in mathematical fluid dynamics, pedagogy, and computer graphics.
\end{abstract}

\tableofcontents
\bigskip

\section{Historical Context}
The history of water wave theory is rich, beginning with early attempts by Newton and Bernoulli to describe wave motion. The first exact nonlinear solution of the full hydrodynamic equations for surface waves was published by Franz Josef von Gerstner in 1802 \cite{gerstner1802}. This solution, later rediscovered by Rankine in the 19th century, describes a progressive periodic wave whose free surface is a \emph{trochoid}, the path traced by a point on a circle rolling along a straight line.

The Gerstner solution was groundbreaking because it was fully nonlinear and exact, in contrast to linear (Airy) theory and later perturbation methods (Stokes expansions). However, it introduced a property that limited its acceptance in oceanography: the flow is \emph{rotational}. Real ocean surface waves are largely irrotational, except near boundaries or in breaking events. Consequently, Gerstner waves did not become the foundation of engineering wave theory. Instead, irrotational Stokes expansions and spectral theories (Pierson–Moskowitz, JONSWAP, etc.) became standard.

Despite this, Gerstner waves remain important in three domains:
\begin{itemize}
    \item As a classical exact solution of the Euler equations, useful for mathematical analysis of nonlinear hydrodynamics \cite{craik2004origins,constantin2001gerstner}.
    \item In pedagogy, to illustrate Lagrangian versus Eulerian descriptions, trochoidal free surfaces, and rotational effects.
    \item In computer graphics, where superpositions of Gerstner-type trochoidal waves are used for realistic ocean surface synthesis \cite{tessendorf2001simulating}.
\end{itemize}

\section{Governing Equations}
The fluid is modeled as incompressible and (for Gerstner's solution) inviscid. Start with Navier--Stokes and then take the Euler limit.

\subsection{Navier--Stokes}
\begin{align}
\rho \left( \frac{\partial \mathbf{u}}{\partial t} + \mathbf{u}\cdot\nabla\mathbf{u} \right) 
 &= - \nabla p + \rho \nu \Delta \mathbf{u} - \rho g \hat{\mathbf{y}}, \\
\nabla\cdot \mathbf{u} &= 0,
\end{align}
where $\mathbf{u}=(u,v)$, $p$ is pressure, $\rho$ density, $\nu$ viscosity, $g$ gravity.

\subsection{Euler limit}
Setting $\nu=0$ gives the incompressible Euler equations
\begin{equation}
\rho \left( \partial_t \mathbf{u} + \mathbf{u}\cdot\nabla\mathbf{u} \right) = - \nabla p - \rho g \hat{\mathbf{y}}, 
\qquad \nabla \cdot \mathbf{u} = 0.
\end{equation}

Boundary conditions:
\begin{itemize}
\item At free surface $y=\eta(x,t)$: kinematic condition (particles remain on surface) and dynamic condition ($p=p_{\mathrm{atm}}$).
\item At $y\to -\infty$: $\mathbf{u}\to 0$.
\end{itemize}

\section{Lagrangian Formulation of Gerstner Waves}
Gerstner’s solution is naturally expressed in Lagrangian coordinates $(a,b)$, where $(a,b)$ labels a fluid particle. The Lagrangian map $(a,b)\mapsto (X,Y)$ is
\begin{align}
X(a,b,t) &= a + \frac{e^{k b}}{k}\sin\big(k(a+ct)\big), \label{eq:gerstnerX}\\
Y(a,b,t) &= b - \frac{e^{k b}}{k}\cos\big(k(a+ct)\big), \label{eq:gerstnerY}
\end{align}
with $k>0$ the wave number and $c$ the phase speed. (Conventions on signs and phase may vary in the literature; this form represents a wave traveling in the $-a$ direction if one changes sign in the phase.)

\subsection{Free surface and trochoid}
Choose a fixed Lagrangian level $b=b_s$ for the free surface. The parametric free-surface in Eulerian coordinates is
\begin{align}
x_s(a,t) &= a + \frac{e^{kb_s}}{k}\sin(k(a+ct)), \\
y_s(a,t) &= b_s - \frac{e^{kb_s}}{k}\cos(k(a+ct)).
\end{align}
This is an inverted trochoid. Define the surface amplitude
\[
a_0 \;=\; \frac{e^{kb_s}}{k},
\]
so total wave height $H=2a_0$. The limiting crest (cusp) occurs when $b_s\to 0$ ($a_0\to 1/k$).

\subsection{Dispersion relation (deep water)}
Imposing the dynamic free-surface condition (constant atmospheric pressure) yields the dispersion relation (same as linear deep-water waves):
\begin{equation}
c^2 = \frac{g}{k}. \label{eq:dispersion}
\end{equation}

\section{Velocity and acceleration}
Differentiate the Lagrangian map w.r.t.\ time for fixed labels $(a,b)$.

\subsection{Velocities}
Let $\theta=k(a+ct)$ so $\partial_t\theta = kc$. Then
\begin{align}
\dot X(a,b,t) &= c\,e^{k b}\cos\theta, \label{eq:velX}\\
\dot Y(a,b,t) &= c\,e^{k b}\sin\theta. \label{eq:velY}
\end{align}
Particles describe circular orbits of radius $e^{kb}/k$ and angular frequency $kc$, with orbital speed decaying exponentially with depth.

\subsection{Accelerations}
Differentiate again:
\begin{align}
\ddot X &= -k c^2 e^{k b}\sin\theta, \\
\ddot Y &= \;\;k c^2 e^{k b}\cos\theta.
\end{align}
Thus the material acceleration vector for a particle is centripetal with magnitude $k c^2 e^{k b}$.

\section{Pressure field}
Insert $\mathbf{a}=\ddot{\mathbf{X}}$ into Euler's equation in Lagrangian description to integrate for pressure. One convenient form is
\begin{equation}\label{eq:pressure}
p(a,b,t) = p_{\mathrm{atm}} + \rho g\big(Y(a,b,t)-y_0\big) - \frac{1}{2}\rho c^2\big(e^{2k b}-e^{2k b_s}\big),
\end{equation}
where $y_0$ is an arbitrary reference level. Evaluated at the free surface $b=b_s$ this gives $p=p_{\mathrm{atm}}$.

\section{Vorticity and rotational nature}
Compute vorticity $\omega=\partial_x v - \partial_y u$. Using the Lagrangian mapping and chain rule one obtains a depth-dependent nonzero vorticity (many equivalent algebraic forms exist). Its nonzero value is the core reason the Gerstner solution is rotational. Physically, this vorticity distribution enforces closed circular particle orbits (i.e., cancels Stokes drift).

\section{Wave geometry and limiting steepness}
Amplitude and derived quantities:
\[
a_0=\frac{e^{kb_s}}{k},\qquad H=2a_0,\qquad \lambda=\frac{2\pi}{k},\qquad T=\frac{2\pi}{kc}.
\]
Steepness $H/\lambda = (k a_0)/\pi$. Breaking (crest cusp) occurs in the limiting case $k a_0=1$, giving
\[
\left.\frac{H}{\lambda}\right|_{\text{cusp}} = \frac{1}{\pi}\approx 0.318.
\]
For comparison, the limiting irrotational Stokes wave steepness is $\approx 0.141$.

\section{Energy and energy flux (remarks)}
For linear deep-water waves the mean energy per unit horizontal area is
\[
\overline{E} = \frac{1}{8}\rho g H^2,
\]
and the energy flux $F=\overline{E}\,c_g$ with group velocity $c_g=c/2$. For Gerstner waves the dispersion relation is identical to linear deep-water waves (Eq.~\ref{eq:dispersion}), so many leading-order energetic scalings coincide; however, rotation alters kinetic/potential partition and care is required for exact energy integrals.

\section{Shallow-depth corrections and limitations}
Gerstner’s classical closed-form solution assumes infinite (deep) water. Finite depth requires different formulations; simple exponential depth decay $e^{kb}$ is replaced by hyperbolic functions ($\sinh, \cosh$) in linear theory, and Gerstner-like exact rotational solutions for finite depth are more involved or don't exist in the same simple form. Thus Gerstner waves are primarily a deep-water analytical tool.

\section{Why Gerstner waves are nonlinear and rotational}
\begin{itemize}
\item \textbf{Nonlinearity:} Gerstner’s mapping is not a small-amplitude expansion — it is exact and finite amplitude; the free surface is trochoidal rather than sinusoidal.
\item \textbf{Rotation:} The vorticity field is nonzero and depth dependent. This vorticity is precisely what allows closed orbits (no net Lagrangian drift) at finite amplitude. In contrast, irrotational Stokes waves exhibit Stokes drift and require a potential function.
\end{itemize}

\section{Mechanical analogy}
Trochoids arise from a point on a rolling circle. A mechanical picture of concentric "rings of vorticity" producing circular orbits is a helpful heuristic: the internal vorticity distribution provides centripetal accelerations that are balanced by pressure gradients and gravity.

\section{Acceleration magnitudes and breaking}
The acceleration amplitude at depth $b$ is $k c^2 e^{k b}$. At the surface:
\[
|\mathbf{a}|_{\mathrm{surf}} = k c^2 e^{k b_s} = g k a_0 = g\frac{\pi H}{\lambda}.
\]
For modest wave steepness this is a fraction of $g$; for very steep trochoidal waves it can approach $g$ and lead to breaking and cusp formation.

\section{Where Gerstner waves are used today}
\begin{itemize}
\item \textbf{Mathematical fluid dynamics}: as a rare exact nonlinear solution used to test uniqueness, existence, and qualitative theory.
\item \textbf{Pedagogy}: illustrating Lagrangian flows, vorticity, and exact solutions.
\item \textbf{Computer graphics}: phenomenological Gerstner-type sums are used widely for realistic ocean rendering because the trochoidal profiles give sharp crests; these graphics versions are not dynamical solutions but are visually useful.
\end{itemize}

\section{Summary}
Gerstner (trochoidal) waves provide an elegant exact nonlinear solution of the free-surface Euler equations in deep water. Their trochoidal surface and circular particle orbits make them mathematically attractive. Their rotationality and amplitude-independent phase speed make them a poor model for naturally occurring wind-generated, irrotational ocean waves, but they remain valuable in theory and graphics.

\section*{Figures}
\begin{figure}[h!]
\centering
\begin{tikzpicture}[scale=1]
  % axes
  \draw[->] (-0.5,0) -- (7,0) node[right] {$x$};
  \draw[->] (0,-2.5) -- (0,1.5) node[above] {$y$};
  % trochoid via plot (samples)
  \draw[thick,domain=0:6.283,smooth,samples=240] plot (\x,{ -0.8*cos(\x r) }) ;
  \draw[dashed] (0,-0.8) -- (6.283,-0.8) node[right]{mean level};
  \node at (1,0.9) {Free surface (trochoid)};
\end{tikzpicture}
\caption{Schematic trochoidal free surface (parametric form).}
\end{figure}

\begin{figure}[h!]
\centering
\begin{tikzpicture}[scale=1]
  \draw[->] (-0.5,0) -- (5,0) node[right] {$x$};
  \draw[->] (0,-3) -- (0,1.5) node[above] {$y$};
  % draw sample particle circles with decaying radii
  \foreach \i/\r in {0/-0.2,1/-0.8,2/-1.4,3/-2.0}{
    \draw[dashed] (2,\r) circle ({0.5*exp(\r)});
  }
  \node at (3.2,1.0) {Particle orbits (decaying radius with depth)};
\end{tikzpicture}
\caption{Schematic circular particle orbits decreasing exponentially with depth.}
\end{figure}

\clearpage
\bibliographystyle{plain}
\begin{thebibliography}{99}

\bibitem{gerstner1802}
F.~J.~Gerstner. \emph{Theorie der Wellen}. Abhandlungen der K. B\"ohmischen Gesellschaft der Wissenschaften, Prague, 1802.

\bibitem{craik2004origins}
A.~D.~D. Craik. The origins of water wave theory. \emph{Annual Review of Fluid Mechanics}, 36:1--28, 2004.

\bibitem{constantin2001gerstner}
A.~Constantin. On the deep water wave motion. \emph{Journal of Fluid Mechanics}, 409:285--299, 2001.

\bibitem{henry2008}
D.~Henry. On Gerstner’s water wave. \emph{Journal of Nonlinear Mathematical Physics}, 15(suppl. 2):87--95, 2008.

\bibitem{mei1989applied}
C.~C. Mei. \emph{The Applied Dynamics of Ocean Surface Waves}. World Scientific, 1989.

\bibitem{tessendorf2001simulating}
J.~Tessendorf. Simulating ocean water. In \emph{Proceedings of the ACM SIGGRAPH Course Notes}, 2001.

\end{thebibliography}

\end{document}
