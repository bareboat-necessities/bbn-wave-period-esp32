\documentclass{article}
\usepackage{amsmath}
\usepackage{amssymb}
\usepackage{mathrsfs}
\usepackage{graphicx}

\title{Nonlinear Periodic Wave Computation via Fenton’s Spectral Method}
\author{}
\date{}

\begin{document}

\maketitle

\section{Nonlinear Periodic Fenton Wave}
\label{sec:fenton_method}

\noindent
We summarize here the mathematical formulation of Fenton’s method for steady periodic surface gravity waves in finite depth \cite{Fenton1988}, followed by a brief description of its \verb|C++| implementation (see \texttt{FentonWave.h}).

\subsection{Governing Equations}

Let $(x,z)$ denote the horizontal and vertical coordinates, with the seabed at $z=-d$ and free surface at $z=\eta(x)$. Under the assumptions of irrotational, incompressible, inviscid flow, there exists a stream function $\psi(x,z)$ satisfying Laplace’s equation,
\begin{equation}
\nabla^2 \psi = \psi_{xx} + \psi_{zz} = 0\,, \quad -d \le z \le \eta(x)\,.
\end{equation}
At the free surface $z=\eta(x)$, the dynamic (Bernoulli) and kinematic boundary conditions are
\begin{align}
\underbrace{\tfrac12\bigl(\psi_x^2 + \psi_z^2\bigr) + g\,\eta}_{\text{total head}}
  &= R + c\,\psi_x\,, 
  \label{eq:bernoulli}\\
\psi_z &= \eta_x\,\psi_x\,,
  \label{eq:kinematic}
\end{align}
where $g$ is gravity, $c$ the phase speed, and $R$ a constant (Bernoulli constant). At the bed $z=-d$, no‐flow implies
\begin{equation}
\psi(x,-d)=0\,.
\end{equation}

\subsection{Spectral Representation}

We expand the stream function and surface elevation in truncated Fourier series up to order $N$:
\begin{align}
\psi(x,z) &= B_0\,(z + d)
   \;+\;\sum_{j=1}^{N} B_j\;\frac{\sinh\!\bigl[j\,k\,(z + d)\bigr]}{\cosh(j\,k\,d)}
   \,\cos(j\,k\,x)\,,
   \label{eq:psi_series}\\
\eta(x) &= \sum_{j=0}^{N} E_j\;\cos(j\,k\,x)\,,
   \label{eq:eta_series}
\end{align}
where
\[
k = \frac{2\pi}{L}, 
\quad L = \text{wavelength}, 
\quad \{B_j\},\{E_j\},c,R
\]
are unknowns to be solved for. The coefficients $E_j$ are obtained from the collocation‐space values $\eta(x_i)$ via a DCT‐I transform:
\[
E = \mathrm{DCT\text{-}I}\bigl[\eta(x_i)\bigr], 
\quad x_i = \frac{i\pi}{N},\; i=0,\dots,N.
\]

\subsection{Discrete Nonlinear System}

Enforce \eqref{eq:bernoulli} and \eqref{eq:kinematic} at the $N+1$ collocation points $x_i$, plus two global constraints:
\begin{itemize}
  \item Mean elevation constraint: \(\displaystyle \frac{1}{N}\sum_{i=0}^N w_i\,\eta(x_i)=0,\quad w_0=w_N=\tfrac12,\;w_i=1\,.\)
  \item Prescribed wave height: \(\max_i\eta(x_i)-\min_i\eta(x_i)=H.\)
\end{itemize}
Let
\[
\mathbf{X} = \begin{bmatrix}
B_0,\dots,B_N,\;
\eta(x_0),\dots,\eta(x_N),\;
Q,\;R
\end{bmatrix}^T,
\]
where $Q$ is the volume flux.  We form the residual vector
\[
\mathbf{F}(\mathbf{X}) = 
\begin{bmatrix}
\text{(Bernoulli at each }x_i)\\
\text{(Kinematic at each }x_i)\\
\text{(mean elevation)}\\
\text{(wave height)}
\end{bmatrix},
\]
and solve 
\[
\mathbf{F}(\mathbf{X}) = \mathbf{0}
\]
by Newton–Raphson:
\begin{equation}
\mathbf{X}^{(n+1)}
= \mathbf{X}^{(n)} 
- \alpha
\;\mathbf{J}^{-1}\!\bigl(\mathbf{X}^{(n)}\bigr)
\;\mathbf{F}\bigl(\mathbf{X}^{(n)}\bigr),
\quad 0<\alpha\le1,
\end{equation}
where $\mathbf{J} = \partial\mathbf{F}/\partial\mathbf{X}$ is the Jacobian (analytically constructed in the code), and $\alpha$ is a relaxation parameter (default 0.5).

\subsection{Jacobian Matrix}
\label{sec:jacobian}

Define the unknown vector
\[
\mathbf{X} = \bigl[B_{0}, B_{1},\dots,B_{N},\;\eta_{0},\eta_{1},\dots,\eta_{N},\;Q,\;R\bigr]^{T},
\]
and the residual vector
\[
\mathbf{F}(\mathbf{X}) =
\begin{bmatrix}
f_{0}\\
\vdots\\
f_{N}\\[0.5ex]
f_{N+1}\\
\vdots\\
f_{2N+1}\\[0.5ex]
f_{2N+2}\\
f_{2N+3}
\end{bmatrix},
\]
where, for each collocation point \(x_{m} = \tfrac{m\pi}{N}\) and surface elevation \(\eta_{m} = \eta(x_{m})\),
\begin{align}
f_{m} &= -\,B_{0}\,\eta_{m}
       + \sum_{j=1}^{N} B_{j}\,\frac{\sinh(k_{j}\,\eta_{m})}{\cosh(k_{j}\,d)}\,\cos(jk\,x_{m})
       + Q,
       \quad m=0,\dots,N,
       \\
f_{N+1+m} &= \tfrac12\bigl(u_{m}^{2}+v_{m}^{2}\bigr)
            + \eta_{m} - R,
       \quad m=0,\dots,N,
\end{align}
with
\begin{align}
k_{j} = j\,k, \\[0.5ex]
u_{m} = -B_{0} + \sum_{j=1}^{N} k_{j}B_{j}\,\frac{\cosh[k_{j}(z=\eta_{m}+d)]}{\cosh(k_{j}d)}\cos(jk\,x_{m}),
 \\[0.5ex]
v_{m} = \sum_{j=1}^{N} k_{j}B_{j}\,\frac{\sinh[k_{j}(z=\eta_{m}+d)]}{\cosh(k_{j}d)}\sin(jk\,x_{m}). \\[0.5ex]
\end{align}
The two global constraints are
\begin{align}
f_{2N+2} &= \frac{1}{N}\sum_{i=0}^{N} w_i\,\eta_{i} - 1,
\quad w_{0}=w_{N}=\tfrac12,\;w_{i}=1\;(1\le i\le N-1),
\\
f_{2N+3} &= \max_{0\le i\le N}\!\eta_{i}\;-\;\min_{0\le i\le N}\!\eta_{i}\;-\;H.
\end{align}

The Jacobian matrix \(\mathbf{J} = [J_{pq}]=\partial F_{p}/\partial X_{q}\) has the following nonzero blocks:

\paragraph{Bernoulli rows \(p=m\), \(0\le m\le N\):}
\begin{align}
\frac{\partial f_{m}}{\partial B_{0}}
  &= -\eta_{m},
  \\[0.5ex]
\frac{\partial f_{m}}{\partial B_{j}}
  &= \frac{\sinh(k_{j}\,\eta_{m})}{\cosh(k_{j}d)}\cos(jk\,x_{m}),\; j\ge1,
  \\
\frac{\partial f_{m}}{\partial \eta_{m}}
  &= -B_{0}
     + \sum_{j=1}^{N}B_{j}\,k_{j}\,\frac{\cosh(k_{j}\,\eta_{m})}{\cosh(k_{j}d)}\cos(jk\,x_{m}),
  &
\frac{\partial f_{m}}{\partial Q}
  &= 1.
\end{align}

\paragraph{Dynamic rows \(p=N+1+m\), \(0\le m\le N\):}
\begin{align}
\frac{\partial f_{N+1+m}}{\partial B_{0}}
  &= -\,u_{m},
  \\[0.5ex]
\frac{\partial f_{N+1+m}}{\partial B_{j}}
  &= k_{j}\,\bigl(u_{m}\,C_{jm} + v_{m}\,S_{jm}\bigr),
  \\
\frac{\partial f_{N+1+m}}{\partial \eta_{m}}
  &= 1
     + \sum_{j=1}^{N}
       B_{j}\,k_{j}^{2}
       \Bigl(u_{m}\,\frac{\partial S_{jm}}{\partial \eta_{m}}
           + v_{m}\,\frac{\partial C_{jm}}{\partial \eta_{m}}\Bigr),
  &
\frac{\partial f_{N+1+m}}{\partial R}
  &= -1,
\end{align}
where
\[
C_{jm} = \frac{\cosh[k_{j}(\eta_{m}+d)]}{\cosh(k_{j}d)}\cos(jk\,x_{m}), 
\quad
S_{jm} = \frac{\sinh[k_{j}(\eta_{m}+d)]}{\cosh(k_{j}d)}\sin(jk\,x_{m}),
\]
and
\(\partial S_{jm}/\partial \eta_{m} = k_{j} \,C_{jm}\), 
\(\partial C_{jm}/\partial \eta_{m} = k_{j} \,S_{jm}.\)

\paragraph{Mean‐elevation row \(p=2N+2\):}
\[
\frac{\partial f_{2N+2}}{\partial \eta_{i}}
  = \frac{w_i}{N},\quad i=0,\dots,N.
\]

\paragraph{Wave‐height row \(p=2N+3\):}
\[
\frac{\partial f_{2N+3}}{\partial \eta_{i_{\max}}} = 1,
\quad
\frac{\partial f_{2N+3}}{\partial \eta_{i_{\min}}} = -1,
\]
where \(i_{\max},i_{\min}\) are the indices of the maximum and minimum \(\eta_{i}\).

\medskip
All other partial derivatives are zero.  In the C++ code (\texttt{compute\_jacobian()}), these expressions are implemented exactly—ensuring an analytically exact Jacobian for robust Newton–Raphson convergence.

\subsection{Derived Wave Quantities}
\label{sec:derived_quantities}

Once the spectral solution \(\{B_j\},\{E_j\},c,R\) is obtained, the implementation computes the following physical fields and integral measures:

\paragraph{Surface kinematics:}
\begin{align}
\eta(x,t) &= \sum_{j=0}^{N} E_{j}\,\cos\bigl[jk(x - ct)\bigr], \\[0.5ex]
\eta_{x}(x,t) &= -\sum_{j=0}^{N} j\,k\,E_{j}\,\sin\bigl[jk(x - ct)\bigr], \\[0.5ex]
\eta_{t}(x,t) &= -\,c\,\eta_{x}(x,t), \\[0.5ex]
\eta_{xx}(x,t) &= -\sum_{j=0}^{N} j^{2}k^{2}\,E_{j}\,\cos\bigl[jk(x - ct)\bigr], \\[0.5ex]
\eta_{xt}(x,t) &= \sum_{j=0}^{N} j^{2}k\,c\,E_{j}\,\sin\bigl[jk(x - ct)\bigr], \\[0.5ex]
\eta_{tt}(x,t) &= -\sum_{j=0}^{N} j^{2}\omega^{2}\,E_{j}\,\cos\bigl[jk(x - ct)\bigr],
\end{align}
where \(\omega = c\,k\).

\paragraph{Stream function and velocities:}
\begin{align}
\psi(x,z,t) 
&= B_{0}\,(z + d) 
   + \sum_{j=1}^{N} B_{j}\,
     \frac{\sinh\!\bigl[k_{j}(z + d)\bigr]}{\cosh(k_{j}d)}\,
     \cos\bigl[jk(x - ct)\bigr], 
   \\[0.5ex]
u(x,z,t) &= \frac{\partial \psi}{\partial z}, 
\quad
w(x,z,t) = -\,\frac{\partial \psi}{\partial x},
\end{align}
with \(k_{j}=j\,k\).

\paragraph{Dynamic pressure:}
\begin{equation}
p(x,z,t)
= \rho\Bigl[R - \tfrac12\bigl(u^{2}+w^{2}\bigr) - g\,(z - \eta(x,t)) + c\,u\Bigr].
\end{equation}

\paragraph{Energy densities and flux:}  
Using the 2D trapezoidal integrator \(\int_{0}^{L}\!\int_{-d}^{\eta}\!\cdot\,dz\,dx\):
\begin{align}
\overline{E}_{k} 
&= \frac{1}{L}\int_{0}^{L}\!\int_{-d}^{\eta(x)} \tfrac12\bigl(u^{2}+w^{2}\bigr)\,dz\,dx, 
\\
\overline{E}_{p} 
&= \frac{g}{L}\int_{0}^{L} \eta(x)\,dx, 
\\
E_{\rm tot} 
&= \overline{E}_{k} + \overline{E}_{p}, 
\\
F_{E} 
&= c\,E_{\rm tot}.
\end{align}

\paragraph{Mean currents and drift:}
\begin{align}
\overline{u}_{E} 
&= \frac{1}{d\,L}\int_{0}^{L}\!\int_{-d}^{\eta(x)} u\,dz\,dx, 
\\
U_{s}\;({\rm Stokes\ drift})
&= \frac{1}{L}\int_{0}^{L} u\bigl(x,\eta(x)\bigr)\,dx.
\end{align}

\paragraph{Momentum and radiation stress:}
\begin{align}
I\;({\rm wave\ impulse})
&= \int_{0}^{L}\!\int_{-d}^{\eta(x)} u\,dz\,dx, 
\\
M\;({\rm momentum\ flux})
&= \int_{0}^{L}\!\int_{-d}^{\eta(x)} u^{2}\,dz\,dx, 
\\
S_{xx}
&= \rho\,\bigl\langle u^{2}\bigr\rangle
   = \rho\,\frac{M}{L}.
\end{align}

\subsection{Implementation Notes}

The provided \verb|FentonWave<N,Real>| class in \texttt{FentonWave.h}:
\begin{itemize}
  \item Uses the Eigen library for vectors and matrices.
  \item Computes DCT‐I transforms in \verb|FentonFFT<N>::compute_inverse_cosine_transform()|.
  \item Handles hyperbolic functions with numerically stable helpers \verb|sinh_by_cosh| and \verb|cosh_by_cosh|.
  \item Solves the dispersion relation for $\lambda$ via Newton iteration in \verb|compute_wavelength()|.
  \item Wraps up all physics (surface kinematics, velocities, pressure, energy flux, Stokes drift, etc.) in member functions.
  \item Includes a \verb|WaveSurfaceTracker| that integrates a floating object’s horizontal motion with RK4 under wave‐induced force and linear drag.
\end{itemize}

\subsection{Floating‐Object Kinematics: \texttt{WaveSurfaceTracker}}
\label{sec:wave_surface_tracker}

In addition to wave kinematics, the C++ implementation provides a \texttt{WaveSurfaceTracker<N,Real>} class that simulates the motion of a small floating body constrained to the wave surface.  The model assumes:

\begin{itemize}
  \item The body’s vertical coordinate exactly follows the instantaneous free‐surface elevation:
    \[
      z(t) = \eta\bigl(x(t),\,t\bigr).
    \]
  \item The horizontal motion is driven by a “slope” force proportional to surface gradient, plus linear drag.
\end{itemize}

\paragraph{Horizontal dynamics.}  
Let \(x(t)\) be the body’s horizontal position and \(v_x = \dot x\) its horizontal velocity.  Newton’s second law yields
\begin{equation}
  m\,\ddot x \;=\; F_{\rm wave} + F_{\rm drag}
  \;=\;-m\,g\,\eta_x\bigl(x(t),t\bigr)
    \;-\;c_{\!d}\,v_x,
  \label{eq:tracker_motion}
\end{equation}
where
\[
  \eta_x = \frac{\partial \eta}{\partial x}, 
  \quad
  c_{\!d} = \text{drag coefficient}, 
  \quad
  m = \text{body mass}.
\]

\paragraph{Vertical velocity and acceleration.}  
Although \(z=\eta\), one can still compute the vertical velocity and acceleration for completeness:
\begin{align}
  \dot z &= \eta_t + \eta_x\,\dot x,
  \\
  \ddot z &\approx \frac{\dot z(t+\Delta t)-\dot z(t)}{\Delta t},
\end{align}
with \(\eta_t = \partial\eta/\partial t = -\,c\,\eta_x\).

\paragraph{Numerical integration.}  
The class uses a classical 4th‐order Runge–Kutta scheme (RK4) with time step \(\Delta t\), updating both \(x\) and \(v_x\) simultaneously:
\[
\begin{aligned}
  k_{1}^{(x)} &= v_x, 
  &\quad
  k_{1}^{(v)} &= \ddot x\bigl(x,\,v_x,\,t\bigr),\\
  k_{2}^{(x)} &= v_x + \tfrac{\Delta t}{2}\,k_{1}^{(v)}, 
  &\quad
  k_{2}^{(v)} &= \ddot x\Bigl(x + \tfrac{\Delta t}{2}k_{1}^{(x)},\;v_x + \tfrac{\Delta t}{2}k_{1}^{(v)},\;t+\tfrac{\Delta t}{2}\Bigr),\\
  k_{3}^{(x)} &= v_x + \tfrac{\Delta t}{2}\,k_{2}^{(v)}, 
  &\quad
  k_{3}^{(v)} &= \ddot x\Bigl(x + \tfrac{\Delta t}{2}k_{2}^{(x)},\;v_x + \tfrac{\Delta t}{2}k_{2}^{(v)},\;t+\tfrac{\Delta t}{2}\Bigr),\\
  k_{4}^{(x)} &= v_x + \Delta t\,k_{3}^{(v)}, 
  &\quad
  k_{4}^{(v)} &= \ddot x\bigl(x + \Delta t\,k_{3}^{(x)},\;v_x + \Delta t\,k_{3}^{(v)},\;t+\Delta t\bigr),
\end{aligned}
\]
\[
\begin{aligned}
  x(t+\Delta t) &= x + \tfrac{\Delta t}{6}\bigl(k_{1}^{(x)} + 2k_{2}^{(x)} + 2k_{3}^{(x)} + k_{4}^{(x)}\bigr),\\
  v_x(t+\Delta t) &= v_x + \tfrac{\Delta t}{6}\bigl(k_{1}^{(v)} + 2k_{2}^{(v)} + 2k_{3}^{(v)} + k_{4}^{(v)}\bigr).
\end{aligned}
\]

\paragraph{Implementation remarks.}
\begin{itemize}
  \item The method is implemented in \texttt{WaveSurfaceTracker::rk4\_step()}.
  \item The “wrap” function ensures periodicity in \(x\) by mapping \(x\) back into \([0,L)\).
  \item At each step, a user‐supplied callback can record \((t,\,z,\,\dot z,\,\ddot z,\,x,\,v_x)\).
  \item Default parameters: drag coefficient \(c_d\), mass \(m\), and time step \(\Delta t\) are configurable.
\end{itemize}

\begin{thebibliography}{1}
\bibitem{Fenton1988}
J.~D.~Fenton, “The numerical solution of steady water wave problems,” \emph{Computers \& Geosciences}, vol.~14, no.~3, pp.~357–368, 1988.
\end{thebibliography}

\end{document}

