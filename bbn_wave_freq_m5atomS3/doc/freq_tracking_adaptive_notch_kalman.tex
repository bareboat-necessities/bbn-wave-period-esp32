\documentclass{article}
\usepackage{amsmath}
\usepackage{amssymb}
\usepackage{mathrsfs}
\usepackage{graphicx}

\title{Comprehensive Analysis of a Kalman-Based Adaptive Notch Filter Frequency Tracker}
\author{}
\date{}

\begin{document}

\maketitle

\section{Theoretical Foundations}

\subsection{Original Reference}
This algorithm is based on the work of  
Ali and van Waterschoot~\cite{AliWaterschoot2023}, which marries a single-parameter adaptive notch filter with a Kalman-filter update to track the instantaneous frequency of a sinusoid in noise.

\subsection{Signal and Notch-Filter Model}
We assume the measurement
\[
y(t) = A(t)\,\sin\bigl(\omega(t)\,t + \phi(t)\bigr) + \nu(t),
\]
with slowly varying amplitude \(A(t)\), phase \(\phi(t)\), and instantaneous frequency \(\omega(t)\).  The core is a second-order digital notch filter with transfer function
\[
H(z^{-1};a) \;=\; \frac{1 - a\,z^{-1} + z^{-2}}{1 - \rho\,a\,z^{-1} + \rho^2\,z^{-2}},
\]
where \(a = 2\cos(\omega\,\Delta t)\) and \(\rho\in(0,1)\) is the pole radius.  By adapting \(a\) via a Kalman update, the instantaneous frequency is recovered as
\[
\omega = \frac{1}{\Delta t}\arccos\!\Bigl(\frac{a}{2}\Bigr).
\]

\section{Resonator Structure and Properties}

\subsection{Difference Equation and Order}

The KalmanANF algorithm is built around a second-order resonator, implemented using the following recursive difference equation:

\begin{equation}
s[n] = y[n] + \rho\, a[n-1]\, s[n-1] - \rho^2 s[n-2]
\label{eq:resonator}
\end{equation}

Here:
\begin{itemize}
    \item $s[n]$ is the internal state of the resonator (filtered signal),
    \item $y[n]$ is the input sample,
    \item $\rho$ is the pole radius ($0 < \rho < 1$), controlling bandwidth,
    \item $a[n-1]$ is the adaptive coefficient (related to cosine of frequency),
    \item $s[n-1]$ and $s[n-2]$ are previous internal states.
\end{itemize}

This equation represents a second-order **recursive IIR filter**, specifically a bandpass resonator tuned adaptively by the coefficient $a[n]$.

\subsection{Oscillator Model and Filter Interpretation}

The resonator structure in Equation~\eqref{eq:resonator} is mathematically equivalent to a second-order damped harmonic oscillator driven by input $y[n]$.

Its **transfer function** in the Z-domain is:

\begin{equation}
H(z) = \frac{1}{1 - \rho a z^{-1} + \rho^2 z^{-2}}
\label{eq:resonator_tf}
\end{equation}

This structure places a pair of complex-conjugate poles at radius $\rho$ and angle $\omega$, where:

\[
a = 2\cos(\omega)
\quad \Rightarrow \quad
\omega = \cos^{-1}\left(\frac{a}{2}\right)
\]

The poles define the resonant frequency and bandwidth:
\begin{itemize}
    \item Resonant frequency: $f_0 = \frac{\omega}{2\pi\Delta t}$
    \item 3dB Bandwidth: Approx. $\frac{1 - \rho}{\pi \Delta t}$ (narrower for $\rho \to 1$)
\end{itemize}

\subsection{Filter Order and Properties}

This resonator is of **second order**, with the following key properties:

\begin{itemize}
    \item \textbf{Linear time-varying (LTV)}: The filter structure is IIR, but the coefficient $a[n]$ changes with time via Kalman update.
    \item \textbf{Bandpass behavior}: The filter amplifies signal components near the estimated frequency.
    \item \textbf{Pole movement}: As the frequency estimate evolves, the pole angle shifts accordingly, dynamically tuning the resonance.
    \item \textbf{Amplitude normalization}: The gain is not normalized, but frequency estimate is extracted from $a[n]$.
    \item \textbf{Stability}: Guaranteed if $\rho < 1$ and $|a[n]| < 2$, which are explicitly enforced in implementation.
\end{itemize}

\subsection{Role in Frequency Estimation}

The resonator acts as a selective amplifier for the dominant sinusoidal component. The Kalman loop continuously adjusts the coefficient $a[n]$ to minimize the prediction error:

\begin{equation}
e[n] = s[n] - a[n-1] s[n-1] + s[n-2]
\end{equation}

This error drives the Kalman update, refining $a[n]$ such that the poles align with the signal's dominant frequency.

\subsection{Energy Interpretation}

The energy of the resonator states $s[n-1], s[n-2]$ implicitly reflects the amplitude of the tracked component. While the KalmanANF does not estimate amplitude explicitly, the phase estimate:

\begin{equation}
\phi[n] = \mathrm{atan2}(s[n-1],\, s[n-2])
\end{equation}

is derived from this oscillatory state trajectory, which traces an elliptical or circular path depending on amplitude consistency.

\section{Signal Interpretation and Limitations}

\subsection{Hilbert Transform Interpretation}

Although the KalmanANF does not compute the Hilbert transform explicitly, its resonator acts similarly to an analytic signal generator.

Recall the Hilbert transform creates an analytic signal:

\[
y_a(t) = y(t) + j\,\mathcal{H}\{y(t)\} = A(t)\,e^{j(\omega t + \phi(t))}
\]

From this form, the **instantaneous frequency** is computed as:

\[
\omega(t) = \frac{d}{dt} \arg\bigl(y_a(t)\bigr)
\]

In KalmanANF:
\begin{itemize}
  \item The states \(s[n-1], s[n-2]\) define an internal 2D oscillator approximating the in-phase and quadrature components of \(y(t)\).
  \item The phase estimate \(\phi[n] = \mathrm{atan2}(s[n-1], s[n-2])\) mimics the analytic phase.
  \item The estimated frequency \(\hat{\omega}[n] = \cos^{-1}\left(\frac{a[n]}{2}\right)\) approximates the derivative of this phase.
\end{itemize}

Hence, the KalmanANF reconstructs a rotating complex exponential component that behaves analogously to the output of a Hilbert-based frequency tracker.

\subsection{Frequency Resolution Limits}

The frequency resolution is limited by:
\begin{itemize}
    \item The sampling interval \(\Delta t\),
    \item The pole radius \(\rho\),
    \item The dynamics of the Kalman filter itself (i.e., process noise \(q\) and measurement noise \(r\)).
\end{itemize}

More specifically:
\begin{itemize}
    \item The **minimum detectable frequency** is limited by \(\frac{1}{T_{\text{obs}}}\), where \(T_{\text{obs}}\) is the observation duration.
    \item The **maximum resolvable frequency** is bounded by the Nyquist frequency \(f_{\text{Nyq}} = \frac{1}{2\Delta t}\). Frequencies beyond this will alias.
    \item The **bandwidth of the resonator** is approximately:
    \[
    BW \approx \frac{1 - \rho}{\pi \Delta t}
    \]
    Narrower bandwidth improves resolution but slows response to fast changes.
\end{itemize}

Thus, high frequency resolution requires:
\begin{itemize}
    \item High \(\rho\) close to 1,
    \item Small process noise \(q\),
    \item Long signal duration or window.
\end{itemize}

\subsection{Effect of Noise and Non-zero Mean}

The KalmanANF is designed under the assumption of zero-mean additive white noise. Deviations from this model degrade performance.

\subsubsection*{Additive White Gaussian Noise (AWGN)}
If the noise is white and zero-mean:
\begin{itemize}
    \item The Kalman gain \(K[n]\) adapts to maintain optimal estimation.
    \item Estimator converges to correct frequency and tracks small variations robustly.
\end{itemize}

\subsubsection*{Non-Zero Mean Noise}
If noise has non-zero mean (e.g., DC bias), the filter exhibits:
\begin{itemize}
    \item Bias in the resonator state \(s[n]\), distorting the harmonic trajectory.
    \item Skewed energy in the oscillator, leading to drift in phase and frequency.
    \item Possible convergence to a shifted frequency or spurious behavior.
\end{itemize}

\textbf{Mitigation strategies} include:
\begin{itemize}
    \item Preprocessing with high-pass filter to remove DC components.
    \item Adaptive DC cancellation or detrending.
\end{itemize}

\subsubsection*{Colored Noise}
If the noise has a spectral peak near the signal frequency:
\begin{itemize}
    \item The estimator may lock onto the noise peak instead of the true frequency.
    \item Increased process noise \(q\) may help the filter escape such spurious modes, at the cost of stability.
\end{itemize}

\subsubsection*{Robustness Summary}
The KalmanANF is fairly robust to zero-mean white noise. However:
\begin{itemize}
    \item DC offsets must be removed.
    \item Noise near the signal frequency limits tracking accuracy.
    \item Tuning of \(q\), \(r\), and \(\rho\) is critical for balance between tracking and stability.
\end{itemize}

\section{Discrete-Time Algorithm}

\subsection{State-Space and Kalman Update}
Define the internal filter state
\[
s[n] = y[n] + \rho\,a[n-1]\,s[n-1] - \rho^2\,s[n-2].
\]
Treat \(a[n]\) as the parameter to estimate.  At each step:

\begin{enumerate}
  \item \textbf{Prediction of Covariance:}
    \[
      P_{n|n-1} = P_{n-1} + q,
    \]
    where \(q\) is the process-noise covariance for \(a\).
  \item \textbf{Innovation:}
    \[
      e[n] = s[n] - a[n-1]\,s[n-1] + s[n-2].
    \]
  \item \textbf{Kalman Gain:}
    \[
      K[n] = \frac{s[n-1]\,P_{n|n-1}}{s[n-1]^2\,P_{n|n-1} + r},
    \]
    with \(r\) the measurement-noise covariance.
  \item \textbf{Parameter Update:}
    \[
      a[n] = a[n-1] + K[n]\,e[n],
      \quad
      P_n = \bigl(1 - K[n]\,s[n-1]\bigr)\,P_{n|n-1}.
    \]
  \item \textbf{Frequency Estimate:}
    \[
      \hat\omega[n]
      = \frac{1}{\Delta t}\arccos\!\Bigl(\frac{a[n]}{2}\Bigr),
      \quad
      \hat f[n] = \frac{\hat\omega[n]}{2\pi}.
    \]
  \item \textbf{State Shift:}
    \[
      s[n-2]\gets s[n-1], 
      \quad 
      s[n-1]\gets s[n].
    \]
\end{enumerate}

\subsection{Explanation of Variables}
\begin{itemize}
  \item \(s[n]\): internal notch-filtered output.  
  \item \(a[n]\): adaptive coefficient, ideally \(2\cos(\omega\,\Delta t)\).  
  \item \(P_n\): error covariance for \(a[n]\).  
  \item \(\rho\): notch-filter pole radius, \(0<\rho<1\).  
  \item \(q\), \(r\): process and measurement noise covariances.  
  \item \(\Delta t\): sampling interval.  
\end{itemize}

\section{Stability and Convergence}

\subsection{Kalman-Optimality and Steady-State}
Under standard Gaussian assumptions and persistence of excitation, the scalar Kalman update is mean-square optimal.  One shows
\[
\mathbb{E}\bigl[(a[n] - a_{\mathrm{true}})^2\bigr]
\;\longrightarrow\;
\frac{r}{s[n-1]^2 + r/q}
\quad\text{as }n\to\infty,
\]
so \(a[n]\to a_{\mathrm{true}} = 2\cos(\omega\,\Delta t)\) in expectation, with variance bounded by the noise ratio \(r/q\).

\subsection{Frequency and Phase Locking}
As \(a[n]\) converges, the notch-filter poles lock onto the signal’s frequency.  The estimate \(\hat\omega[n]\) thus converges to \(\omega\).  Moreover, the internal phase
\[
\phi[n] = \mathrm{atan2}\bigl(s[n-1],\,s[n-2]\bigr)
\]
locks to the true phase (modulo \(\pi\)).

\section{Extended Kalman Filter Properties and Adaptive Behavior}

\subsection{EKF Interpretation}

The KalmanANF is an instance of a scalar Extended Kalman Filter (EKF), operating on a nonlinear observation model embedded within a second-order resonator. The filter aims to estimate a single nonlinear parameter — the notch filter pole angle \( \omega \) — by adaptively updating the coefficient \( a[n] \approx 2 \cos(\omega \Delta t) \). The state-space model can be viewed as:

\begin{align*}
x[n] &= a[n] \\
s[n] &= y[n] + \rho\,s[n-1]\,x[n] - \rho^2 s[n-2] \\
e[n] &= s[n] - \hat{s}[n] = s[n] - \rho\,s[n-1]\,x[n] + \rho^2 s[n-2]
\end{align*}

Since the relationship between the coefficient \( a[n] \) and the measurement \( s[n] \) is nonlinear, a first-order linearization is performed around the prior estimate. This is the core idea of the EKF: it applies the Kalman filter framework to a system with nonlinear dynamics or measurements by linearizing the model at each step.

The Kalman gain \( K[n] \) is computed from the Jacobian of the measurement equation with respect to the parameter \( a[n] \). The gain modulates how much the error \( e[n] \) affects the update of the coefficient \( a[n] \), making the filter both adaptive and recursive.

\subsection{Adaptive Estimation Properties}

The KalmanANF exhibits several adaptive properties that make it effective for frequency tracking:

\begin{enumerate}
  \item \textbf{Online Adaptation:} The coefficient \( a[n] \) is updated in real time using the Kalman gain:
  \[
  a[n] = a[n-1] + K[n] \cdot e[n]
  \]
  This update corrects the estimated frequency based on instantaneous tracking error \( e[n] \), improving convergence and responsiveness.

  \item \textbf{Uncertainty-Weighted Updates:} The update is modulated by the error covariance \( p[n] \), which reflects the filter's confidence in its current estimate. If uncertainty is high (large \( p[n] \)), updates are larger; if the estimate is already trusted, the updates are small.

  \item \textbf{Noise-Aware Estimation:} Through the use of \( q \) (process noise) and \( r \) (measurement noise), the KalmanANF balances reactivity against noise sensitivity. These parameters control the filter’s memory and how much it trusts new observations.

  \item \textbf{Automatic Gain Scheduling:} The Kalman gain \( K[n] \) adapts over time based on the signal content and residual error. This is analogous to adaptive step-size in gradient descent, helping the filter quickly converge initially, then stabilize.

  \item \textbf{Embedded Model Adaptation:} Rather than estimating frequency directly, KalmanANF updates an internal model (a second-order resonator) whose behavior reflects the signal frequency. This leads to more robust performance, particularly when frequency is time-varying.
\end{enumerate}

\subsection{Comparison with Standard EKF}

Unlike general-purpose EKFs that track full state vectors, KalmanANF:
\begin{itemize}
    \item Tracks a single parameter \( a[n] \) with scalar covariance \( p[n] \),
    \item Embeds its dynamics into a recursive filter structure (not a state vector),
    \item Avoids matrix operations — all updates are scalar and efficient.
\end{itemize}

This structure makes KalmanANF highly efficient and well-suited for embedded and real-time applications, while retaining many of the strengths of a traditional EKF: local linearization, uncertainty modeling, and adaptive gain control.

\section{Numerical Implementation}

\subsection{Parameter Selection}
\begin{itemize}
  \item \(\rho\approx0.99\)–0.999 for a narrow notch (high resolution).  
  \item \(q\ll r\) yields smooth, slow adaptation.  
  \item \(q\gg r\) yields fast tracking but more jitter.  
  \item Initial \(P_0\) large (e.g.\ \(10^2\)) if \(a[0]\) is uncertain.  
\end{itemize}


\section{Noise Robustness}

The Kalman update balances process noise \(q\) and measurement noise \(r\).  The notch filter suppresses out-of-band noise so that the effective SNR for the update is improved, facilitating convergence even at low input SNR.

\begin{thebibliography}{9}
\bibitem{AliWaterschoot2023}
Randy Ali and Tom van Waterschoot,  
``A frequency tracker based on a Kalman filter update of a single parameter adaptive notch filter,''  
in \emph{Proceedings of the 26th International Conference on Digital Audio Effects (DAFx23)},  
Copenhagen, Denmark, September 2023.
\end{thebibliography}

\end{document}
