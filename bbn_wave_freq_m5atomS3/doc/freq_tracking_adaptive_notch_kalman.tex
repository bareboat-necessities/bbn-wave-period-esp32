\documentclass[12pt,letterpaper]{article}
\usepackage{amsmath}
\usepackage{amssymb}
\usepackage{mathrsfs}
\usepackage{graphicx}
\usepackage{hyperref}
\usepackage{cite}

\title{Comprehensive Analysis of a Kalman-Based Adaptive Notch Filter Frequency Tracker}
\author{}
\date{}

\begin{document}

\maketitle

\section{Introduction}
This work presents a frequency tracking algorithm based on a Kalman-updated adaptive notch filter, following the approach of Ali and van Waterschoot~\cite{AliWaterschoot2023}. The method estimates the instantaneous frequency of a noisy sinusoidal signal through recursive adaptation of a single parameter.

\section{Signal Model and Notch Filter}
\subsection{Signal Model}
The measured signal is assumed to be:
\begin{equation}
y(t) = A(t)\,\sin\bigl(\omega(t)\,t + \phi(t)\bigr) + \nu(t),
\label{eq:signal_model}
\end{equation}
where $A(t)$, $\phi(t)$, and $\omega(t)$ vary slowly, and $\nu(t)$ is zero-mean noise.

\subsection{Notch Filter}
A notch filter is a type of linear time-invariant (LTI) filter designed to attenuate or remove a specific frequency (and a very narrow band around it) from a signal while leaving other frequencies largely unaffected. It is characterized by its frequency response, which has a sharp dip (a "notch") at the target frequency.

\subsection{Notch Filter Structure}
The second-order digital notch filter has transfer function:
\begin{equation}
H(z^{-1};a) = \frac{1 - a\,z^{-1} + z^{-2}}{1 - \rho\,a\,z^{-1} + \rho^2\,z^{-2}},
\label{eq:notch_tf}
\end{equation}
where $a = 2\cos(\omega \Delta t)$ and $\rho \in (0,1)$ is the pole radius. The instantaneous frequency is recovered as:
\begin{equation}
\omega = \frac{1}{\Delta t}\arccos\!\left(\frac{a}{2}\right).
\label{eq:freq_estimate}
\end{equation}

\section{Continuous-Time Interpretation}
\subsection{Resonator Dynamics}
The resonator state $s(t)$ follows:
\begin{equation}
\ddot{s}(t) + 2\zeta(a)\omega(a)\dot{s}(t) + \omega(a)^2 s(t) = y(t),
\label{eq:resonator_dynamics}
\end{equation}
where:
\begin{align}
\omega(a) &= \frac{1}{\Delta t}\cos^{-1}\!\left(\frac{a}{2}\right), \label{eq:omega_def} \\
\zeta(a) &= -\frac{\ln\rho}{\sqrt{\pi^2 + (\ln\rho)^2}}. \label{eq:zeta_def}
\end{align}

\subsection{Energy Interpretation}
The resonator's energy evolves as:
\begin{equation}
\dot{E}(t) = \dot{s}(t)y(t) - 2\zeta\omega \dot{s}(t)^2,
\label{eq:energy}
\end{equation}
showing energy transfer from input to resonator at the resonant frequency.

\section{Discrete-Time Algorithm}
\subsection{State-Space Update}
\begin{enumerate}
\item \textbf{State Prediction}:
\begin{equation}
s[n] = y[n] + \rho a[n-1]s[n-1] - \rho^2 s[n-2].
\label{eq:state_update}
\end{equation}

\item \textbf{Innovation}:
\begin{equation}
e[n] = s[n] - \rho a[n-1]s[n-1] + \rho^2 s[n-2].
\label{eq:innovation}
\end{equation}

\item \textbf{Kalman Update}:
\begin{align}
K[n] &= \frac{\rho s[n-1] P_{n|n-1}}{(\rho s[n-1])^2 P_{n|n-1} + r}, \label{eq:kalman_gain} \\
a[n] &= a[n-1] + K[n]e[n], \label{eq:param_update} \\
P_n &= (1 - K[n]\rho s[n-1]) P_{n|n-1}. \label{eq:cov_update}
\end{align}

\item \textbf{Frequency Estimate}:
\begin{equation}
\hat{\omega}[n] = \frac{1}{\Delta t}\arccos\!\left(\frac{a[n]}{2}\right).
\label{eq:freq_update}
\end{equation}
\end{enumerate}

\subsection{Parameter Selection}
\begin{itemize}
\item Pole radius: $\rho \approx 0.99$ (narrow notch) to $0.999$ (high resolution)
\item Initial covariance: $P_0 = 10^2$ for uncertain initial $a[0]$
\end{itemize}

\section{Stability and Convergence}
\subsection{Steady-State Behavior}
The error covariance converges to:
\begin{equation}
\mathbb{E}\bigl[(a[n] - a_{\text{true}})^2\bigr] \to \frac{r}{s[n-1]^2 + r/q}.
\label{eq:steady_state}
\end{equation}

\subsection{Numerical Stability}
\begin{itemize}
\item Clamp $-2 < a[n] < 2$ to avoid $\arccos$ domain errors
\item Maintain $0 < P[n] < P_{\text{max}}$ to prevent divergence
\end{itemize}

\section{Performance and Limitations}
\subsection{Frequency Resolution}
The bandwidth is approximately:
\begin{equation}
BW \approx \frac{1 - \rho}{\pi \Delta t}.
\label{eq:bandwidth}
\end{equation}

\subsection{Noise Robustness}
\begin{itemize}
\item Optimal for zero-mean white noise
\item Requires DC removal for non-zero mean signals
\end{itemize}

\section{Extended Kalman Filter Interpretation}
The algorithm implements a scalar EKF with Jacobian:
\begin{equation}
H_a(t) = \frac{\partial h(s,a)}{\partial a} = \rho s(t).
\label{eq:jacobian}
\end{equation}

\section{Numerical Considerations}
\begin{itemize}
\item Use $\rho \leq 0.99$ for stability
\item Double precision recommended when $\rho > 0.99$
\end{itemize}

\section{Conclusion}
The Kalman-based adaptive notch filter provides computationally efficient frequency tracking with inherent noise robustness. The scalar EKF structure ensures stability while maintaining adaptation capability.

\begin{thebibliography}{9}
\bibitem{AliWaterschoot2023}
R. Ali and T. van Waterschoot, 
``A frequency tracker based on a Kalman filter update of a single parameter adaptive notch filter,'' 
in \emph{Proc. 26th Int. Conf. Digital Audio Effects (DAFx23)}, 
Copenhagen, Denmark, Sep. 2023.
\end{thebibliography}

\end{document}
