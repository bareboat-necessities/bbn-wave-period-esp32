\documentclass[12pt,letterpaper]{article}
\usepackage{amsmath,amssymb,amsthm,fullpage}
\usepackage{hyperref}
\usepackage{cite}

\newtheorem{theorem}{Theorem}

\title{Comprehensive Analysis of the Monotonic Wedge Algorithm for Sliding Window Extrema}
\date{}

\begin{document}
\maketitle

\begin{abstract}
Sliding‐window extrema are critical in real‐time financial analysis, sensor networks, and streaming algorithms.  Computing the minimum and maximum over a sliding window of length \(W\) in a data stream of length \(N\) is a fundamental operation in signal processing and time‐series analysis.  The monotonic wedge algorithm maintains two deques (one for minima, one for maxima) in amortized \(O(1)\) time per update, with worst‐case \(O(\log W)\) via a hybrid search strategy.  This paper presents a comprehensive analysis of its computational and memory usage complexity, including detailed memory‐utilization estimates, and compares it with alternative data structures.
\end{abstract}

\section{Problem Statement}
Given a sequence \(\{x_k\}_{k=1}^N\) and a fixed window size \(W\), define
\[
  m_k = \min_{i=k-W+1}^k x_i,\qquad
  M_k = \max_{i=k-W+1}^k x_i,
\]
for \(k \ge W\).  The goal is to update \(m_k\) and \(M_k\) online in sub-\(O(W)\) time per step.

\section{Monotonic Wedge Algorithm}
Maintain two deques:
\[
  \mathcal{D}_{\min},\quad \mathcal{D}_{\max},
\]
each storing pairs \((x_j,j)\) in index order but with monotonic values:
\[
  \mathcal{D}_{\min}\colon x_{j_1}\le x_{j_2}\le\cdots,\quad
  \mathcal{D}_{\max}\colon x_{j_1}\ge x_{j_2}\ge\cdots.
\]
At each time \(k\):
\begin{enumerate}
  \item \textbf{Prune back:}\\
    \texttt{while\,(!D\_min.empty() \&\& D\_min.back().x > x\_k) D\_min.pop\_back();}\\
    similarly for \texttt{D\_max} when its back value \(<x_k\).
  \item \textbf{Push:}\\
    \texttt{D\_min.push\_back(\{x\_k,k\});} and likewise for \texttt{D\_max}.
  \item \textbf{Prune front:}\\
    \texttt{while\,(!D\_min.empty() \&\& D\_min.front().i <= k-W) D\_min.pop\_front();}\\
    and same for \texttt{D\_max}.
  \item \textbf{Report:}\\
    \[
      m_k = D_{\min}.\texttt{front}().x,\qquad
      M_k = D_{\max}.\texttt{front}().x.
    \]
\end{enumerate}

\section{Time Complexity}
\subsection{Amortized \(O(1)\)}
\begin{theorem}
Over \(N\) updates, the total number of deque push and pop operations is \(O(N)\).
\end{theorem}
\begin{proof}
Each element is pushed exactly once, and can be popped at most once from the back and once from the front.  Hence the total operations \(\le3N\), so \(O(1)\) amortized per update.
\end{proof}

\subsection{Worst–Case \(O(\log W)\) Comparisons}
By using a small hybrid of linear scan plus binary search when pruning, each update does at most \(O(\log W)\) comparisons~\cite{Lemire2006}.

\section{Memory Utilization}
\subsection{Per‐Element Storage}
Each deque entry holds:
\begin{itemize}
  \item A value \(x_j\) (e.g.\ 8 bytes for a double).
  \item An index \(j\) (e.g.\ 4 bytes for a 32-bit int).
\end{itemize}
Payload per entry is \(\beta+\gamma\) bytes.

\subsection{Container Overhead}
A block-based deque typically uses pointer fields (\(p\) bytes each) and blocks of size \(B\).  Amortized overhead per entry is \(O(p/B)\).

\subsection{Total}
In the worst case each deque grows to \(W\) entries, so  
\[
  \text{Payload} = W(\beta+\gamma),\quad
  \text{Overhead} = O\bigl(p\,\lceil W/B\rceil\bigr).
\]

\section{Comparison to Alternatives}
Segment trees or two-heap approaches incur \(\Theta(\log W)\) per update and higher pointer overhead, while the monotonic wedge gives \(O(1)\) amortized updates with minimal extra cost.

\section{Conclusion}
The monotonic wedge algorithm achieves \(O(1)\) amortized update time, \(O(\log W)\) worst-case comparisons, and \(O(W)\) space with very low constants, making it ideal for high-throughput, resource-constrained streaming applications.

\begin{thebibliography}{9}
\bibitem{Lemire2006}
D.~T. Lemire, “Streaming maximum–minimum filter using no more than three comparisons per element,” \emph{arXiv:cs/0610046}, 2006.

\bibitem{Balster2016}
E.~Balster, “STL compatible monotonic wedge for fast rolling min/max,” GitHub repository, 2016. \url{https://github.com/EvanBalster/STL_mono_wedge}
\end{thebibliography}

\end{document}
