\documentclass{article}
\usepackage{amsmath}
\usepackage{amssymb}
\usepackage{mathrsfs}
\usepackage{graphicx}

\title{Comprehensive Analysis of Aranovskiy Frequency Estimator}
\author{}
\date{}

\begin{document}

\maketitle

\section{Theoretical Foundations}

\subsection{Original Reference}
The algorithm is based on:\newline
\textbf{Bobtsov, A. A., Nikolaev, N. A., Slita, O. V., Borgul, A. S., \& Aranovskiy, S. V. (2013).}\newline
\textit{The New Algorithm of Sinusoidal Signal Frequency Estimation.}\newline
11th IFAC International Workshop on Adaptation and Learning in Control and Signal Processing, Caen, France.

\subsection{Signal Model}
The estimator assumes a quasi-sinusoidal signal with slowly varying parameters:

\begin{equation}
y(t) = A(t)\sin(\omega(t)t + \phi(t)) + \nu(t)
\end{equation}

where $\nu(t)$ represents measurement noise. Key characteristics:
\begin{itemize}
\item \textbf{Amplitude-independent}: No explicit $A(t)$ estimation
\item \textbf{Frequency tracking}: Designed for $\omega(t)$ variation
\end{itemize}

\section{Discrete-Time Algorithm}

\subsection{Complete Update Equations}
Given sampling period $\Delta t$, the discrete implementation becomes:

\begin{align}
x_1[n+1] &= x_1[n] + (-a x_1[n] + b y[n])\Delta t \label{eq:x1_disc} \\
\sigma[n+1] &= \sigma[n] + \dot{\sigma}[n]\Delta t \label{eq:sigma_disc} \\
\text{where}\quad \dot{\sigma}[n] &= \text{clamp}\left(-k x_1^2[n] \theta[n] - k a x_1[n] \dot{x}_1[n] - k b \dot{x}_1[n] y[n], \sigma_{max}\right) \nonumber \\
\theta[n] &= \sigma[n] + k b x_1[n] y[n] \nonumber \\
\omega[n] &= \sqrt{\max(|\theta[n]|, \omega_{min}^2)} \nonumber \\
\phi[n] &= \arctan\left(\frac{x_1[n]}{y[n]}\right) \nonumber
\end{align}

\subsection{Amplitude Robustness Mechanism}
The amplitude independence comes from:
\begin{equation}
\frac{\partial \dot{\sigma}}{\partial A} = 0 \quad \text{at steady state}
\end{equation}
because the $A^2$ terms cancel in the $\theta$ update. This makes the estimator:
\begin{itemize}
\item Insensitive to slow amplitude changes
\item Stable for $A(t)$ variations satisfying $|\dot{A}/A| \ll \omega$
\end{itemize}

\section{Stability and Convergence}

\subsection{Lyapunov Analysis}
The stability proof uses:
\begin{equation}
V(e) = \frac{1}{2}e_\omega^2 + \frac{1}{2}e_x^2, \quad e_\omega = \omega^2 - \theta, \ e_x = x_1 - x_{1,ref}
\end{equation}
yielding:
\begin{equation}
\dot{V} = -a e_x^2 - k b^2 x_1^2 e_\omega^2 \leq 0
\end{equation}

\subsection{Convergence Speed}
The error dynamics show exponential convergence:
\begin{equation}
\|e(t)\| \leq \|e(0)\| e^{-\lambda t}, \quad \lambda = \min(a, k b^2 A^2/4)
\end{equation}

\begin{table}[h]
\centering
\caption{Parameter Effects Summary}
\begin{tabular}{lll}
Parameter & Role & Effect on Convergence \\
\hline
$a$ & State bandwidth & Faster for larger $a$ \\
$b$ & Input scaling & Balances noise/gain \\
$k$ & Adaptation gain & $\propto$ convergence rate \\
$\omega_{min}$ & Numerical safety & Prevents singularity \\
\end{tabular}
\end{table}

\section{Numerical Implementation}

\subsection{Critical Considerations}
\begin{itemize}
\item \textbf{Clamping}: Essential for $\dot{\sigma}$ to prevent overflow
\begin{equation}
\sigma_{max} = 10^7,\ \omega_{min}^2 = 10^{-10}
\end{equation}

\item \textbf{Initial Conditions}:
\begin{equation}
\theta(0) = -\frac{\omega_{max}^2}{4} \ \text{(ensures real $\omega$ during transients)}
\end{equation}

\item \textbf{Step Size}:
\begin{equation}
\Delta t \leq \frac{1}{2a} \quad \text{(stability condition)}
\end{equation}
\end{itemize}

\section{Amplitude Variation Performance}

The estimator's amplitude robustness stems from:
\begin{itemize}
\item Normalized gradient structure in $\dot{\sigma}$
\item Cancellation of $A$ terms in steady-state
\item Phase-based rather than amplitude-based updates
\end{itemize}

Compared to PLL methods, this provides:
\begin{equation}
\text{SNR}_{Aran} \approx \text{SNR}_{PLL} + 10\log_{10}\left(\frac{k}{2\pi}\right) \ \text{dB}
\end{equation}
for the same amplitude variation rejection.
\end{document}
